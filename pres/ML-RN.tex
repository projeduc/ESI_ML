% !TEX TS-program = pdflatex
% !TeX program = pdflatex
% !TEX encoding = UTF-8
% !TEX spellcheck = fr

\documentclass[xcolor=table]{beamer}

\def\accyear{2024-2025}
\institute{
	Laboratoire de la Communication dans les Systèmes Informatiques (LCSI)\\
	École  nationale Supérieure d'Informatique (ESI, ex. INI), Algiers, Algeria
}

\author[Abdelkrime Aries (\accyear)] %
{Abdelkrime Aries}

\titlegraphic{%
	\includegraphics[height=1cm]{../extra/logo/esi-logo.png}%
	\hspace*{2cm}%
	\includegraphics[height=1cm]{../extra/logo/lcsi-logo.png}%
	\hspace*{2cm}%
	\includegraphics[height=1cm]{../extra/logo/esi.ml.pdf}%
}%\hspace*{4.75cm}~

\date{Academic year: \accyear} %\today

\title[ML]{Machine Learning} 

\usefolder{../extra/beamer}

\usetheme{Karimml} % Antibes Boadilla Warsaw

\beamertemplatenavigationsymbolsempty

\usepackage{longfbox}

\newfboxstyle{kared}{padding=1pt,margin=1pt,border-radius=1ex,border-color=red}
\newfboxstyle{kgboxx}{padding=1pt,margin=1pt,border-radius=1ex,border-color=green}


\title[ML: Réseaux de neurones] %
{Machine Learning \\Les réseaux de neurones}  

\changegraphpath{../img/RN/}

\begin{document}

\begin{frame}
\frametitle{Les réseaux de neurones}
\framesubtitle{Motivation}

\begin{figure}
	\centering
	\hgraphpage[.8\textwidth]{neurone.png}
	\caption{Schéma d'un neurone biologique \cite{2017-cain}}
\end{figure}

\end{frame}

\begin{frame}
\frametitle{Les réseaux de neurones}
\framesubtitle{Plan}

\begin{multicols}{2}
	%	\small
	\tableofcontents
\end{multicols}
\end{frame}

\section{Neurone}

\begin{frame}
\frametitle{Les réseaux de neurones}
\framesubtitle{Neurone}

\begin{itemize}
	\item Rappelons \optword{La régression logistique}
	\item On fait la combinaison linéaire des valeurs des caractéristiques : ceci peut être vu comme le cumul des signaux
	\[z(x) = \theta_0 + \sum\limits_{j=1}^{N} \theta_j x_j\]
	\item Ensuite, on applique la fonction logistique sur cette somme : on l'appelle une \keyword{fonction d'activation} (envoyer le signal ou non selon un seuil) 
	\[\sigma(z) = \frac{1}{1 + e^{-z}}\] 
	\item Pour calculer l'erreur de classification, on utilise une \keyword{fonction de coût}
	\item Pour entrainer les paramètres, on utilise une \keyword{fonction d'optimisation}
\end{itemize}

\end{frame}

\begin{frame}
\frametitle{Les réseaux de neurones}
\framesubtitle{Neurone}

\hgraphpage[\textwidth]{neurone.pdf}

\end{frame}

\subsection{Le réseau}

\begin{frame}
\frametitle{Les réseaux de neurones : Neurone}
\framesubtitle{Le réseau : Motivation}

\begin{minipage}{0.59\textwidth}
\begin{itemize}
	\item Pour appliquer des fonctions complexes, un seul neurone n'est pas suffisant
	\item Par exemple, \expword{la fonction XOR avec deux variables}
	\item Si on fait une régression logistique (un neurone avec la fonction logistique comme fonction d'activation), l'algorithme essaye de séparer les échantillons avec une ligne de décision
	\item Quelque soit la ligne de décision, il y a au moins un échantillon mal-classé
\end{itemize}
\end{minipage}
\begin{minipage}{0.4\textwidth}
	\hgraphpage[\textwidth]{xor.png}
\end{minipage}

\end{frame}

\begin{frame}
\frametitle{Les réseaux de neurones : Neurone}
\framesubtitle{Le réseau : Architecture}

\begin{minipage}{0.59\textwidth}
	\begin{itemize}
		\item \optword{Couche d'entrée} : contient les entrées d'information, \expword{un peu comme les nerfs optiques de l'être humain} 
		\item \optword{Couches cachées} : chaque couche reçoit les sorties de la couche précédente. 
		Elle contient un nombre de neurones qui calculent la somme pondérée avec une fonction d'activation. 
		\item \optword{Couche de sortie} : contient les neurones qui calculent la sortie. 
	\end{itemize}
\end{minipage}
\begin{minipage}{0.4\textwidth}
	\hgraphpage[\textwidth]{RN.pdf}
\end{minipage}

\end{frame}

\subsection{Les fonctions d'activation}

\begin{frame}
\frametitle{Les réseaux de neurones : Neurone}
\framesubtitle{Les fonctions d'activation : Fonctions sigmoïde}

\begin{tabular}{ll}
	Fonction logistique & Tangente hyperbolique \\
	\hgraphpage[.4\textwidth]{logistique.png} & 
	\hgraphpage[.4\textwidth]{tanh.png} \\
	$\color{blue}\sigma(x) = \frac{1}{1 + e^{-x}}$ & 
	$\color{blue}tanh(x) = \frac{e^{x} - e^{-x}}{e^{x} + e^{-x}} = \frac{2}{1 + e^{-2x}} - 1$ \\
	
	$\sigma(x) \in ]0, 1[$ & 
	$tanh(x) \in ]-1, 1[$ \\
	
	$\color{red}\sigma'(x) = \sigma(x) (1-\sigma(x))$ & 
	$\color{red}tanh'(x) = 1 - tanh(x)^2$ \\
\end{tabular}

\end{frame}

\begin{frame}
\frametitle{Les réseaux de neurones : Neurone}
\framesubtitle{Les fonctions d'activation : Fonctions sigmoïde (2)}

\begin{tabular}{ll}
	Arc tangente & Signe doux \\
	\hgraphpage[.4\textwidth]{arctan.png} & 
	\hgraphpage[.4\textwidth]{so.png} \\
	$\color{blue}f(x) = \frac{2}{\pi} tan^{-1}(x)$ & 
	$\color{blue}f(x) = \frac{x}{1 + |x|}$ \\
	
	$f(x) \in ]-1, 1[$ & 
	$f(x) \in ]-1, 1[$ \\
	
	$\color{red}f'(x) = \frac{2}{\pi} \frac{1}{x^2 + 1}$ & 
	$\color{red}f'(x) = \frac{1}{(1 + |x|)^2}$ \\
\end{tabular}

\end{frame}

\begin{frame}
\frametitle{Les réseaux de neurones : Neurone}
\framesubtitle{Les fonctions d'activation : Fonctions linéaires}

\begin{tabular}{ll}
	Fonction linéaire & Fonction identité \\
	\hgraphpage[.4\textwidth]{lineaire.png} & 
	\hgraphpage[.4\textwidth]{identite.png} \\
	$\color{blue} f(x) = k x$ & 
	$\color{blue} f(x) = x$ \\
	
	$f(x) \in ]-\infty, +\infty[$ & 
	$f(x) \in ]-\infty, +\infty[$ \\
	
	$\color{red}f'(x) = k$ & 
	$\color{red}f'(x) = 1$ \\
\end{tabular}

\end{frame}

\begin{frame}
\frametitle{Les réseaux de neurones : Neurone}
\framesubtitle{Les fonctions d'activation : Fonctions échelon}

\begin{tabular}{ll}
	Fonction de Heaviside & Fonction signe \\
	\hgraphpage[.4\textwidth]{heaviside.png} & 
	\hgraphpage[.4\textwidth]{signe.png} \\
	$\color{blue} H(x) = \begin{cases}
	0 & \text{si } x < 0 \\
	1 & \text{sinon}
	\end{cases}$ & 
	$\color{blue} S(x) = \begin{cases}
	-1 & \text{si } x < 0 \\
	1 & \text{sinon}
	\end{cases} = 2 H(x) - 1$ \\
	
	$H(x) \in \{0, 1\}$ & 
	$S(x) \in \{-1, 1\}$ \\
	
	$\color{red}H'(x) = \begin{cases}
	0 & \text{si } x \ne 0 \\
	+\infty & \text{sinon}
	\end{cases}$ & 
	$\color{red}S'(x) = 2H'(x)$ \\
\end{tabular}

\end{frame}

\begin{frame}
\frametitle{Les réseaux de neurones : Neurone}
\framesubtitle{Les fonctions d'activation : Fonctions à bâton de hockey}

\begin{tabular}{ll}
	Unité de rectification linéaire & ReLU paramétrique \\
	\hgraphpage[.4\textwidth]{relu.png} & 
	\hgraphpage[.4\textwidth]{prelu.png} \\
	$\color{blue} ReLU(x) = x H(x) = \max(0, x)$ & 
	$\color{blue} PReLU(\alpha, x) = \begin{cases}
	\alpha x & \text{si } x < 0 \\
	x & \text{sinon}
	\end{cases}$ \\
	
	$ReLU(x) \in [0, +\infty[$ & 
	$PReLU(x) \in ]-\infty, +\infty[$ \\
	
	$\color{red}ReLU'(x) = H(x) = \begin{cases}
	0 & \text{si } x < 0 \\
	1 & \text{sinon}
	\end{cases}$ & 
	$\color{red}PReLU'(x) = \begin{cases}
	\alpha & \text{si } x < 0 \\
	1 & \text{sinon}
	\end{cases}$ \\
\end{tabular}

\end{frame}

\begin{frame}
\frametitle{Les réseaux de neurones : Neurone}
\framesubtitle{Les fonctions d'activation : Fonctions à bâton de hockey (2)}

\begin{tabular}{ll}
	Unité exponentielle linéaire & Unité sigmoïde linéaire \\
	\hgraphpage[.4\textwidth]{elu.png} & 
	\hgraphpage[.4\textwidth]{slu.png} \\
	$\color{blue} ELU(\alpha, x) = \begin{cases}
	\alpha (e^x - 1) & \text{si } x < 0 \\
	x & \text{sinon}
	\end{cases}$ & 
	$\color{blue} SLU(x) = x \sigma(x) = \frac{x}{1 + e^{-x}}$ \\
	
	$ELU(\alpha, x) \in ]-\infty, +\infty[$ & 
	$SLU(x) \in ]-\infty, +\infty[$ \\
	
	$\color{red}ELU'(\alpha, x) = \begin{cases}
	ELU(\alpha, x) + \alpha & \text{si } x < 0 \\
	1 & \text{sinon}
	\end{cases}$ & 
	$\color{red}SLU'(x) = \sigma(x) (1 + x \sigma(-x))$ \\
\end{tabular}

\end{frame}

\subsection{Les fonctions de coût}

\begin{frame}
\frametitle{Les réseaux de neurones : Neurone}
\framesubtitle{Les fonctions de coût : régression (MSE)}

%\begin{minipage}{0.68\textwidth}
\begin{columns}
	\begin{column}{.68\linewidth}
		\begin{block}{Erreur quadratique moyenne \\(mean squared error, L2 loss)}
			\[MSE = \frac{1}{2} (y - \hat{y})^2\]
			
			\[
			\frac{\partial MSE}{\partial \hat{y}} = \hat{y} - y
			\]
		\end{block}
		
		%\end{minipage}
		%\begin{minipage}{0.3\textwidth}
	\end{column}%
	\begin{column}{.3\linewidth}
		\begin{figure}
			\hgraphpage[\textwidth]{regression-loss_.pdf}
			\caption{Fonction du coût de régression \cite{2017-rosenberg}}
		\end{figure}
		%\end{minipage}
	\end{column}
\end{columns}


\end{frame}

\begin{frame}
\frametitle{Les réseaux de neurones : Neurone}
\framesubtitle{Les fonctions de coût : régression (MSE)}

\begin{itemize}
	\item \optword{Avantages}
	\begin{itemize}
		\item Une fonction quadratique n'a qu'un minimum global.
	\end{itemize}
	\item \optword{Inconvénients}
	\begin{itemize}
		\item MSE est moins robuste aux valeurs aberrantes.
	\end{itemize}
\end{itemize}

\end{frame}

\begin{frame}
\frametitle{Les réseaux de neurones : Neurone}
\framesubtitle{Les fonctions de coût : régression (MAE)}

%\begin{minipage}{0.59\textwidth}
\begin{columns}
	\begin{column}{.68\linewidth}
		\begin{block}{Erreur moyenne absolue \\(mean absolute error, L1 loss)}
			\[MAE = |y - \hat{y}|\]
			
			\[
			\frac{\partial MAE}{\partial \hat{y}} = 
			\begin{cases}
			+1 & \text{ si } \hat{y} > y \\
			-1 & \text{ si } \hat{y} < y \\
			[-1, +1] & \text{ si } \hat{y} = y \\
			\end{cases}
			\]
		\end{block}
	\end{column}%
	\begin{column}{.3\linewidth}
		%\end{minipage}
		%\begin{minipage}{0.4\textwidth}
		\begin{figure}
			\hgraphpage[\textwidth]{regression-loss_.pdf}
			\caption{Fonction du coût de régression \cite{2017-rosenberg}}
		\end{figure}
		%\end{minipage}
	\end{column}
\end{columns}

\end{frame}

\begin{frame}
\frametitle{Les réseaux de neurones : Neurone}
\framesubtitle{Les fonctions de coût : régression (MAE)}

\begin{itemize}
	\item \optword{Avantages}
	\begin{itemize}
		\item Moins sensible aux valeurs aberrantes
	\end{itemize}
	\item \optword{Inconvénients}
	\begin{itemize}
		\item L'amplitude du gradient ne dépend pas de la taille de l'erreur, seulement du signe de y - ŷ. Cela conduit à ce que la magnitude du gradient soit grande même lorsque l'erreur est faible
	\end{itemize}
\end{itemize}

\end{frame}


\begin{frame}
\frametitle{Les réseaux de neurones : Neurone}
\framesubtitle{Les fonctions de coût : régression (Huber)}

\begin{columns}
\begin{column}{.68\linewidth}
%\begin{minipage}{0.59\textwidth}
\begin{block}{Huber Loss}
	\[J = %\frac{1}{M} \sum\limits_{i=1}^{M} 
	\begin{cases}
	\frac{1}{2} (y - \hat{y})^2 & \text{si } |y - \hat{y}| \le \delta \\
	\delta |y - \hat{y}| - \frac{1}{2} \delta^2 & \text{sinon }\\
	\end{cases}
	\]
	
	\[
	\frac{\partial J}{\partial \hat{y}} = 
	\begin{cases}
	\hat{y} - y & \text{ si } |\hat{y} - y| \le \delta \\
	-\delta & \text{ si } \hat{y} - y < -\delta \\
	+\delta & \text{ si } \hat{y} - y > \delta \\
	\end{cases}
	\]
	
\end{block}

%\end{minipage}
\end{column}%
\begin{column}{.3\linewidth}
%\begin{minipage}{0.4\textwidth}
\begin{figure}
	\hgraphpage[\textwidth]{regression-loss_.pdf}
	\caption{Fonction du coût de régression \cite{2017-rosenberg}}
\end{figure}
%\end{minipage}
\end{column}
\end{columns}

\end{frame}

\begin{frame}
\frametitle{Les réseaux de neurones : Neurone}
\framesubtitle{Les fonctions de coût : régression (Huber)}

\begin{itemize}
	\item \optword{Avantages}
	\begin{itemize}
		\item Prendre les avantages des deux fonctions du coût : MSE et MAE
	\end{itemize}
	\item \optword{Inconvénients}
	\begin{itemize}
		\item Il faut entraîner l'hyper-paramètre $\delta$
	\end{itemize}
\end{itemize}

\end{frame}

\begin{frame}
\frametitle{Les réseaux de neurones : Neurone}
\framesubtitle{Les fonctions de coût : classement binaire (BCE)}

\begin{block}{Binary Cross Entropy Loss}
	\[J = - (y \log(\hat{y}) + (1 - y) \log(1 - \hat{y}))\]
	
	\[
	\frac{\partial J}{\partial \hat{y}} = \frac{\hat{y} - y}{\hat{y} - \hat{y}^2}
	\]
\end{block}

\begin{itemize}
	\item \optword{Avantages}
	\begin{itemize}
		\item marche bien avec les fonctions d'activation qui modélisent des probabilités (sigmoïde)
		\item Donc, meilleur estimation des probabilités
	\end{itemize}
	\item \optword{Inconvénients}
	\begin{itemize}
		\item la marge de séparation linéaire n'est pas garantie d'être optimale
	\end{itemize}
\end{itemize}

\end{frame}

\begin{frame}
\frametitle{Les réseaux de neurones : Neurone}
\framesubtitle{Les fonctions de coût : classement binaire (Hinge)}

\begin{block}{Hinge Loss (SVM Loss)}
	\[J = \max(0, 1 - y \hat{y}), y \in \{-1, 1\}\]
	
	\[
	\frac{\partial J}{\partial \hat{y}} = 
	\begin{cases}
	0 & \text{si } y \hat{y} \ge 1 \\
	-y & \text{sinon}
	\end{cases}
	\]
\end{block}

\begin{itemize}
	\item \optword{Avantages}
	\begin{itemize}
		\item entraîne une meilleure précision
	\end{itemize}
	\item \optword{Inconvénients}
	\begin{itemize}
		\item non lisse et non différenciable, donc moins d'algorithmes d'optimisation
		\item ne modélise pas la probabilité $p(y|x)$ directement
	\end{itemize}
\end{itemize}

\end{frame}

\begin{frame}
\frametitle{Les réseaux de neurones : Neurone}
\framesubtitle{Les fonctions de coût : classement multi-classes (Cross entropy)}

\begin{block}{Cross Entropy Loss}
	\[J = - \sum\limits_{k=1}^{K} y_{k} \log(\hat{y}_{k}) \text{, où } K \text{ est le nnombre des classes} \]
	
	\[
	\frac{\partial J}{\partial \hat{y}_k} = - \frac{y_{k}}{\hat{y}_{k}}
	\]
\end{block}

%\begin{itemize}
%	\item \optword{Avantages}
%	\begin{itemize}
%		\item 
%	\end{itemize}
%	\item \optword{Inconvénients}
%	\begin{itemize}
%		\item 
%	\end{itemize}
%\end{itemize}

\end{frame}

\subsection{Les fonctions d'optimisation}

\begin{frame}
\frametitle{Les réseaux de neurones : Neurone}
\framesubtitle{Les fonctions d'optimisation : descente du gradient}

\begin{figure}
	\vgraphpage[.7\textheight]{steepest.png}
	\caption{Illustration de la descente du gradient \cite{2020-calin}}
\end{figure}

\end{frame}

\begin{frame}
\frametitle{Les réseaux de neurones : Neurone}
\framesubtitle{Les fonctions d'optimisation : descente du gradient}

\begin{algorithm}[H]
	\KwData{$ X, Y, \alpha, T $}
	\KwResult{$ \theta $}
	initialiser $ \theta $ ; $ t = 0 $\;
	\While{$ t < T$ et pas de convergence}{
		$ \theta = \theta - \alpha \Delta_\theta J(X, Y; \theta) $\;
%		Mettre à jours les $\theta$\;
		t = t + 1 \;
	}
	\caption{descente du gradient}
\end{algorithm}

\end{frame}

\begin{frame}
\frametitle{Les réseaux de neurones : Neurone}
\framesubtitle{Les fonctions d'optimisation : descente du gradient}

\begin{itemize}
	\item \optword{Propriétés} : 
	\begin{itemize}
		\item Résoudre la valeur optimale dans le sens de la descente du gradient.
		\item La méthode converge avec un taux linéaire.
	\end{itemize}
	\item \optword{Avantages} : 
	\begin{itemize}
		\item La solution est un optimum global lorsque la fonction objective est convexe.
	\end{itemize}
	\item \optword{Inconvénients} : 
	\begin{itemize}
		\item Pour chaque mise à jours des paramètres, les gradients de tous les échantillons doivent être calculés. Donc, le coût de calcul est très élevé. 
		\item L'ensemble de l'entraînement peut être trop volumineux pour être traité dans la mémoire
		\item peut converger vers un minimum local lorsque la fonction objective n'est pas convexe
	\end{itemize}
\end{itemize}

\end{frame}

\begin{frame}
\frametitle{Les réseaux de neurones : Neurone}
\framesubtitle{Les fonctions d'optimisation : descente du gradient stochastique}

\begin{algorithm}[H]
	\KwData{$ X, Y, \alpha, T$}
	\KwResult{$ \theta $}
	initialiser $ \theta $ ; $ t = 0 $\;
	\While{$ t < T$ et pas de convergence}{
		Choisir $ (X^{(i)}, Y^{(i)}) \in (X, Y)$ aléatoirement\;
		$ \theta = \theta - \alpha \Delta_\theta J(X^{(i)}, Y^{(i)}; \theta) $\;
%		Mélanger les données aléatoirement\;
%		\ForEach{$ (X^{(i)}, Y^{(i)}) \in (X, Y)$}{
%			$ \theta = \theta - \alpha \Delta_\theta J(X^{(i)}, Y^{(i)}; \theta) $\;
%%			Mettre à jours les $\theta$\;
%		}
		t = t + 1 \;
	}
	\caption{descente du gradient stochastique}
\end{algorithm}

\end{frame}

\begin{frame}
\frametitle{Les réseaux de neurones : Neurone}
\framesubtitle{Les fonctions d'optimisation : descente du gradient stochastique}

\begin{itemize}
\item \optword{Propriétés} : 
\begin{itemize}
	\item La mise à jours des paramètres se fait par rapport chaque échantillon
\end{itemize}
\item \optword{Avantages} : 
\begin{itemize}
	\item Peut éviter des minimums locaux
\end{itemize}
\item \optword{Inconvénients} : 
\begin{itemize}
%	\item La mise à jour fréquente est plus coûteuse en terme de calcul
	\item Il est difficile au modèle à converger vers un minimum
\end{itemize}
\end{itemize}

\end{frame}


\begin{frame}
\frametitle{Les réseaux de neurones : Neurone}
\framesubtitle{Les fonctions d'optimisation : descente du gradient Mini-Batch}

\begin{algorithm}[H]
	\KwData{$ X, Y, \alpha, T, b $}
	\KwResult{$ \theta $}
	initialiser $ \theta $ ; $ t = 0 $\;
	\While{$ t < T$ et pas de convergence}{
		Mélanger les données aléatoirement\;
		Diviser les données sur $b$ batches\;
		\ForEach{$ (X^b, Y^b) \in batches$}{
			$ \theta = \theta - \alpha \Delta_\theta J(X^b, Y^b; \theta) $\;
%			Mettre à jours les $\theta$\;
		}
		\tcp{On peut mettre à jours les paramètres avec la moyenne des gradinets}
		t = t + 1 \;
	}
	\caption{descente du gradient Mini-Batch}
\end{algorithm}

\end{frame}

\begin{frame}
\frametitle{Les réseaux de neurones : Neurone}
\framesubtitle{Les fonctions d'optimisation : descente du gradient Mini-Batch}

\begin{itemize}
	\item \optword{Propriétés} : 
	\begin{itemize}
		\item L'apprentissage se fait sur des sous ensembles du dataset
		\item Un hyper-paramètre $b$ pour la taille d'un batch
	\end{itemize}
	\item \optword{Avantages} : 
	\begin{itemize}
		\item Peut éviter le minimum local à cause des mises à jours fréquents
		\item Efficace en cas des données d'apprentissage volumineux
	\end{itemize}
	\item \optword{Inconvénients} : 
	\begin{itemize}
		\item Il faut configurer un hyper-paramètre additionnel 
	\end{itemize}
\end{itemize}

\end{frame}


\begin{frame}
\frametitle{Les réseaux de neurones : Neurone}
\framesubtitle{Les fonctions d'optimisation : descente du gradient adaptative (AdaGrad)}

\begin{algorithm}[H]
	\KwData{$ X, Y, \alpha, T$}
	\KwResult{$ \theta $}
	initialiser $ \theta $ ; $ t = 0 $\;
	initialiser $v$ ($|v| = |\theta|$) à zéro\;
	\While{$ t < T$ et pas de convergence}{
		$ v = v + (\Delta_\theta J(X, Y; \theta))^2 $\;
		$ \theta = \theta - \frac{\alpha}{\sqrt{v + \epsilon}} \Delta_\theta J(X, Y; \theta) $\;
%		Mettre à jours les $\theta$\;
		\tcp{epsilon est utilisé pour éviter la division par 0, en général 1e-8}
		t = t + 1 \;
	}
	\caption{AdaGrad \cite{2011-duchi-al}}
\end{algorithm}

\end{frame}

\begin{frame}
\frametitle{Les réseaux de neurones : Neurone}
\framesubtitle{Les fonctions d'optimisation : descente du gradient adaptative (AdaGrad)}

\begin{itemize}
	\item \optword{Propriétés} : 
	\begin{itemize}
		\item Le taux d'apprentissage est ajusté de manière adaptative en fonction de la somme des carrés de tous les gradients historiques.
	\end{itemize}
	\item \optword{Avantages} : 
	\begin{itemize}
		\item Le taux d'apprentissage de chaque paramètre s'ajuste d'une façon adaptative
	\end{itemize}
	\item \optword{Inconvénients} : 
	\begin{itemize}
		\item Avec le temps, le taux d'apprentissage tombe vers zéro ; l'algorithme n'est pas souhaitable pour les fonctions non convexes.
	\end{itemize}
\end{itemize}

\end{frame}

\begin{frame}
\frametitle{Les réseaux de neurones : Neurone}
\framesubtitle{Les fonctions d'optimisation : Root Mean Square Propagation (RMSProp)}

\begin{algorithm}[H]
	\KwData{$ X, Y, \alpha, T, \beta$}
	\KwResult{$ \theta $}
	initialiser $ \theta $ ; $ t = 0 $\;
	initialiser $v$ ($|v| = |\theta|$) à zéro\;
	\While{$ t < T$ et pas de convergence}{
		$ v = \beta v + (1-\beta)(\Delta_\theta J(X, Y; \theta))^2 $\;
		$ \theta = \theta - \frac{\alpha}{\sqrt{v} + \epsilon} \Delta_\theta J(X, Y; \theta) $\;
%		Mettre à jours les $\theta$\;
		t = t + 1 \;
	}
	\caption{RMSProp}
\end{algorithm}

\end{frame}

\begin{frame}
\frametitle{Les réseaux de neurones : Neurone}
\framesubtitle{Les fonctions d'optimisation : Root Mean Square Propagation (RMSProp)}

\begin{itemize}
	\item \optword{Propriétés} : 
	\begin{itemize}
		\item Modifiez le mode d'accumulation totale du gradient en moyenne mobile exponentielle.
		\item Par défaut, $\beta = 0.9$ 
	\end{itemize}
	\item \optword{Avantages} : 
	\begin{itemize}
		\item Amélioration du problème d'apprentissage inefficace au stade avancé d'AdaGrad. 
		\item Il convient à l'optimisation de problèmes non stationnaires et non convexes.
	\end{itemize}
	\item \optword{Inconvénients} : 
	\begin{itemize}
		\item A la fin d'entraînement, le processus de mise à jour peut être répété autour du minimum local.
	\end{itemize}
\end{itemize}

\end{frame}

\begin{frame}
\frametitle{Les réseaux de neurones : Neurone}
\framesubtitle{Les fonctions d'optimisation : Adaptive Moment Estimation — Adam}

\begin{algorithm}[H]
	\KwData{$ X, Y, \alpha, T, \beta_1, \beta_2$}
	\KwResult{$ \theta $}
	initialiser $ \theta $ ; $ t = 0 $\;
	initialiser $v$ et $m$ ($|v| = |m| = |\theta|$) à zéro\;
	\While{$ t < T$ et pas de convergence}{
		$ g = \Delta_\theta J(X, Y; \theta) $\;
		$ m = \beta_1 m + (1-\beta_1) g $; $\hat{m} = \frac{m}{1-\beta_1^t}$\;
		$ v = \beta_2 v + (1-\beta_2) g^2 $;
		$\hat{v} = \frac{v}{1-\beta_2^t}$\;
		$ \theta = \theta - \frac{\alpha \hat{m}}{\sqrt{\hat{v}} + \epsilon} $\;
%		Mettre à jours les $\theta$\;
		t = t + 1 \;
	}
	\caption{Adam \cite{2015-kingma-ba}}
\end{algorithm}

\end{frame}

\begin{frame}
\frametitle{Les réseaux de neurones : Neurone}
\framesubtitle{Les fonctions d'optimisation : Adaptive Moment Estimation — Adam}

\begin{itemize}
	\item \optword{Propriétés} : 
	\begin{itemize}
		\item Utilisation du moment de premier ordre et du moment de second ordre pour ajuster dynamiquement le taux d'apprentissage de chaque paramètre. 
		\item Ajout de la correction de biais.
		\item Par défaut, $\beta_1 = 0.9, \beta_2 = 0.999$ 
	\end{itemize}
	\item \optword{Avantages} : 
	\begin{itemize}
		\item Le processus de descente de gradient est relativement stable. 
		\item Il convient à la plupart des problèmes d'optimisation non convexes avec de grands ensembles de données et un espace dimensionnel élevé.
	\end{itemize}
	\item \optword{Inconvénients} : 
	\begin{itemize}
		\item La méthode peut ne pas converger dans certains cas.
	\end{itemize}
\end{itemize}

\end{frame}

\section{Réseau de neurones à propagation avant}

\subsection{Architecture multicouches}

\begin{frame}
\frametitle{Les réseaux de neurones : à propagation avant}
\framesubtitle{Architecture multicouches : notation}

\hgraphpage[\textwidth]{RNPA.pdf}

\end{frame}

\begin{frame}
\frametitle{Les réseaux de neurones : à propagation avant}
\framesubtitle{Architecture multicouches : exemple}

\hgraphpage[\textwidth]{RNPA-exp.pdf}

\end{frame}

\begin{frame}
\frametitle{Les réseaux de neurones : à propagation avant}
\framesubtitle{Architecture multicouches : rétro-propagation [exemple] (1)}

%\begin{itemize}
%	\item On essaye de mettre à jour $w_{11}^{(3)}$
%\end{itemize}
On essaye de mettre à jour $w_{11}^{(4)}$

$ 
\frac{\partial J}{\partial w_{11}^{(4)}} = \overbrace{\frac{\partial J}{\partial f_{1}^{(4)}} \frac{\partial f_{1}^{(4)}}{\partial z_{1}^{(4)}}}^{\delta_{1}^{(4)}} \overbrace{\frac{\partial z_{1}^{(4)}}{\partial w_{11}^{(4)}}}^{a_{1}^{(3)}}
$
 
$ 
\frac{\partial J}{\partial f_{1}^{(4)}} = \frac{(0.840, 0.843) - (0, 1)}{(0.840, 0.843) - (0.840, 0.843)^2} 
= (6.25, -1.186)
$

$ 
\frac{\partial f_{1}^{(4)}}{\partial z_{1}^{(4)}} = (0.840, 0.843) (0.160, 0.157) = (0.134, 0.132)
$

$
\delta_{1}^{(4)} = (6.25, -1.186) (0.134, 0.132) \approx (0.838, -0.157)
$

$
\frac{\partial J}{\partial w_{11}^{(4)}} = moy((0.838, -0.157) (0.555, 0.612)) 
\approx moy(0.465, -0.096) = 0.184
$

$
w_{11}^{(4)} = 0.7 - 1 * (0.184) = 0.516 \text{ , supossant } \alpha = 1
$

\end{frame}

\begin{frame}
\frametitle{Les réseaux de neurones : à propagation avant}
\framesubtitle{Architecture multicouches : rétro-propagation [exemple] (2)}

%\begin{itemize}
%	\item On essaye de mettre à jour $w_{11}^{(3)}$
%\end{itemize}
On essaye de mettre à jour $w_{21}^{(4)}$

$ 
\frac{\partial J}{\partial w_{21}^{(4)}} = \overbrace{\frac{\partial J}{\partial f_{1}^{(4)}} \frac{\partial f_{1}^{(4)}}{\partial z_{1}^{(4)}}}^{\delta_{1}^{(4)}} \overbrace{\frac{\partial z_{1}^{(4)}}{\partial w_{21}^{(4)}}}^{a_{2}^{(3)}}
$


$
\delta_{1}^{(4)} = (0.838, -0.157)
$

$
\frac{\partial J}{\partial w_{21}^{(4)}} = moy((0.838, -0.157) (0.386, 0.360)) 
\approx moy(0.323, -0.056) = 0.134
$

$
w_{21}^{(4)} = 0.7 - 1 * (0.134) = 0.566 \text{ , supossant } \alpha = 1
$

\end{frame}

\begin{frame}
\frametitle{Les réseaux de neurones : à propagation avant}
\framesubtitle{Architecture multicouches : rétro-propagation [exemple] (3)}

On essaye de mettre à jour $w_{11}^{(3)}$

$
\frac{\partial J}{\partial w_{11}^{(3)}} = 
\overbrace{
	\overbrace{
		\frac{\partial J}{\partial f_{1}^{(4)}} 
		\frac{\partial f_{1}^{(4)}}{\partial z_{1}^{(4)}}
	}^{\delta_{1}^{(4)}} 
	\overbrace{
		\frac{\partial z_{1}^{(4)}}{\partial f_{1}^{(3)}}
	}^{w_{11}^{(4)}} 
	\frac{\partial f_{1}^{(3)}}{\partial z_{1}^{(3)}} 
}^{\delta_{1}^{(3)}} 
\overbrace{
	\frac{\partial z_{1}^{(3)}}{\partial w_{11}^{(3)}}
}^{a_{1}^{(2)}}
\text{ Ici, on utilise l'ancien } w_{11}^{(4)}
$

$
\frac{\partial f_{1}^{(3)}}{\partial z_{1}^{(3)}} \approx 
(0.555, 0.612) (0.445, 0.388) = (0.247, 0.237)
$

$
\delta_{1}^{(3)} = (0.838, -0.157) * 0.7 * (0.247, 0.237) \approx (0.145, -0.026)
$

$
\frac{\partial J}{\partial w_{11}^{(2)}} = moy((0.145, -0.026) (0.622, 0.900)) 
= moy(0.090, -0.023) = 0.033
$

$
w_{11}^{(2)} = 0.3 - 1 * (0.033) = 0.267 \text{ , supossant } \alpha = 1
$

\end{frame}


\begin{frame}
\frametitle{Les réseaux de neurones : à propagation avant}
\framesubtitle{Architecture multicouches : rétro-propagation [exemple] (4)}

On essaye de mettre à jour $w_{12}^{(3)}$

$
\frac{\partial J}{\partial w_{12}^{(3)}} = 
\overbrace{
	\overbrace{
		\frac{\partial J}{\partial f_{1}^{(4)}} 
		\frac{\partial f_{1}^{(4)}}{\partial z_{1}^{(4)}}
	}^{\delta_{1}^{(4)}} 
	\overbrace{
		\frac{\partial z_{1}^{(4)}}{\partial f_{2}^{(3)}}
	}^{w_{11}^{(4)}} 
	\frac{\partial f_{2}^{(3)}}{\partial z_{2}^{(3)}} 
}^{\delta_{2}^{(3)}} 
\overbrace{
	\frac{\partial z_{2}^{(3)}}{\partial w_{12}^{(3)}}
}^{a_{1}^{(2)}}
\text{ Ici, on utilise l'ancien } w_{12}^{(4)}
$

$
\frac{\partial f_{2}^{(3)}}{\partial z_{2}^{(3)}} \approx 
(0.386, 0.360) (0.614, 0.64) = (0.237, 0.230)
$

$
\delta_{2}^{(3)} = (0.838, -0.157) * 0.7 * (0.237, 0.230) \approx (0.139, -0.025)
$

$
\frac{\partial J}{\partial w_{11}^{(2)}} = moy((0.139, -0.025) (0.622, 0.900)) 
= moy(0.086, -0.023) = 0.032
$

$
w_{11}^{(2)} = - 0.1 - 1 * (0.032) = - 0.132 \text{ , supossant } \alpha = 1
$

\end{frame}


\begin{frame}
\frametitle{Les réseaux de neurones : à propagation avant}
\framesubtitle{Architecture multicouches : rétro-propagation [exemple] (5)}

On essaye de mettre à jour $w_{11}^{(2)}$

$
\frac{\partial J}{\partial w_{11}^{(2)}} = 
\overbrace{ 
\left(
	\delta_{1}^{(3)} 
	\overbrace{
		\frac{\partial z_{1}^{(3)}}{\partial f_{1}^{(2)}}
	}^{w_{11}^{(3)}}
	+
	\delta_{2}^{(3)} 
	\overbrace{
		\frac{\partial z_{2}^{(3)}}{\partial f_{1}^{(2)}}
	}^{w_{12}^{(3)}}
\right)
\frac{\partial f_{1}^{(2)}}{\partial z_{1}^{(2)}}
}^{\delta_{1}^{(2)}} 
\overbrace{
	\frac{\partial z_{1}^{(2)}}{\partial w_{11}^{(2)}}
}^{a_{1}^{(1)}}
$

$
\frac{\partial f_{1}^{(2)}}{\partial z_{1}^{(2)}} = (0.622, 0.900) (0.378, 0.100) = (0.235, 0.09)
$

$
\delta_{1}^{(2)} = \left((0.145, -0.026) * (0.3) + (0.139, -0.025) * (-0.1)\right) * (0.235, 0.09) \approx (0.006956, -0.00047683)
$


%$
%\frac{\partial J}{\partial w_{11}^{(2)}} = moy((0.039, -0.01) (2, 3)) 
%= moy(0.078, -0.03) = 0.024
%$
%
%$
%w_{11}^{(2)} = 0.5 - 1 * (0.024) = 0.476 \text{ , supossant } \alpha = 1
%$

\end{frame}

\begin{frame}
\frametitle{Les réseaux de neurones : à propagation avant}
\framesubtitle{Architecture multicouches : rétro-propagation [cas général]}

\begin{itemize}
	\item On calcule les $\delta^{(l)}$ où $l$ est le numéro de la couche
\end{itemize}

\[ 
\delta^{(sortie)} = 
\frac{\partial J}{\partial f^{(sortie)}} \frac{\partial f^{(sortie)}}{\partial z^{(sortie)}}
\,,\,
\delta^{(l)} = \frac{\partial f^{(l)}}{\partial z^{(l)}} w^{(l+1)} \delta^{(l+1)}
\]

\begin{itemize}
	\item On calcule les gradients
\end{itemize}

\[ 
\frac{\partial J}{\partial w^{(l)}} = a^{(l-1)} \delta^{(l)}
\,,\,
\frac{\partial J}{\partial b^{(l)}} = \delta^{(l)}
\]

\begin{itemize}
	\item On met à jour les paramètres
\end{itemize}

\[ 
w = w - \alpha \frac{\partial J}{\partial w^{(l)}}
\,,\,
b = b - \alpha \frac{\partial J}{\partial b^{(l)}}
\]


\end{frame}

\begin{frame}
\frametitle{Les réseaux de neurones : à propagation avant}
\framesubtitle{Architecture multicouches : rétro-propagation (un peu d'humour)}

\hgraphpage[.58\textwidth]{humour-retro1.jpeg}
\hgraphpage[.4\textwidth]{humour-retro2.jpg}


\end{frame}

\subsection{Auto-encodeur}

\begin{frame}
\frametitle{Les réseaux de neurones : à propagation avant}
\framesubtitle{Auto-encodeur}

\begin{itemize}
	\item C'est un réseau de neurones multicouches
	\item Il apprend un algorithme de compression
	\item Il est caractérisé par \cite{2016-keras} :
	\begin{itemize}
		\item axé sur les données : contrairement aux algorithmes de compression comme JPEG
		\item avec perte de données : la sortie construite n'est pas totalement identique à celle de l'entrée
		\item apprentissage non supervisé 
	\end{itemize}
	\item Il n'est pas vraiment bon pour la tâche de compression 
\end{itemize}

\end{frame}

\begin{frame}
\frametitle{Les réseaux de neurones : à propagation avant}
\framesubtitle{Auto-encodeur : Architecture}

\hgraphpage[\textwidth]{AE.pdf}

\end{frame}

\begin{frame}
\frametitle{Les réseaux de neurones : à propagation avant}
\framesubtitle{Auto-encodeur : Auto-encodeur débruiteur}

\hgraphpage[\textwidth]{AE-noise.pdf}

\end{frame}

\begin{frame}
\frametitle{Les réseaux de neurones : à propagation avant}
\framesubtitle{Auto-encodeur : Auto-encodeur débruiteur (quelques applications)}

\begin{itemize}
	\item Amélioration de la qualité du discours (son) \cite{2013-lu}
	\item Nettoyage des documents sales (voir cette compétition : \url{https://www.kaggle.com/c/denoising-dirty-documents})
	\item Récupération des documents historiques \cite{2019-neji}
	\item Complétion des parties cachées du visage \cite{2017-li-al}
\end{itemize}

\end{frame}

\begin{frame}
\frametitle{Les réseaux de neurones : à propagation avant}
\framesubtitle{Auto-encodeur : Auto-encodeur épars (sparse)}

\begin{itemize}
	\item Taille des couches cachées peut être égale ou plus que la taille de l'entrée 
	\item Il est utilisé pour apprendre automatiquement des caractéristiques (feature engineering) pour être utilisés dans une autre tâche (comme la classification)
	\item On applique une régularisation des paramètres (poids) d'une couche 
	\item Les régularisations utilisées sont \textbf{L1} et \textbf{la divergence de Kullback-Leibler}
	\item Certains poids convergent vers zéro ce qui permet d'apprendre des représentations comme par exemple le contours d'un objet dans une image
\end{itemize}

\end{frame}

\begin{frame}
\frametitle{Les réseaux de neurones : à propagation avant}
\framesubtitle{Auto-encodeur : Auto-encodeur variationnel}

\begin{itemize}
	\item Pour générer du nouveau contenu, on peut passer au décodeur une variable (vecteur) dans l'espace latent
	\item Le problème est qu'un encodeur apprend à séparer les différentes clusters parfaitement  
	\item Solution : forcer l'auto-encodeur à apprendre une distribution  
\end{itemize}

\begin{minipage}{0.60\textwidth} 
	\begin{align*}
	z = \mu + \sigma * N(0, 1) \\
	J'(x, \hat{x}) = J(x, \hat{x}) + KL(N(\mu, \sigma), N(0, 1)) \\
	KL(p||q) = \sum_i p(x_i) log(\frac{p(x_i)}{q(x_i)})
	\end{align*}
\end{minipage}
%
\begin{minipage}{0.39\textwidth}
	\hgraphpage[\textwidth]{AE-var.pdf}
\end{minipage}

\end{frame}

\begin{frame}
\frametitle{Les réseaux de neurones : à propagation avant}
\framesubtitle{Auto-encodeur : Auto-encodeur variationnel (Récapitulatif)}

\begin{itemize}
	\item L'auto-encodeur standard apprend une séparation parfaite des clusters (représentation des états latents non lisse)
	\item Solution : forcer le modèle à apprendre une distribution (en général, similaire à la distribution normale)
	\item Le modèle peut apprendre des distributions étroites 
	\item Solution : utiliser une régularisation (Divergence KL)
	\item Le modèle peut apprendre des valeurs négatives pour $\sigma$
	\item Solution : apprendre $\log \sigma$
\end{itemize}

\end{frame}

\begin{frame}
\frametitle{Les réseaux de neurones : à propagation avant}
\framesubtitle{Auto-encodeur : Un peu d'humour}

\hgraphpage[\textwidth]{humour-unsupervised.jpg}

\end{frame}

\subsection{Réseau neuronal convolutif}

\begin{frame}
\frametitle{Les réseaux de neurones : à propagation avant}
\framesubtitle{Réseau neuronal convolutif : traitement d'image (exemple d'une architecture traditionnelle)}

\begin{center}
	\hgraphpage[\textwidth]{img-learn.pdf}
\end{center}

\end{frame}

\begin{frame}
\frametitle{Les réseaux de neurones : à propagation avant}
\framesubtitle{Réseau neuronal convolutif : traitement d'image (prétraitement)}

\begin{itemize}
	\item Il existe plusieurs techniques de prétraitement (voir \cite{2010-alginahi})
	\item On s'intéresse à celles par convolution (modifier la valeur d'un pixel par rapport à ces voisins)
	\item deux paramètres : \optword{padding} (entourer l'image avec des 0 afin de préserver la taille originale), \optword{stride} (le pas de défilement du noyau/masque)
\end{itemize}

\begin{center}
	\hgraphpage[.8\textwidth]{conv.pdf}
\end{center}

\end{frame}

\begin{frame}
\frametitle{Les réseaux de neurones : à propagation avant}
\framesubtitle{Réseau neuronal convolutif : traitement d'image (exemple de prétraitement)}

Un exemple de \href{https://fr.wikipedia.org/wiki/Noyau\_(traitement\_d\%27image)}{Wikipedia}

\begin{tabular}{p{.3\textwidth}p{.1\textwidth}p{.2\textwidth}p{.2\textwidth}}
	\hline\hline
	Opération & originale & Noyau & transformée \\
	\hline
	La détection de contours & 
	\graphpage[valign=c]{Vd-Orig.png} & 
	$\begin{bmatrix}
	-1 & -1 & -1\\ 
	-1 & 8 & -1\\ 
	-1 & -1 & -1
	\end{bmatrix}$ & 
	\graphpage[valign=c]{Vd-Edge3.png} \\
	
	\hline
	Amélioration de la netteté & 
	\graphpage[valign=c]{Vd-Orig.png} & 
	$\begin{bmatrix}
	0 & -1 & 0\\ 
	-1 & 5 & -1\\ 
	0 & -1 & 0
	\end{bmatrix}$ & 
	\graphpage[valign=c]{Vd-Sharp.png} \\
	
	\hline
	Box blur & 
	\graphpage[valign=c]{Vd-Orig.png} & 
	$\frac{1}{9}\begin{bmatrix}
	1 & 1 & 1\\ 
	1 & 1 & 1\\ 
	1 & 1 & 1
	\end{bmatrix}$ & 
	\graphpage[valign=c]{Vd-Blur2.png} \\
	\hline\hline
	
\end{tabular}

\end{frame}

\begin{frame}
\frametitle{Les réseaux de neurones : à propagation avant}
\framesubtitle{Réseau neuronal convolutif : limites de la solution précédente}

D'après LeCun et ses collèques \cite{1998-lecun} : 
\begin{itemize}
	\item Lorsque la taille des images est grande; on aura plusieurs paramètres à entraîner
	\item Pour le faire, on doit fournir un grand dataset 
	\item La mémoire pour stocker les paramètres sera très grande 
	\item Pour entraîner le modèle sur les images, on doit faire des prétraitements comme les translations et les distorsions
	\item Les variations dans les images (comme par exemple la position d'un objet) ne peuvent pas être capturées que lorsqu'on utilise plusieurs couches
\end{itemize}

\end{frame}

\begin{frame}
\frametitle{Les réseaux de neurones : à propagation avant}
\framesubtitle{Réseau neuronal convolutif : Couche Conv2D}

\begin{minipage}{0.60\textwidth} 
	\begin{itemize}
		\item On préserve la structure spatiale des images
		\item La couche apprend un ou plusieurs noyaux
		\item $ w' = \frac{w - w_f + 2P}{S} + 1$,  $ h' = \frac{h - h_f + 2P}{S} + 1$
		\item On peut spécifier le nombres des filtres $k$ à générer 
		\item Le nombre des paramètres sera $w_f * h_f * c * k$ plus $k$ biais
		\item Exemple, \expword{image : 32x32x3; noyau : 5x5; s : 1; p: 0; k : 6.} Dans ce cas, on aura \expword{81} paramètres à entraîner
	\end{itemize}
\end{minipage}
%
\begin{minipage}{0.39\textwidth}
	\hgraphpage[\textwidth]{conv2d.pdf}
	
	\hgraphpage[.8\textwidth]{conv2d-exp_.pdf}
\end{minipage}

\end{frame}

\begin{frame}
\frametitle{Les réseaux de neurones : à propagation avant}
\framesubtitle{Réseau neuronal convolutif : Pooling}

\begin{minipage}{0.60\textwidth} 
	\begin{itemize}
		\item Rendre la représentation plus petite et plus gérable
		\item $ w' = \frac{w - w_f + 2P}{S} + 1$,  $ h' = \frac{h - h_f + 2P}{S} + 1$
		\item Pas de paramètres 
		\item \optword{Max Pool} : le gradient est passé seulement à la cellule gagnante (qui a le max) 
		\item \optword{Average Pool} : le gradient passé aux cellules participantes à la moyennes est $\frac{\text{gradient}}{w_f * h_f}$ 
	\end{itemize}
\end{minipage}
%
\begin{minipage}{0.39\textwidth}
	\hgraphpage[\textwidth]{maxpool.pdf}
	
	\hgraphpage[.8\textwidth]{pool-exp_.pdf}
\end{minipage}

\end{frame}

\begin{frame}
\frametitle{Les réseaux de neurones : à propagation avant}
\framesubtitle{Réseau neuronal convolutif : Un peu d'humour}

\hgraphpage[\textwidth]{humour-convnet.jpg}

\end{frame}

\subsection{Régularisation}

\begin{frame}
\frametitle{Les réseaux de neurones : à propagation avant}
\framesubtitle{Régularisation : Abandon (Droupout)}

\begin{itemize}
	\item Lorsqu'on entraîne un réseaux de neurones larges sur un petit dataset, on peut avoir un sur-apprentissage
	\item Une solution est de diminuer la taille du réseau 
	\item Une autre est de désactiver des connections temporairement d'une manière aléatoire
	\item Cette technique s'appelle \optword{le décrochage} ou \optword{l'abandon}
	\item La couche d'abandon prend comme paramètre une fréquence d'abandon
	\item Cette couche désactive certains sorties (les considère comme 0) pendant l'entraînement (pas l'inférence)
\end{itemize}

\end{frame}


\begin{frame}
\frametitle{Les réseaux de neurones : à propagation avant}
\framesubtitle{Régularisation : Un peu d'humour}

\hgraphpage[.35\textwidth]{humour-overfiting2.jpeg}
\hgraphpage[.63\textwidth]{humour-overfiting1.png}

\end{frame}


\section{Réseaux de neurones récurrents}


\subsection{Architecture (RNN)}

\begin{frame}
\frametitle{Les réseaux de neurones : récurrents}
\framesubtitle{Architecture (RNN)}

\begin{minipage}{0.49\textwidth} 
	\begin{itemize}
		\item \optword{Elman network}
		\begin{align*}
		h_t = f(w_x x_t + w_h \textcolor{red}{h_{t-1}} + b_h) \\
		\hat{y}_t = g(w_y h_t + b_y)
		\end{align*}
		\item \optword{Jordan network}
		\begin{align*}
		h_t = f(w_x x_t + w_h \textcolor{red}{\hat{y}_{t-1}} + b_h) \\
		\hat{y}_t = g(w_y h_t + b_y)
		\end{align*}
	\end{itemize}
\end{minipage}
%
\begin{minipage}{0.5\textwidth}
	\hgraphpage[\textwidth]{RNN.pdf}
\end{minipage}

\end{frame}

\begin{frame}
\frametitle{Les réseaux de neurones : récurrents}
\framesubtitle{Architecture (RNN) : Rétro-propagation}

\begin{itemize}
	\item Définir la fonction du coût de chaque sortie
	\item Utiliser la rétro-propagation dans le temps \cite{1990-werbos}  
	\begin{itemize}
		\item Déplier le réseau récurrent sur le temps 
		\item Le réseau sera similaire à celui de type feed-forward 
		\item Cumuler les gradients en commençant par le dernier état 
		\item Mettre à jour les paramètres lorsqu'on arrive au premier état
	\end{itemize}
\end{itemize}

%\begin{algorithm}[H]
%	\KwData{$ X, Y, \alpha, T $}
%	\KwResult{Les paramètres}
%	a = []\;
%	\For{$ t = 0 $ jusqu'à $T-1$}{
%		Les entrées sont $a,\, x[0],\, ...\, x[t]$ 
%		$ \theta = \theta - \alpha \Delta_\theta J(X, Y; \theta) $\;
%		%		Mettre à jours les $\theta$\;
%		t = t + 1 \;
%	}
%	\caption{Rétropropagation dans le temps (BPTT) \cite{1990-werbos}}
%\end{algorithm}

\end{frame}

\begin{frame}
\frametitle{Les réseaux de neurones : récurrents}
\framesubtitle{Architecture (RNN) : Applications}

\begin{tabular}{p{.32\textwidth}p{.15\textwidth}p{.4\textwidth}}
	\hline\hline
	Type & Illustration & Exemple \\
	\hline
	Plusieurs à plusieurs & 
	\vgraphpage[1.5cm, valign=c]{RNNpp1.pdf} & 
	Détection d'entités nommées \\
	
	\hline
	Plusieurs à plusieurs & 
	\vgraphpage[1.5cm, valign=c]{RNNpp2.pdf} & 
	Traduction automatique \\
	
	\hline
	Plusieurs à un & 
	\vgraphpage[1.5cm, valign=c]{RNNp1.pdf} & 
	Classification de sentiments \\
	
	\hline
	Un à plusieurs & 
	\vgraphpage[1.5cm, valign=c]{RNN1p.pdf} & 
	Sous-titrage d'images \\
	
	\hline\hline
	
\end{tabular}


\end{frame}


\begin{frame}
	\frametitle{Les réseaux de neurones : récurrents}
	\framesubtitle{Architecture (RNN) : Applications (plusieurs à plusieurs direct)}
	\vspace*{-0.5cm}
	\[ h_t = f(x_t, h_{t-1}, W_x, W_h),\,\,\, \hat{y}_t = g(h_t, W_y),\,\,\,  J(\hat{y}, y) = \textcolor{red}{[\frac{1}{T}]?} \sum_{t=1}^{T} j(\hat{y}_t, y_t)\]
	
	\vspace*{-0.5cm}
	\begin{minipage}{0.7\textwidth}\scriptsize
	\[\frac{\partial J}{\partial W_h} = \sum_{t=1}^{T} \frac{\partial j(\hat{y}_t, y_t)}{\partial W_h}
	= \sum_{t=1}^{T} \frac{\partial j(\hat{y}_t, y_t)}{\partial \hat{y}_t} 
	\frac{\partial g(h_t, W_y)}{\partial h_t} 
	\frac{\partial h_t}{\partial W_h}
	\]
	\[\frac{\partial h_t}{\partial W_h} = 
	\frac{\partial f(x_t, h_{t-1}, W_x, W_h)}{\partial W_h} + 
	\frac{\partial f(x_t, h_{t-1}, W_x, W_h)}{\partial h_{t-1}} \frac{\partial h_{t-1}}{\partial W_h}
	\]
	\[\frac{\partial J}{\partial W_y} 
	= \sum_{t=1}^{T} \frac{\partial j(\hat{y}_t, y_t)}{\partial W_y}
	= \sum_{t=1}^{T} \frac{\partial j(\hat{y}_t, y_t)}{\partial \hat{y}_t} 
	\frac{\partial g(h_t, W_y)}{\partial W_y}
	\]
	\[\frac{\partial J}{\partial W_x} = \sum_{t=1}^{T} \frac{\partial j(\hat{y}_t, y_t)}{\partial W_x}
	= \sum_{t=1}^{T} \frac{\partial j(\hat{y}_t, y_t)}{\partial \hat{y}_t} 
	\frac{\partial g(h_t, W_y)}{\partial h_t} 
	\frac{\partial h_t}{\partial W_x}
	\]
	\[\frac{\partial h_t}{\partial W_x} = 
	\frac{\partial f(x_t, h_{t-1}, W_x, W_h)}{\partial W_x} + 
	\frac{\partial f(x_t, h_{t-1}, W_x, W_h)}{\partial h_{t-1}} \frac{\partial h_{t-1}}{\partial W_x}
	\]
	\end{minipage}
	\begin{minipage}{0.2\textwidth}
		\hgraphpage[1.5\textwidth]{RNNpp1_exp.pdf}
	\end{minipage}
	
\end{frame}

\begin{frame}
	\frametitle{Les réseaux de neurones : récurrents}
	\framesubtitle{Architecture (RNN) : Applications (plusieurs à plusieurs indirect)}
	
	\begin{itemize}
		\item Ici, deux cellules LSTMs sont utilisées : encodeur et décodeur
	\end{itemize}

	\[ h^e_t = f^e(x_t, h^e_{t-1}, W^e_x, W^e_h), \,\,\, \hat{y}_1 = g^e(h^e_{T1}, W_y)\]
	\[ h^d_{t} = f^d(\hat{y}_{t}, h^d_{t-1}, W^d_x, W^d_h), \,\,\, \hat{y}_{t+1} = g^d(h^d_{t}, W^d_y)\]
	\[ J(\hat{y}, y) = \textcolor{red}{[\frac{1}{T2}]?} \sum_{t=1}^{T2} j(\hat{y}_{t}, y_t)\]
	\begin{center}
		\hgraphpage[0.5\textwidth]{RNNpp2_exp.pdf}
	\end{center}
\end{frame}

\begin{frame}
	\frametitle{Les réseaux de neurones : récurrents}
	\framesubtitle{Architecture (RNN) : Applications (plusieurs à un)}
	
	\vspace*{-0.5cm}
	\[ h_t = f(x_t, h_{t-1}, W_x, W_h),\,\,\, \hat{y}_t = g(h_t, W_y),\,\,\,  J(\hat{y}, y) = j(\hat{y}_T, y)\]
	
%	\vspace*{-0.5cm}
	\begin{minipage}{0.7\textwidth}\scriptsize
		\[\frac{\partial J}{\partial W_h} = \frac{\partial j(\hat{y}_T, y)}{\partial W_h}
		= \frac{\partial j(\hat{y}_T, y)}{\partial \hat{y}_T} 
		\frac{\partial g(h_T, W_y)}{\partial h_T} 
		\frac{\partial h_T}{\partial W_h}
		\]
		\[\frac{\partial h_t}{\partial W_h} = 
		\frac{\partial f(x_t, h_{t-1}, W_x, W_h)}{\partial W_h} + 
		\frac{\partial f(x_t, h_{t-1}, W_x, W_h)}{\partial h_{t-1}} \frac{\partial h_{t-1}}{\partial W_h}
		\]
%		\[\frac{\partial J}{\partial W_y} 
%		= \sum_{t=1}^{T} \frac{\partial j(\hat{y}_t, y_t)}{\partial W_y}
%		= \sum_{t=1}^{T} \frac{\partial j(\hat{y}_t, y_t)}{\partial \hat{y}_t} 
%		\frac{\partial g(h_t, W_y)}{\partial W_y}
%		\]
%		\[\frac{\partial J}{\partial W_x} = \sum_{t=1}^{T} \frac{\partial j(\hat{y}_t, y_t)}{\partial W_x}
%		= \sum_{t=1}^{T} \frac{\partial j(\hat{y}_t, y_t)}{\partial \hat{y}_t} 
%		\frac{\partial g(h_t, W_y)}{\partial h_t} 
%		\frac{\partial h_t}{\partial W_x}
%		\]
%		\[\frac{\partial h_t}{\partial W_x} = 
%		\frac{\partial f(x_t, h_{t-1}, W_x, W_h)}{\partial W_x} + 
%		\frac{\partial f(x_t, h_{t-1}, W_x, W_h)}{\partial h_{t-1}} \frac{\partial h_{t-1}}{\partial W_x}
%		\]
	\end{minipage}
	\begin{minipage}{0.2\textwidth}
		\hgraphpage[1.5\textwidth]{RNNp1_exp.pdf}
	\end{minipage}

\end{frame}

\begin{frame}
	\frametitle{Les réseaux de neurones : récurrents}
	\framesubtitle{Architecture (RNN) : Applications (un à plusieurs)}
	
	\vspace*{-0.5cm}
	\[ h^e = f^e(x, h^e_{0}, W^e_x, W^e_h), \,\,\, \hat{y}_1 = g^e(h^e, W_y)\]
	\[ h^d_{t} = f^d(\hat{y}_{t}, h^d_{t-1}, W^d_x, W^d_h), \,\,\, \hat{y}_{t+1} = g^d(h^d_{t}, W^d_y)\]
	\[ J(\hat{y}, y) = \textcolor{red}{[\frac{1}{T}]?} \sum_{t=1}^{T} j(\hat{y}_{t}, y_t)\]
	
%	\vspace*{-0.5cm}
%	\begin{minipage}{0.7\textwidth}\scriptsize
%		\[\frac{\partial J}{\partial W_h} = \sum_{t=1}^{T} \frac{\partial j(\hat{y}_t, y_t)}{\partial W_h}
%		= \sum_{t=1}^{T} \frac{\partial j(\hat{y}_t, y_t)}{\partial \hat{y}_t} 
%		\frac{\partial g(h_t, W_y)}{\partial h_t} 
%		\frac{\partial h_t}{\partial W_h}
%		\]
%		\[\frac{\partial h_t}{\partial W_h} = 
%		\frac{\partial f(x_t, h_{t-1}, W_x, W_h)}{\partial W_h} + 
%		\frac{\partial f(x_t, h_{t-1}, W_x, W_h)}{\partial h_{t-1}} \frac{\partial h_{t-1}}{\partial W_h}
%		\]
%		\[\frac{\partial J}{\partial W_y} 
%		= \sum_{t=1}^{T} \frac{\partial j(\hat{y}_t, y_t)}{\partial W_y}
%		= \sum_{t=1}^{T} \frac{\partial j(\hat{y}_t, y_t)}{\partial \hat{y}_t} 
%		\frac{\partial g(h_t, W_y)}{\partial W_y}
%		\]
%		\[\frac{\partial J}{\partial W_x} = \sum_{t=1}^{T} \frac{\partial j(\hat{y}_t, y_t)}{\partial W_x}
%		= \sum_{t=1}^{T} \frac{\partial j(\hat{y}_t, y_t)}{\partial \hat{y}_t} 
%		\frac{\partial g(h_t, W_y)}{\partial h_t} 
%		\frac{\partial h_t}{\partial W_x}
%		\]
%		\[\frac{\partial h_t}{\partial W_x} = 
%		\frac{\partial f(x_t, h_{t-1}, W_x, W_h)}{\partial W_x} + 
%		\frac{\partial f(x_t, h_{t-1}, W_x, W_h)}{\partial h_{t-1}} \frac{\partial h_{t-1}}{\partial W_x}
%		\]
%	\end{minipage}
%	\begin{minipage}{0.2\textwidth}
%		\hgraphpage[1.5\textwidth]{RNN1p_exp.pdf}
%	\end{minipage}
\begin{center}
	\hgraphpage[0.4\textwidth]{RNN1p_exp.pdf}
\end{center}

\end{frame}

\subsection{Long Short-Term Memory (LSTM)}

\begin{frame}
\frametitle{Les réseaux de neurones : récurrents}
\framesubtitle{Long Short-Term Memory (LSTM) : architecture}

\begin{minipage}{0.50\textwidth} 
	\begin{itemize}
		\item $i$ : \optword{Porte d'entrée (actualisation)} ;
		Combien d'information doit-on ajouter à l'état interne de la cellule ?
		\item $f$ : \optword{Porte d'oubli} ;
		Est-ce qu'on oubli le contexte passé ?
		\item $o$ : \optword{Porte de sortie} ;
		Sortie générée à partir de l'état interne et actuel de la cellule	
	\end{itemize}
\end{minipage}
%
\begin{minipage}{0.49\textwidth}
	\hgraphpage[\textwidth]{LSTM.pdf}
\end{minipage}

\begin{itemize}
	\item $g$ : \optword{Porte de modulation d'entrée} ;
	Est-ce qu'on augmente ou diminuer l'état ?
	(souvent considérée comme partie de la porte d'entrée)
%	\item La dérivée de l'erreur sera ue somme de la porte d'oubli avec d'autres dérivées
\end{itemize}

\end{frame}

\begin{frame}
	\frametitle{Les réseaux de neurones : récurrents}
	\framesubtitle{Long Short-Term Memory (LSTM) : équations}
	
	\begin{align*}
	f_t &= \sigma(W_f X_t + U_f h_{t-1} + b_f) \\
	i_t &= \sigma(W_i X_t + U_i h_{t-1} + b_i) \\
	g_t &= \tanh(W_g X_t + U_g h_{t-1} + b_g) \\
	o_t &= \tanh(W_o X_t + U_o h_{t-1} + b_o) \\
	c_t &= f_t \circ c_{t-1} \oplus i_t \circ g_t \\
	h_t &= o_t + \tanh(c_t)
	\end{align*}
	
	
	
\end{frame}

\begin{frame}
\frametitle{Les réseaux de neurones : récurrents}
\framesubtitle{Long Short-Term Memory (LSTM) : Un peu d'humour}

\begin{center}
	\hgraphpage[.5\textwidth]{humour-lstm.jpg}
\end{center}

\end{frame}

\subsection{Gated Recurrent Unit (GRU)}

\begin{frame}
\frametitle{Les réseaux de neurones : récurrents}
\framesubtitle{Gated Recurrent Unit (GRU) : architecture}
	
	\begin{minipage}{0.50\textwidth} 
		\begin{itemize}
			\item $z$ : \optword{Porte d'actualisation} ; 
			Fusionne les portes d'entrée et d'oubli des LSTMs.
			\item $r$ : \optword{Porte de réinitialisation} ; 
			Combien on passe de l'état précédent ?
			\item $h'$ : \optword{Porte de l'état candidat} ;
			L'état courant	
		\end{itemize}
	\end{minipage}
	%
	\begin{minipage}{0.49\textwidth}
		\hgraphpage[\textwidth]{GRU.pdf}
	\end{minipage}
	
%	\begin{itemize}
%		\item $g$ : \optword{Porte de modulation d'entrée} (souvent considérée comme partie de la porte d'entrée) : 
%		Est-ce qu'on augmente ou diminuer l'état ?
%		%	\item La dérivée de l'erreur sera ue somme de la porte d'oubli avec d'autres dérivées
%	\end{itemize}

\end{frame}

\begin{frame}
	\frametitle{Les réseaux de neurones : récurrents}
	\framesubtitle{Gated Recurrent Unit (GRU) : équations}
	
	\begin{align*}
	z_t &= \sigma(W_z X_t + U_z h_{t-1} + b_z) \\
	r_t &= \sigma(W_r X_t + U_r h_{t-1} + b_r) \\
	h'_t &= \tanh(W X_t + U (h_{t-1} \circ r_t) + b) \\
	h_t &= z_t \circ h_{t-1} \oplus (1-z_t) \circ h'_{t-1}
	\end{align*}
	
	Tutoriel sur la rétro-propagation dans les GRUs :
	
	\url{https://www.math.ucla.edu/~minchen/doc/BPTTTutorial.pdf}
	
\end{frame}

\begin{frame}
	\frametitle{Les réseaux de neurones : récurrents}
	\framesubtitle{Gated Recurrent Unit (GRU) : Un peu d'humeur}
	
	\begin{center}
		\vgraphpage{humour-gru.png}
	\end{center}
	
\end{frame}


\subsection{Attention}

\begin{frame}
	\frametitle{Les réseaux de neurones : récurrents}
	\framesubtitle{Attention}
	
	\begin{itemize}
		\item Modèle : Seq2Seq (encodeur-décodeur)
		\item Les dépendances à long terme ne sont pas considérées pour des longues séquences
		\item La génération des mots destinataires dépend seulement des mots générés précédemment et les derniers mots du texte original
		\item IDÉE 
		\begin{itemize}
			\item Garder $T$ états cachés de l'encodeur : $h_i$
			\item Pour chaque sortie précédente du décodeur $s_{t-1}$, nous calculons son alignement avec les états cachés.
			$ e_{t, i} = a(s_{t-1}, h_i) $
			\item Calculer les poids des états cachés : $\alpha_{t, i} = softmax(e_{t, i})$
			\item Calculer le vecteur de l'état $c_t = \sum_{i=1}^{T} \alpha_{t, i} h_i$
		\end{itemize}
	\end{itemize}
	
\end{frame}

\begin{frame}
	\frametitle{Les réseaux de neurones : récurrents}
	\framesubtitle{Attention : Formulation}
	
	\begin{itemize}
		\item Clés (keys) : $m$ états sur lesquels le poids de chaque valeur est estimé $k_i$
		\item Valeurs (Values) : $m$ états avec lesquels le contexte est construit $v_i$
		\item Requête (Query) : représentation courante $q$
	\end{itemize}
	
	\[e_i = a(q, k_i)\]
	\[\alpha_i = softmax(e_i) = \frac{exp(e_i)}{\sum_j exp(e_j)}\]
	\[attention(q, K, V) = \sum_{i=1}^{m} \alpha_i v_i\]
	
	\begin{itemize}
		\item Des fois l'opération softmax est masquée
		\item Un vecteur des 0 et des 1 est utilisé pour décider les éléments à prendre en compte
	\end{itemize}
	
\end{frame}

\begin{frame}
	\frametitle{Les réseaux de neurones : récurrents}
	\framesubtitle{Attention : Architecture}
	
	\hgraphpage{attention.pdf}
	
\end{frame}


\begin{frame}
	\frametitle{Les réseaux de neurones : récurrents}
	\framesubtitle{Attention : Fonctions de notation de l'attention (sans paramètres)}
	
	\[Q \in \mathbb{R}^{n \times d}, \, K \in \mathbb{R}^{m \times d}, \, a(Q, K) \in \mathbb{R}^{n \times m}, V \in \mathbb{R}^{m \times v} \]
	
	Dot product (\textbf{tf.keras.layers.Attention})
	\[a(Q, K) = Q \cdot K^\top\]
	
	Scaled dot product 
	\[a(Q, K) = \frac{Q \cdot K^\top}{\sqrt{d}}\]
	
	Cosine similarity
	\[a(Q, K) = \frac{Q \cdot K^\top}{||Q|| \, ||K||}\]
	
\end{frame}

\begin{frame}
	\frametitle{Les réseaux de neurones : récurrents}
	\framesubtitle{Attention : Fonctions de notation de l'attention (avec paramètres)}
	
	\[Q \in \mathbb{R}^{n \times d}, \, K \in \mathbb{R}^{m \times d}, \, a(Q, K) \in \mathbb{R}^{n \times m}, V \in \mathbb{R}^{m \times v} \]
	
	Additive attention (\textbf{tf.keras.layers.AdditiveAttention})
	
	Bahdanau Attention
	
	\[a(Q, K) = W_v^\top \cdot \tanh(W_q Q + W_k K)\]
	
	
	
\end{frame}


\begin{frame}
	\frametitle{Les réseaux de neurones : récurrents}
	\framesubtitle{Attention : Attention multi-têtes (Multi-Head Attention)}
	\vspace{-0.5cm}
	\begin{figure}
		\centering
		\hgraphpage[0.9\textwidth]{multi_head_att_.pdf}
		\caption{Attention multi-têtes \cite{2017-vaswani-al}}
	\end{figure}
	
\end{frame}

\begin{frame}
	\frametitle{Les réseaux de neurones : récurrents}
	\framesubtitle{Attention : Attention multi-têtes (Multi-Head Attention)}
	
	\[Q \in \mathbb{R}^{n \times d}, \, K \in \mathbb{R}^{m \times d}, \, MultiHead \in \mathbb{R}^{n \times d_{model}}, V \in \mathbb{R}^{m \times v} \]
	
	\[W^Q_i \in \mathbb{R}^{d_{model} \times d}, \,  W^K_i \in \mathbb{R}^{d_{model} \times d}, \, W^V_i \in \mathbb{R}^{d_{model} \times v}, \, W^O \in \mathbb{R}^{hv \times d_{model}}\]
	
	
	\[MultiHead(Q, K, V) = Concat(head_1, \ldots, head_h) \cdot W^O\]
	
	\[head_i = Attention(Q W^Q_i, K W^Q_i, V W^V_i)\]
	
	
\end{frame}

\begin{frame}
	\frametitle{Les réseaux de neurones : récurrents}
	\framesubtitle{Attention : Self attention}
	
	\[Q = K = V\]
	
	\begin{itemize}
		\item Pour chaque élément d'une séquence, on essaye d'encoder son attention par rapport au même séquence
		\item Attention multi-têtes peut être utilisée pour ce faire
		\item La sortie sera une séquence encodée comme celle des RNN
		\item Le problème est que la séquence n'a pas l'information de l'ordre des séquences 
		\item Un embedding de la position peut être ajouté à l'encodage de chaque séquence
	\end{itemize}
	
\end{frame}


\begin{frame}
	\frametitle{Les réseaux de neurones : récurrents}
	\framesubtitle{Attention : Transformers}
	\vspace{-0.4cm}
	\begin{figure}
		\centering
		\hgraphpage[0.35\textwidth]{transformers_.pdf}
		\caption{Architecture des transformers \cite{2017-vaswani-al}}
	\end{figure}
	
\end{frame}


\insertbibliography{ML-RN}{*}

\end{document}


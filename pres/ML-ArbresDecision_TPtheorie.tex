% !TEX TS-program = pdflatex
% !TeX program = pdflatex
% !TEX encoding = UTF-8
% !TEX spellcheck = fr

\documentclass[xcolor=table]{beamer}

\def\accyear{2024-2025}
\institute{
	Laboratoire de la Communication dans les Systèmes Informatiques (LCSI)\\
	École  nationale Supérieure d'Informatique (ESI, ex. INI), Algiers, Algeria
}

\author[Abdelkrime Aries (\accyear)] %
{Abdelkrime Aries}

\titlegraphic{%
	\includegraphics[height=1cm]{../extra/logo/esi-logo.png}%
	\hspace*{2cm}%
	\includegraphics[height=1cm]{../extra/logo/lcsi-logo.png}%
	\hspace*{2cm}%
	\includegraphics[height=1cm]{../extra/logo/esi.ml.pdf}%
}%\hspace*{4.75cm}~

\date{Academic year: \accyear} %\today

\title[ML]{Machine Learning} 

\usefolder{../extra/beamer}

\usetheme{Karimml} % Antibes Boadilla Warsaw

\beamertemplatenavigationsymbolsempty

\usepackage{longfbox}

\newfboxstyle{kared}{padding=1pt,margin=1pt,border-radius=1ex,border-color=red}
\newfboxstyle{kgboxx}{padding=1pt,margin=1pt,border-radius=1ex,border-color=green}


\title[ML : Arbres de décision (TP)] %
{Machine Learning \\Arbres de décision (support TP)} 

\changegraphpath{../img/arbres/}

\begin{document}
	
\begin{frame}
	\frametitle{Machine Learning}
	\framesubtitle{Arbres de décision (support TP) : Introduction}
	
	\begin{itemize}
		\item Les arbres de décision
		\begin{itemize}
			\item comment créer un arbre de décision en utilisant ID3
			\item en utilisant la régression logistique
			\item dans le cas binaire (appartient à une classe ou hors classe)
		\end{itemize}
	\end{itemize}
\end{frame}


\begin{frame}
	\frametitle{Machine Learning}
	\framesubtitle{Arbres de décision (support TP) : Plan}
	
	\begin{multicols}{2}
		%	\small
		\tableofcontents
	\end{multicols}
\end{frame}

\section{Arbres de décision}

\begin{frame}
	\frametitle{Arbres de décision}
	
	\begin{itemize}
		\item \optword{ID3 (Iterative Dichotomiser 3)} : dévelopé en 1986 par Ross Quinlan. Il peut être appliqué seulement sur les caractéristiques nominales. Il est utilisé pour le classement.
		\item \optword{C4.5} : une extension de ID3 par Ross Quinlan. Il peut être appliqué sur tous les types de caractéristiques. Il est utilisé pour le classement.
		\item \optword{C5.0} : une extension commerciale de C4.5, toujours par Ross Quinlan.
		\item \optword{CART (Classification and Regression Trees)} : comme C4.5 mais utilise d'autres métriques. Aussi, l'algorithme supporte la régression.
	\end{itemize}
	
\end{frame}

\subsection{Algorithmes}

\begin{frame}
	\frametitle{Arbres de décision}
	\framesubtitle{Algorithmes : Construction (apprentissage)}
	
	\begin{algorithm}[H]
		\KwData{$ X, Y$}
		\KwResult{$ (S, A) $ : un arbre} 
		créer un sommet $ s \in S $\;
		
		\eIf{$Y$ est homogène ou on a atteint une condition d'arrêt}{
			$s.classe=\arg\max_{k}{|\{y \in Y / y = k \}|}$ \tcp{$s$ est une feuille}
		}{
			déterminer la caractéristique $X_j$ qui divise le mieux $Y$\;
			diviser $Y$ en sous-ensembles $Y_1, \ldots, Y_K$ selon les valeurs de $X_j$\;
			diviser $X$ en sous-ensembles $X_{*,1}, \ldots, X_{*, K}$ selon les valeurs de $X_j$\;
			$s.j = j$\;
			\ForEach{$ k \in \{1, \ldots, K\}$}{
				$s_k = Construction(X_{*,k}, Y_k) \in S$ $(s, s_k) \in A$ $val(s, s_k) = k$\;
			}
		}
		\caption{Génération d'un arbre de décision (algorithme général)}
	\end{algorithm}
	
\end{frame}

\begin{frame}
	\frametitle{Arbres de décision}
	\framesubtitle{Algorithmes : Recherche (prédiction)}
	
	\begin{algorithm}[H]
		\KwData{$x$ : un échantillon; $(S, A)$ : l'arbre de décision}
		\KwResult{$ y $ : sa classe} 
		
		$s$ est la racine de $(S, A)$\;
		
		
		\eIf{$s$ est une feuille}{
			\Return $s.classe$\;
		}{
			$j = s.j$\; 
			choisir l'arrête $(s, s_k)$ tel que $val(s, s_k) \equiv x_j$\;
			\Return Parcourir(x, $(s_k, A)$)\;
		}
		\caption{Parcours d'un arbre de décision (algorithme général)}
	\end{algorithm}
	
\end{frame}

\subsection{Conditions d'arrêt}

\begin{frame}
	\frametitle{Arbres de décision}
	\framesubtitle{Conditions d'arrêt}
	
	\begin{itemize}
		\item \optword{Homogénéité} : les observations dans le nœud ont la même classe
		\item \optword{Impureté minimale} : l'impureté ou l'erreur de classification/régression des observations dans le nœud est inférieure ou égale à ce seuil
		\item \optword{Nombre minimal des observations} : le nombre des observations dans un nœud est inférieure ou égale à ce seuil
		\item \optword{Profondeur} : la profondeur du nœud dans l'arbre est inférieure à ce seuil
	\end{itemize}
	
\end{frame}

\subsection{Critique}

\begin{frame}
	\frametitle{Arbres de décision}
	\framesubtitle{Critique : Avantages}
	
	\begin{itemize}
		\item simples à comprendre et à interpréter. On peut visualiser les arbres. Aussi, on peut expliquer les résultats obtenus facilement.
		\item peuvent travailler sur des données avec peu de préparation. Par exemple, ils n'ont pas besoin de la normalisation des données.
		\item acceptent les données numériques et nominales. Les autres algorithmes d'apprentissage sont spécialisés dans un seul type de données.
		\item donnent de bonne performance même si leurs hypothèses sont un peu violées par le modèle réel à partir duquel les données ont été générées.
	\end{itemize}
	
\end{frame}

\begin{frame}
	\frametitle{Arbres de décision}
	\framesubtitle{Critique : Inconvénients}
	
	\begin{itemize}
		\item peuvent être aussi complexes, ils ne généralisent pas bien (overfitting: surapprentissage). On peut régler ça en fixant le nombre minimum des échantillons dans les feuilles ou en fixant la profondeur maximale de l'arbre.
		\item peuvent être instables à cause des variations des données.
		\item il existe des problèmes qui sont un peu difficiles à apprendre par les arbres de décision. Ils ne sont pas faciles à exprimer, par exemple: XOR.
		\item peuvent être biaisés à la classe dominante. Donc, il faut balancer les données avant d'entrainer le système.
		\item ce n'est pas garanti de tomber sur l'arbre de décision optimal.
	\end{itemize}
	
\end{frame}

\section{ID3 (Rappel)}

\begin{frame}
	\frametitle{ID3 (Rappel)}
	
	\begin{itemize}
		\item \keyword{Iterative Dichotomiser 3}
		\item n'accepte que les caractéristiques nominales
		\item seulement pour le classement
		\item on s'arrête si les ensembles des classes dans les feuilles sont homogènes
	\end{itemize}
	
\end{frame}

\subsection{Homogénéité d'un ensemble}

\begin{frame}
	\frametitle{ID3 (Rappel)}
	\framesubtitle{Homogénéité d'un ensemble}
	
	\begin{itemize}
		\item Entropie de Shannon $H(Y)$ pour mesurer l'incertitude d'un ensemble $Y$
		\item $H(Y)=0$ : $Y$ contient les mêmes valeurs (classes)
		\item $H(Y)=1$ : $Y$ contient des valeurs totalement différentes
		\item étant donné $V_y$ l'ensemble du vocabulaire de $Y$ (différentes valeurs ou classes)
	\end{itemize}

	\[H(Y) = - \sum\limits_{v \in V_y} p(v/Y) \log_2 p(v/Y)\]
	
	\[p(v/Y) = \frac{|\{y / y \in Y \wedge y = v\}|}{|S|}\]
	
\end{frame}

\subsection{Division d'un ensemble}

\begin{frame}
	\frametitle{ID3 (Rappel)}
	\framesubtitle{Division d'un ensemble}
	
	\begin{itemize}
		\item $Y$ : un ensemble des prédictions 
		\item $X_j$ : les valeurs de la caractéristique $j$ des échantillons ayant comme prédictions $Y$
		\item $v$ : une valeur parmi les valeurs possibles de $X_j$ (vocabulaire $V_j$)
		\item pour chaque valeur $v \in V_j$, on crée un ensemble $Y_{j, v}$
		\item si $|V| = K$, on aura $K$ ensembles $Y_1, \cdots, Y_K$ et 
	\end{itemize}
	
	\[Y_{j,v} = \{y^{(i)} \in Y / X_j^{(i)} \in X_j \wedge X_j^{(i)} = v\}\]

	
\end{frame}

\subsection{Choix de caractéristique de division}

\begin{frame}
	\frametitle{ID3 (Rappel)}
	\framesubtitle{Choix de caractéristique de division}
	
	\begin{itemize}
		\item On utilise le gain d'entropie (information gain) $IG(Y, X_j)$
		\item $j = \arg\max_{j'} IG(Y, X_{j'})$
		\item Cette mesure calcule combien d'incertitude dans $Y$ a été réduite après sa division en utilisant l'attribut $j$.
		\item c'est la différence entre l'entropie de $Y$ et l'entropie pondérée des ensembles divisées
	\end{itemize}

	\[IG(Y, X_j) = H(Y) - \sum_{v \in V_j} p(v/X_j) H(Y_{j, v})\]
	
	
\end{frame}

\section{CART}

\begin{frame}
	\frametitle{CART}
	
	\begin{itemize}
		\item \keyword{Classification and Regression Trees}
		\item supporte la régression.
		\item essaye de minimiser une fonction de coût.
		\item utilise le pré-élagage en utilisant un critère d'arrêt.
		\item crée des arbres binaires 
	\end{itemize}
	
\end{frame}

\subsection{Homogénéité d'un ensemble}

\begin{frame}
	\frametitle{CART}
	\framesubtitle{Homogénéité d'un ensemble}
	
	\begin{itemize}
		\item indexe de diversité $Gini(Y)$ pour mesurer l'erreur de classification de $Y$
		\item $Gini(Y)=0$ représente la meilleur division
		\item $H(Y)=0.5$ représente la division la plus pire
		\item étant donné $V_y$ l'ensemble du vocabulaire de $Y$ (différentes valeurs ou classes)
		\item Dans le cas de la régression, on utilise \keyword{MSE}
	\end{itemize}
	
	\[Gini(S) = \sum\limits_{v \in V_y} p(v/Y) (1-p(v/Y)) = 1 - \sum\limits_{v \in V_y} p(v/Y)^2 \]
	
	\[p(v/Y) = \frac{|\{y / y \in Y \wedge y = v\}|}{|S|}\]
	
\end{frame}

\subsection{Division d'un ensemble}

\begin{frame}
	\frametitle{CART}
	\framesubtitle{Division d'un ensemble}
	
	\begin{itemize}
		\item $Y$ : un ensemble des prédictions 
		\item $X_j$ : les valeurs de la caractéristique $j$ des échantillons ayant comme prédictions $Y$
		\item $v$ : une valeur parmi les valeurs possibles de $X_j$ (vocabulaire $V_j$)
		\item pour chaque valeur $v \in V_j$, on crée deux ensembles $Y_G$ et $Y_D$
		\item Donc, on aura 2 ensembles $Y_G$ (avec des valeurs $X_j > v$) et $Y_D$ (avec des valeurs $X_j \le v$)
	\end{itemize}
	
	\[Y_G = \{y^{(i)} \in Y / X_j^{(i)} \in X_j \wedge X_j^{(i)} > v\}\]
	\[Y_D = \{y^{(i)} \in Y / X_j^{(i)} \in X_j \wedge X_j^{(i)} \le v\}\]
	
\end{frame}

\subsection{Choix de caractéristique de division}

\begin{frame}
	\frametitle{CART}
	\framesubtitle{Choix de caractéristique de division}
	
	\begin{itemize}
		\item On utilise la diversité Gini de la division $Gini_{div}(Y_G, Y_D)$
		\item $j, v = \arg\min_{j', v'} Gini_{div}(Y_G, Y_D)$
		\item Ici, on ne cherche pas seulement la caractéristique $j$, mais aussi la valeur $v$ qui minimise la diversité Gini de la division
		\item $|S| = |S_G| + |S_D|$
		\item pour la régression, on utilise \keyword{MSE} à la place de \keyword{Gini}
		\item l'estimation dans le cas de la régression est la moyenne des $Y$ dans la feuille
	\end{itemize}
	
	\[Gini_{div}(S_G, S_D) = \frac{|S_G|}{|S|} Gini(S_G) + \frac{|S_D|}{|S|} Gini(S_D)\]
	
	
\end{frame}

\section{Forêts aléatoires}

\begin{frame}
	\frametitle{Forêts aléatoires (Rappel)}
	
	\begin{itemize}
		\item méthode d'ensembles
		\item utilise les arbres de décision pour l'estimation
		\item l'estimation finale se fait par vote majoritaire 
		\item utilise des Bootstraps (ensembles d'observations aléatoires) pour l'apprentissage des arbres 
		\item utilise des attributs aléatoires pour entrainer chaque arbre : en utilisant moins d'attributs dans chaque arbre, on peut prévenir le problème de sur-apprentissage
	\end{itemize}
	
\end{frame}

\subsection{Apprentissage ensembliste}

\begin{frame}
	\frametitle{Forêts aléatoires}
	\framesubtitle{Apprentissage ensembliste}
	
	\begin{itemize}
		\item \optword{Bootstrap aggregating (Bagging)}
		\begin{itemize}
			\item Entraîner des estimateurs en parallèle sur des données aléatoires
		\end{itemize}
		\item \optword{Boosting}
		\begin{itemize}
			\item Entrainer des estimateurs sur les mêmes données d'une façon séquentielle ; l'un améliore la performance de l'aure
		\end{itemize}
		\item \optword{Stacking}
		\begin{itemize}
			\item Entrainer des estimateurs sur les mêmes données d'une façon parallèle ; Entraîner un estimateur qui fusionne les estimation des autres estimateurs
		\end{itemize}
	\end{itemize}
	
\end{frame}

\begin{frame}
	\frametitle{Forêts aléatoires}
	\framesubtitle{Apprentissage ensembliste : Bootstrap aggregating (Bagging)}
	
	\begin{itemize}
		\item \textbf{Entraînement}
		\begin{itemize}
			\item Créer \textbf{K} Bootstraps (ensembles aléatoires) à partir du dataset d'entraînement
			\item Entraîner un modèle par Bootstrap
		\end{itemize}
		\item \textbf{Estimation}
		\begin{itemize}
			\item Utiliser les modèles pour avoir \textbf{K} estimations
			\item estimation finale (Classification) : par vote majoritaire
			\item estimation finale (Régression) : par moyenne mathématique
		\end{itemize}
		\item \textbf{Exemples}
		\begin{itemize}
			\item \expword{Forêts aléatoires}
		\end{itemize}
	\end{itemize}
	
\end{frame}

\begin{frame}
	\frametitle{Forêts aléatoires}
	\framesubtitle{Apprentissage ensembliste : Boosting}
	
	\begin{itemize}
		\item \textbf{Entraînement}
		\begin{itemize}
			\item Entrainer un estimateur sur le dataset d'entraînement
			\item Faire des prédictions sur ce dataset pour extraire les échantillons qui ne sont pas bien estimés
			\item Entrainer un autre estimateur sur le même dataset mais avec plus de poids aux échantillons mal-estimés
			\item Fait la même opération jusqu'à génération de \textbf{K} estimateurs
		\end{itemize}
		\item \textbf{Estimation}
		\begin{itemize}
			\item Utiliser les modèles pour avoir \textbf{K} estimations
			\item estimation finale (Classification) : par vote majoritaire
			\item estimation finale (Régression) : par moyenne mathématique
		\end{itemize}
		\item \textbf{Exemples}
		\begin{itemize}
			\item \expword{AdaBoost}
		\end{itemize}
	\end{itemize}
	
\end{frame}


\begin{frame}
	\frametitle{Forêts aléatoires}
	\framesubtitle{Apprentissage ensembliste : Stacking}
	
	\begin{itemize}
		\item \textbf{Entraînement}
		\begin{itemize}
			\item Entrainer \textbf{K} estimateurs sur le même dataset d'entraînement
			\item Ces estimateurs doivent avoir des hyper-paramètres différents ou des algorithmes d'apprentissage différents
			\item Entraîner un estimateur qui a comme entrée les sorties des autres \textbf{K} estimateurs 
			\item Ce dernier va apprendre à fusionner les estimations des autres estimateurs pour avoir une qui est meilleure
		\end{itemize}
		\item \textbf{Estimation}
		\begin{itemize}
			\item Utiliser les \textbf{K} estimateurs pour avoir des estimations initiales
			\item Utiliser le dernier estimateur pour combiner ces estimations
		\end{itemize}
	\end{itemize}
	
\end{frame}

\subsection{Paramètres d'un forêt}

\begin{frame}
	\frametitle{Forêts aléatoires}
	\framesubtitle{Paramètres d'un forêt}
	
	\begin{itemize}
		\item \optword{bootstrap}
		\begin{itemize}
			\item Est-ce qu'on utilise le dataset comme il est ou on utilise le \keyword{bootraping} ?
			\item Quelle est la taille d'un Bootstrap ?
			\item Combien d'aléatoire utilise-t-on pour générer le Bootstrap ? 
		\end{itemize}
		\item \optword{attribut}
		\begin{itemize}
			\item Combien d'attributs doit-t-on utiliser par arbre ?
			\item Comment choisir ces attributs (quantité d'aléatoire) ?
		\end{itemize}
		\item \optword{arbre}
		\begin{itemize}
			\item Combien d'arbres utilise-t-on pour l'estimation ?
			\item plus les paramètres des arbres
		\end{itemize}
	\end{itemize}
	
\end{frame}

%\insertbibliography{ML-preparation}{*}

\end{document}




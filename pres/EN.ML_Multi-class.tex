% !TEX TS-program = pdflatex
% !TeX program = pdflatex
% !TEX encoding = UTF-8
% !TeX spellcheck = en_US

\documentclass[xcolor=table]{beamer}

\def\accyear{2024-2025}
\institute{
	Laboratoire de la Communication dans les Systèmes Informatiques (LCSI)\\
	École  nationale Supérieure d'Informatique (ESI, ex. INI), Algiers, Algeria
}

\author[Abdelkrime Aries (\accyear)] %
{Abdelkrime Aries}

\titlegraphic{%
	\includegraphics[height=1cm]{../extra/logo/esi-logo.png}%
	\hspace*{2cm}%
	\includegraphics[height=1cm]{../extra/logo/lcsi-logo.png}%
	\hspace*{2cm}%
	\includegraphics[height=1cm]{../extra/logo/esi.ml.pdf}%
}%\hspace*{4.75cm}~

\date{Academic year: \accyear} %\today

\title[ML]{Machine Learning} 

\usefolder{../extra/beamer}

\usetheme{Karimml} % Antibes Boadilla Warsaw

\beamertemplatenavigationsymbolsempty

\usepackage{longfbox}

\newfboxstyle{kared}{padding=1pt,margin=1pt,border-radius=1ex,border-color=red}
\newfboxstyle{kgboxx}{padding=1pt,margin=1pt,border-radius=1ex,border-color=green}


\title[ML: Multi-class classification] %
{Machine Learning\\Multi-class classification} 

\changegraphpath{../img/multi-class/}

\begin{document}
	
\begin{frame}
	\frametitle{Machine Learning}
	\framesubtitle{Multi-class classification: Introduction}
	
	\begin{itemize}
		\item We have already seen ...
		\begin{itemize}
			\item how to estimate the probability of a class
			\item using logistic regression
			\item in case of binary classification (belongs to a class or not)
		\end{itemize}
		\item But ...
		\begin{itemize}
			\item how to estimate them in case of multiple classes?
			\item how to affect a sample into many classes at once?
		\end{itemize}

	\end{itemize}
\end{frame}


\begin{frame}
	\frametitle{Machine Learning}
	\framesubtitle{Multi-class classification: Plan}
	
	\begin{multicols}{2}
		%	\small
		\tableofcontents
	\end{multicols}
\end{frame}

\section{Classification}

\begin{frame}
	\frametitle{Machine Learning}
	\framesubtitle{Classification}
	
	\hgraphpage{classif.pdf}
	
\end{frame}

\subsection{Binary classification}


\begin{frame}
	\frametitle{Machine Learning: Classification}
	\framesubtitle{Binary classification}
	
	\begin{minipage}{0.65\textwidth}
	\begin{itemize}
		\item Number of available classes: 2 (Actually, it is just one class. We test if a sample belongs to it or not)
		\item Number of chosen classes: 1
		\item Examples
		\begin{itemize}
			\item \expword{Classify an image as containing a specific object or not.}
			\item \expword{Finding out if a message is a spam or not.}
		\end{itemize}
	\end{itemize}
	\end{minipage}
	\begin{minipage}{0.3\textwidth}
		\hgraphpage{binary.pdf}
	\end{minipage}
	
\end{frame}

\subsection{Multi-class classification}

\begin{frame}
	\frametitle{Machine Learning: Classification}
	\framesubtitle{Multi-class classification}
	
	\begin{minipage}{0.65\textwidth}
	\begin{itemize}
		\item Number of available classes: L $ \ge $ 2
		\item Number of chosen classes: 1
		\item Examples
		\begin{itemize}
			\item \expword{Detect an animal from an input image.}
			\item \expword{Detect a movie's certification (PG, R, etc.) from its description.}
		\end{itemize}
	\end{itemize}
	\end{minipage}
	\begin{minipage}{0.3\textwidth}
		\hgraphpage{multiclass.pdf}
	\end{minipage}
	
\end{frame}

\begin{frame}
	\frametitle{Machine Learning: Classification}
	\framesubtitle{Multi-class classification: Methods}
	
	\hgraphpage{multi-class.pdf}
	
\end{frame}

\subsection{Multi-label classification}

\begin{frame}
	\frametitle{Machine Learning: Classification}
	\framesubtitle{Multi-label classification}
	
	\begin{minipage}{0.65\textwidth}
	\begin{itemize}
		\item Number of available classes: L $ \ge $ 2
		\item Number of chosen classes: K $ \le $ L
		\item Examples
		\begin{itemize}
			\item \expword{Detect many animals from an input image.}
			\item \expword{Find the clothes (hat, jeans, scarf, etc.) from an image.}
			\item \expword{Detect movie's genres (Sci-fi, Action, Comedy, etc.) from its description.}
		\end{itemize}
	\end{itemize}
	\end{minipage}
	\begin{minipage}{0.3\textwidth}
		\hgraphpage{multilabel.pdf}
	\end{minipage}
	
\end{frame}

\begin{frame}
	\frametitle{Machine Learning: Classification}
	\framesubtitle{Multi-label classification: Methods \cite{2012-madjarov-al}}
	
	\hgraphpage{multi-label.pdf}
	
	\begin{itemize}
		\item scikit-multilearn: \url{http://scikit.ml/}
	\end{itemize}
	
\end{frame}

\section{Binary logistic regression}

\begin{frame}
	\frametitle{Machine Learning}
	\framesubtitle{Binary logistic regression}
	
	\begin{center}
		\vgraphpage{LR_bin.pdf}
	\end{center}
	
\end{frame}

\subsection{Probability estimation}

\begin{frame}
\frametitle{Machine Learning: Binary logistic regression}
\framesubtitle{Probability estimation}

\[Z = \sum_{j=1}^{N} \theta_j X_j = X \cdot \theta\]
\begin{itemize}
	\item $Z[M]$ is a vector of $M$ elements (samples)
	\item $X[M, N]$ is a matrix of $M$ samples and $N$ features
	\item $\theta[N]$ is a vector of $N$ parameters (weights)
\end{itemize}

\[H = \sigma(Z) = \frac{1}{1+e^{-Z}}\]

\begin{itemize}
	\item $H[M]$ is a vector of $M$ probabilities (samples)
\end{itemize}

\end{frame}

\subsection{Cost and Gradient}

\begin{frame}
	\frametitle{Machine Learning: Binary logistic regression}
	\framesubtitle{Cost and Gradient}

	\[J_\theta = BCE = \frac{-1}{M} \sum\limits_{i=1}^{M} [Y^{(i)} \log(H^{(i)}) + (1- Y^{(i)}) \log(1 - H^{(i)})]\]
	\begin{itemize}
		\item $Y[M]$ and $H[M]$ are two vectors of $M$ elements (samples)
		\item $J_\theta$ is a scalar
	\end{itemize}
	
	
	\[
	\frac{\partial BCE}{\partial \theta_j} = \frac{1}{M} \sum\limits_{i=1}^{M} (H^{(i)} - Y^{(i)}) X_j^{(i)}
	\]
	
	\[
	\frac{\partial BCE}{\partial \theta} = \frac{1}{M} (H - Y) \cdot X
	\]
	
	\begin{itemize}
		\item $\frac{\partial BCE}{\partial \theta}[N]$ is a vector of $N$ elements (features)
	\end{itemize}
	
\end{frame}

\subsection{Gradient (derivation)}

\begin{frame}
	\frametitle{Machine Learning: Binary logistic regression}
	\framesubtitle{Gradient (derivation)}

	\tiny\vspace{-6pt}
	\begin{align*}
	\frac{\partial BCE}{\partial \theta_j} 
	& = & \frac{-1}{M} \sum\limits_{i=1}^{M} \frac{\partial}{\partial \theta_j} [Y^{(i)} \log(H^{(i)}) + (1- Y^{(i)}) \log(1 - H^{(i)})] \\
	& = & \frac{-1}{M} \sum\limits_{i=1}^{M} [ Y^{(i)} \frac{\partial}{\partial \theta_j} \log(H^{(i)}) + (1- Y^{(i)}) \frac{\partial}{\partial \theta_j}\log(1 - H^{(i)})]\\
	& = & \frac{-1}{M} \sum\limits_{i=1}^{M} [ Y^{(i)} \frac{1}{H^{(i)}} \frac{\partial}{\partial \theta_j} H^{(i)} + (1- Y^{(i)}) \frac{-1}{1-H^{(i)}} \frac{\partial}{\partial \theta_j} H^{(i)})] \\
	& = & \frac{-1}{M} \sum\limits_{i=1}^{M} \frac{Y^{(i)}-H^{(i)}}{H^{(i)}(1-H^{(i)})} \frac{\partial}{\partial \theta_j} H^{(i)} \\
	\end{align*}
	
	\vspace{-3pt}
	\begin{align*}
	\frac{\partial H^{(i)}}{\partial \theta_j} 
	& = & \frac{\partial \sigma(Z^{(i)})}{\partial Z^{(i)}} \frac{\partial Z^{(i)}}{\partial \theta_j}
	 =  [\sigma(Z^{(i)}) (1-\sigma(Z^{(i)}))]\frac{\partial}{\partial \theta_j} \sum\limits_{k=0}^{N} \theta_k X_k^{(i)} 
	 =  H^{(i)} (1-H^{(i)})  X_j^{(i)}\\
	\end{align*}
	
	\vspace{-3pt}
	\begin{align*}
	\frac{\partial BCE}{\partial \theta_j} 
	& = & \frac{-1}{M} \sum\limits_{i=1}^{M} \frac{Y^{(i)}-H^{(i)}}{H^{(i)}(1-H^{(i)})} [H^{(i)} (1-H^{(i)}) X_j^{(i)}] \\
	& = & \frac{1}{M} \sum\limits_{i=1}^{M} (H^{(i)} - Y^{(i)}) X_j^{(i)}\\
	\end{align*}
		
\end{frame}

\subsection{Parameters' update}

\begin{frame}
	\frametitle{Machine Learning: Binary logistic regression}
	\framesubtitle{Parameters' update}

	\[\theta = \theta - \alpha \frac{\partial J_\theta}{\partial \theta}\]
	
	\begin{itemize}
		\item $\frac{\partial J_\theta}{\partial \theta}[N]$ is a vector of $N$ elements (features)
		\item $\theta[N]$ is a vector of $N$ elements (features)
		\item $\alpha $ is a learning rate
	\end{itemize}
	
\end{frame}

\section{Multi-class logistic regression}

%\begin{frame}
%	\frametitle{Machine Learning}
%	\framesubtitle{Multi-class logistic regression}
%	
%	\hgraphpage{multi-class.pdf}
%	
%\end{frame}

\subsection{One-vs-Rest}

\begin{frame}
	\frametitle{Machine Learning: Multi-class logistic regression}
	\framesubtitle{One-vs-Rest}
	
	\begin{center}
		\vgraphpage{LR_ovr.pdf}
	\end{center}
	
\end{frame}

\begin{frame}
	\frametitle{Machine Learning: Multi-class logistic regression}
	\framesubtitle{One-vs-Rest: Description}
	
	Given $L$ output classes:
	\begin{itemize}
		\item Training
		\begin{itemize}
			\item For each class $C_l$, we train a binary model $M_l$ separately
			\item In this case, we will have $L$ binary models
			\item The positive class is represented by $C_l$ class's samples
			\item The negative class is represented by the rest of the samples
		\end{itemize}
		\item Estimation
		\begin{itemize}
			\item Given a sample
			\item For each class $C_l$, we estimate its probability using $M_l$
			\item We take the class with the maximum probability
			\item In this case, the sum of probabilities does not always give $1$
			\[\sum_{l=1}^{L} P(l|x; \theta^{M_l}) \in [0, L]\]
		\end{itemize}
	\end{itemize}
	
\end{frame}

\subsection{One-vs-One}

\begin{frame}
	\frametitle{Machine Learning: Multi-class logistic regression}
	\framesubtitle{One-vs-One}
	
%	\begin{center}
		\hgraphpage{LR_ovo.pdf}
%	\end{center}
	
\end{frame}

\begin{frame}
	\frametitle{Machine Learning: Multi-class logistic regression}
	\framesubtitle{One-vs-One: Description}
	
	Given $L$ output classes:
	\begin{itemize}
		\item Training
		\begin{itemize}
			\item For each two classes $C_{l}$ and $C_{l'}$, we train a binary model $M_{ll'}$ separately
			\item In this case, we will have $L (L-1)/2$ binary models
			\item The positive class is represented one of the two classes
			\item The negative class is represented by the other class
		\end{itemize}
		\item Estimation
		\begin{itemize}
			\item Given a sample
			\item For each model $C_{ll'}$, we estimate a probability
			\item In this case, the class is either $l$ or $l'$
			\item We count the number of each class being estimated
			\item The majority wins (vote)
		\end{itemize}
	\end{itemize}
	
\end{frame}

\subsection{Multinomial}

\begin{frame}
	\frametitle{Machine Learning: Multi-class logistic regression}
	\framesubtitle{Multinomial}
	
	\begin{center}
		\vgraphpage{LR_multi.pdf}
	\end{center}
	
\end{frame}

\begin{frame}
	\frametitle{Machine Learning: Multi-class logistic regression}
	\framesubtitle{Multinomial: Description}
	
	Given $L$ output classes:
	\begin{itemize}
		\item This is a generalization of binary classification in logistic regression
		\item Training
		\begin{itemize}
			\item For each class $C_{l}$, we calculate a weighted sum
			\item We apply \textbf{Softmax} function on these outputs
			\item It is a function which transforms the sums into probabilities
			\item Also, it normalizes these probabilities to get a sum of $1$
			\item The output classes are represented using One-Hot
			\item The cost function is a generalized version of that of binary classification
		\end{itemize}
		\item Estimation
		\begin{itemize}
			\item Given a sample
			\item We apply the model to get a vector of probabilities
			\item The most probable class wins
		\end{itemize}
	\end{itemize}
	
\end{frame}

\begin{frame}
	\frametitle{Machine Learning: Multi-class logistic regression}
	\framesubtitle{Multinomial: Probability estimation}
	
	\[Z = \sum_{j=1}^{N} \theta_j X_j = X \cdot \theta\]
	\begin{itemize}
		\item $Z[M, L]$ is a matrix of $M$ samples and $L$ classes
		\item $X[M, N]$ is a matrix of $M$ samples and $N$ features
		\item $\theta[N, L]$ is a matrix of $N$ features and $L$ classes
	\end{itemize}
	
	\[H = softmax(Z) = \frac{e^{Z}}{\sum_{k=1}^{L} e^{Z_k}}\]
	
	\begin{itemize}
		\item $H[M, L]$ is a matrix of $M$ samples and $L$ classes
	\end{itemize}
	
\end{frame}

\begin{frame}
	\frametitle{Machine Learning: Multi-class logistic regression}
	\framesubtitle{Multinomial: Cost and gradient}
	
	\[J_\theta = \frac{-1}{M} \sum\limits_{i=1}^{M} \sum_{k=1}^{L} Y^{(i)}_k \log(H^{(i)}_k)\]
	\begin{itemize}
		\item $Y[M, L]$ and $H[M, L]$ are two matrices of $M\times L$ elements (features X classes)
		\item $J_\theta$ is a scalar
	\end{itemize}
	
	
	\[
	\frac{\partial BCE}{\partial \theta_j} = \frac{1}{M} \sum\limits_{i=1}^{M} (H^{(i)} - Y^{(i)}) X_j^{(i)}
	\]
	
	\[
	\frac{\partial BCE}{\partial \theta} = \frac{1}{M} (H - Y) \cdot X
	\]
	
	\begin{itemize}
		\item $\frac{\partial BCE}{\partial \theta}[N, L]$ is a matrix of $N\times L$ elements (features X classes)
	\end{itemize}
	
\end{frame}

\begin{frame}
	\frametitle{Machine Learning: Multi-class logistic regression}
	\framesubtitle{Multinomial: Parameters' update}
	
	\[\theta = \theta - \alpha \frac{\partial J_\theta}{\partial \theta}\]
	
	\begin{itemize}
		\item $\frac{\partial BCE}{\partial \theta}[N, L]$ is a matrix of $N\times L$ elements (features X classes)
		\item $\theta[N, L]$ is a matrix of $N$ features and $L$ classes
		\item $\alpha$ is the learning rate
	\end{itemize}
	
\end{frame}

\section{Multi-label logistic regression}

\subsection{Binary relevance}

\begin{frame}
	\frametitle{Machine Learning: Multi-label logistic regression}
	\framesubtitle{Binary relevance}
	
	\begin{center}
		\vgraphpage{LR_binary_relevance.pdf}
	\end{center}
	
\end{frame}

\begin{frame}
	\frametitle{Machine Learning: Multi-class logistic regression}
	\framesubtitle{Binary relevance: Description}
	
	Given $L$ output classes:
	\begin{itemize}
		\item Training
		\begin{itemize}
			\item For each class $C_l$, we train a binary model $M_l$ separately
			\item In this case, we will have $L$ binary models
			\item The positive class is represented by $C_l$ class's samples
			\item The negative class is represented by the rest of the samples
		\end{itemize}
		\item Estimation
		\begin{itemize}
			\item Given a sample
			\item For each class $C_l$, we estimate its probability using $M_l$
			\item If the probability is greater or equals 50\%, then the sample belongs to the class $C_l$
		\end{itemize}
	\end{itemize}
	
\end{frame}

\subsection{Label powerset}

\begin{frame}
	\frametitle{Machine Learning: Multi-label logistic regression}
	\framesubtitle{Label powerset}
	
%	\begin{center}
		\hgraphpage{LR_label_powerset.pdf}
%	\end{center}
	
\end{frame}

\begin{frame}
	\frametitle{Machine Learning: Multi-class logistic regression}
	\framesubtitle{Label powerset: Description}
	
	Given $L$ output classes:
	\begin{itemize}
		\item Training
		\begin{itemize}
			\item We look for all combinations of classes
			\item Each combination is considered as a new class
			\item We train a multi-class model
		\end{itemize}
		\item Estimation
		\begin{itemize}
			\item Given a sample
			\item We use our trained multi-class model to get one class
			\item This class is a combination of many original classes
		\end{itemize}
	\end{itemize}
	
\end{frame}

\insertbibliography{EN.ML_Multi-class}{*}

\end{document}




% !TEX TS-program = pdflatex
% !TeX program = pdflatex
% !TEX encoding = UTF-8
% !TeX spellcheck = en_US

\documentclass[xcolor=table]{beamer}

\usepackage{../extra/beamer/karimml}

\def\accyear{2024-2025}
\institute{
	Laboratoire de la Communication dans les Systèmes Informatiques (LCSI)\\
	École  nationale Supérieure d'Informatique (ESI, ex. INI), Algiers, Algeria
}

\author[Abdelkrime Aries (\accyear)] %
{Abdelkrime Aries}

\titlegraphic{%
	\includegraphics[height=1cm]{../extra/logo/esi-logo.png}%
	\hspace*{2cm}%
	\includegraphics[height=1cm]{../extra/logo/lcsi-logo.png}%
	\hspace*{2cm}%
	\includegraphics[height=1cm]{../extra/logo/esi.ml.pdf}%
}%\hspace*{4.75cm}~

\date{Academic year: \accyear} %\today

\title[ML]{Machine Learning} 

\usefolder{../extra/beamer}

\usetheme{Karimml} % Antibes Boadilla Warsaw

\beamertemplatenavigationsymbolsempty

\usepackage{longfbox}

\newfboxstyle{kared}{padding=1pt,margin=1pt,border-radius=1ex,border-color=red}
\newfboxstyle{kgboxx}{padding=1pt,margin=1pt,border-radius=1ex,border-color=green}


\subtitle[Neural Networks]{Neural Networks}  

\changegraphpath{../img/RN/}

\begin{document}

\begin{frame}
	\frametitle{\inserttitle}
	\framesubtitle{\insertshortsubtitle: Motivation}

	\begin{figure}
		\centering
		\hgraphpage[.8\textwidth]{neuron.png}
		\caption{Biological neuron \cite{2017-cain}}
	\end{figure}

\end{frame}

\begin{frame}
	\frametitle{\inserttitle}
	\framesubtitle{\insertshortsubtitle: Network types}
	
	\hgraphpage[\textwidth]{NN-types.pdf}
	
\end{frame}

\begin{frame}
	\frametitle{\inserttitle}
	\framesubtitle{\insertshortsubtitle: Plan}

	\begin{multicols}{2}
		%	\small
		\tableofcontents
	\end{multicols}
\end{frame}

\section{Neuron}

\begin{frame}
	\frametitle{\insertshortsubtitle}
	\framesubtitle{\insertsection}
	
	\begin{itemize}
		\item Let's remember \optword{Logistic regression}
		\item We make a linear combination of features' values: this can be seen as the accumulation of signals
		\[z(x) = \theta_0 + \sum\limits_{j=1}^{N} \theta_j x_j\]
		\item Then, we apply the logistic function to this sum: we call it an \keyword{activation function} (send the signal or not according to a threshold)
		\[\sigma(z) = \frac{1}{1 + e^{-z}}\]
		\item To calculate the classification error, we use a \keyword{cost function}
		\item To train the parameters, we use an \keyword{optimization function}
	\end{itemize}

\end{frame}

\begin{frame}
	\frametitle{\insertshortsubtitle}
	\framesubtitle{\insertsection}

\begin{center}
	\hgraphpage[0.6\textwidth]{aneuron.pdf}
\end{center}

\[\hat{y} = \varphi (b + \sum_{j=1}^n X_j W_j)\]

\end{frame}

\subsection{Network}

\begin{frame}
	\frametitle{\insertshortsubtitle}
	\framesubtitle{\insertsection: Motivation}
	
	\begin{minipage}{0.59\textwidth}
		\begin{itemize}
			\item To train complex functions, a single neuron is not enough
			\item For example, \expword{Two variables' XOR function}
			\item If we train a logistic regression model (a neuron with the logistic function as activation function), the algorithm tries to separate the samples with a decision line
			\item Whatever the decision line, there is at least one misclassified sample
		\end{itemize}
	\end{minipage}
	\begin{minipage}{0.4\textwidth}
		\hgraphpage[\textwidth]{xor.png}
	\end{minipage}

\end{frame}

\begin{frame}
	\frametitle{\insertshortsubtitle: \insertsection}
	\framesubtitle{\insertsubsection: Architecture}
	
	\begin{minipage}{0.59\textwidth}
		\begin{itemize}
			\item \optword{Input layer}: contains input values and no parameters
			\item \optword{Hidden layers}: each layer receives the outputs of the previous layer.
			It contains a number of neurons which calculate the weighted sum with an activation function.
			\item \optword{Output layer}: contains the neurons that calculate the output.
		\end{itemize}
	\end{minipage}
	\begin{minipage}{0.4\textwidth}
		\hgraphpage[\textwidth]{NN.pdf}
	\end{minipage}

\end{frame}

\begin{frame}
	\frametitle{\insertshortsubtitle: \insertsection}
	\framesubtitle{\insertsubsection: Some humor}
	
	\begin{center}
		\hgraphpage[0.6\textwidth]{humor/humor-neuron.jpeg}
	\end{center}
	
\end{frame}

\subsection{Activation functions}

\begin{frame}
	\frametitle{\insertshortsubtitle: \insertsection}
	\framesubtitle{\insertsubsection: Sigmoid functions}
	
	\begin{tabular}{ll}
		Logistic function & Hyperbolic tangent \\
		\hgraphpage[.4\textwidth]{logistique.png} & 
		\hgraphpage[.4\textwidth]{tanh.png} \\
		$\color{blue}\sigma(x) = \frac{1}{1 + e^{-x}}$ & 
		$\color{blue}tanh(x) = \frac{e^{x} - e^{-x}}{e^{x} + e^{-x}} = \frac{2}{1 + e^{-2x}} - 1$ \\
		
		$\sigma(x) \in ]0, 1[$ & 
		$tanh(x) \in ]-1, 1[$ \\
		
		$\color{red}\sigma'(x) = \sigma(x) (1-\sigma(x))$ & 
		$\color{red}tanh'(x) = 1 - tanh(x)^2$ \\
	\end{tabular}


\end{frame}

\begin{frame}
	\frametitle{\insertshortsubtitle: \insertsection}
	\framesubtitle{\insertsubsection: Sigmoid functions (2)}
	
	\begin{tabular}{ll}
		Arc tangent & Softsign \\
		\hgraphpage[.4\textwidth]{arctan.png} & 
		\hgraphpage[.4\textwidth]{so.png} \\
		$\color{blue}f(x) = \frac{2}{\pi} tan^{-1}(x)$ & 
		$\color{blue}f(x) = \frac{x}{1 + |x|}$ \\
		
		$f(x) \in ]-1, 1[$ & 
		$f(x) \in ]-1, 1[$ \\
		
		$\color{red}f'(x) = \frac{2}{\pi} \frac{1}{x^2 + 1}$ & 
		$\color{red}f'(x) = \frac{1}{(1 + |x|)^2}$ \\
	\end{tabular}

\end{frame}

\begin{frame}
	\frametitle{\insertshortsubtitle: \insertsection}
	\framesubtitle{\insertsubsection: Linear functions}

\begin{tabular}{ll}
	Linear function & Identity function \\
	\hgraphpage[.4\textwidth]{lineaire.png} & 
	\hgraphpage[.4\textwidth]{identite.png} \\
	$\color{blue} f(x) = k x$ & 
	$\color{blue} f(x) = x$ \\
	
	$f(x) \in ]-\infty, +\infty[$ & 
	$f(x) \in ]-\infty, +\infty[$ \\
	
	$\color{red}f'(x) = k$ & 
	$\color{red}f'(x) = 1$ \\
\end{tabular}

\end{frame}

\begin{frame}
	\frametitle{\insertshortsubtitle: \insertsection}
	\framesubtitle{\insertsubsection: Step functions}

\begin{tabular}{ll}
	Heaviside function & Sign function \\
	\hgraphpage[.4\textwidth]{heaviside.png} & 
	\hgraphpage[.4\textwidth]{signe.png} \\
	$\color{blue} H(x) = \begin{cases}
	0 & \text{if } x < 0 \\
	1 & \text{otherwise}
	\end{cases}$ & 
	$\color{blue} S(x) = \begin{cases}
	-1 & \text{if } x < 0 \\
	1 & \text{otherwise}
	\end{cases} = 2 H(x) - 1$ \\
	
	$H(x) \in \{0, 1\}$ & 
	$S(x) \in \{-1, 1\}$ \\
	
	$\color{red}H'(x) = \begin{cases}
	0 & \text{if } x \ne 0 \\
	+\infty & \text{otherwise}
	\end{cases}$ & 
	$\color{red}S'(x) = 2H'(x)$ \\
\end{tabular}

\end{frame}

\begin{frame}
	\frametitle{\insertshortsubtitle: \insertsection}
	\framesubtitle{\insertsubsection: Hockey stick functions}

\begin{tabular}{ll}
	Rectified Linear Unit & Parametric ReLU \\
	\hgraphpage[.4\textwidth]{relu.png} & 
	\hgraphpage[.4\textwidth]{prelu.png} \\
	$\color{blue} ReLU(x) = x H(x) = \max(0, x)$ & 
	$\color{blue} PReLU(\alpha, x) = \begin{cases}
	\alpha x & \text{if } x < 0 \\
	x & \text{otherwise}
	\end{cases}$ \\
	
	$ReLU(x) \in [0, +\infty[$ & 
	$PReLU(x) \in ]-\infty, +\infty[$ \\
	
	$\color{red}ReLU'(x) = H(x) = \begin{cases}
	0 & \text{if } x < 0 \\
	1 & \text{otherwise}
	\end{cases}$ & 
	$\color{red}PReLU'(x) = \begin{cases}
	\alpha & \text{if } x < 0 \\
	1 & \text{otherwise}
	\end{cases}$ \\
\end{tabular}

\end{frame}

\begin{frame}
	\frametitle{\insertshortsubtitle: \insertsection}
	\framesubtitle{\insertsubsection: Hockey stick functions (2)}

\begin{tabular}{ll}
	Exponential Linear Unit & Sigmoid Linear Unit \\
	\hgraphpage[.4\textwidth]{elu.png} & 
	\hgraphpage[.4\textwidth]{slu.png} \\
	$\color{blue} ELU(\alpha, x) = \begin{cases}
	\alpha (e^x - 1) & \text{if } x < 0 \\
	x & \text{otherwise}
	\end{cases}$ & 
	$\color{blue} SLU(x) = x \sigma(x) = \frac{x}{1 + e^{-x}}$ \\
	
	$ELU(\alpha, x) \in ]-\infty, +\infty[$ & 
	$SLU(x) \in ]-\infty, +\infty[$ \\
	
	$\color{red}ELU'(\alpha, x) = \begin{cases}
	ELU(\alpha, x) + \alpha & \text{if } x < 0 \\
	1 & \text{otherwise}
	\end{cases}$ & 
	$\color{red}SLU'(x) = \sigma(x) (1 + x \sigma(-x))$ \\
\end{tabular}

\end{frame}

\begin{frame}
	\frametitle{\insertshortsubtitle: \insertsection}
	\framesubtitle{\insertsubsection: Some humor}
	
	\begin{center}
		\hgraphpage[0.7\textwidth]{humor/humor-relu.png}
	\end{center}
	
\end{frame}

\subsection{Cost functions}

\begin{frame}
	\frametitle{\insertshortsubtitle: \insertsection}
	\framesubtitle{\insertsubsection: Regression (MSE)}
	
	\begin{columns}
		\begin{column}{.68\linewidth}
			\begin{block}{Mean Squared Error, L2 loss}
				\[MSE = \frac{1}{2} (y - \hat{y})^2\]
				
				\[
				\frac{\partial MSE}{\partial \hat{y}} = \hat{y} - y
				\]
			\end{block}
		\end{column}%
		\begin{column}{.3\linewidth}
			\begin{figure}
				\hgraphpage[\textwidth]{regression-loss_.pdf}
				\caption{Regression cost functions \cite{2017-rosenberg}}
			\end{figure}
		\end{column}
	\end{columns}


\end{frame}

\begin{frame}
	\frametitle{\insertshortsubtitle: \insertsection}
	\framesubtitle{\insertsubsection: Regression (MSE)}

	\begin{itemize}
		\item \optword{Advantages}
		\begin{itemize}
			\item A quadratic function has only one global minimum.
		\end{itemize}
		\item \optword{Disadvantages}
		\begin{itemize}
			\item MSE is less robust to outliers.
		\end{itemize}
	\end{itemize}

\end{frame}

\begin{frame}
	\frametitle{\insertshortsubtitle: \insertsection}
	\framesubtitle{\insertsubsection: Regression (MAE)}
	
	\begin{columns}
		\begin{column}{.68\linewidth}
			\begin{block}{Mean Absolute Error, L1 loss}
				\[MAE = |y - \hat{y}|\]
				
				\[
				\frac{\partial MAE}{\partial \hat{y}} = 
				\begin{cases}
					+1 & \text{ if } \hat{y} > y \\
					-1 & \text{ if } \hat{y} < y \\
					[-1, +1] & \text{ if } \hat{y} = y \\
				\end{cases}
				\]
			\end{block}
		\end{column}%
		\begin{column}{.3\linewidth}
			\begin{figure}
				\hgraphpage[\textwidth]{regression-loss_.pdf}
				\caption{Regression cost functions \cite{2017-rosenberg}}
			\end{figure}
		\end{column}
	\end{columns}

\end{frame}

\begin{frame}
	\frametitle{\insertshortsubtitle: \insertsection}
	\framesubtitle{\insertsubsection: Regression (MAE)}
	
	\begin{itemize}
		\item \optword{Advantages}
		\begin{itemize}
			\item Less sensitive to outliers
		\end{itemize}
		\item \optword{Disadvantages}
		\begin{itemize}
			\item The amplitude of the gradient does not depend on the size of the error, only on the sign of y - ŷ. This leads to the gradient magnitude being large even when the error is small
		\end{itemize}
	\end{itemize}

\end{frame}


\begin{frame}
	\frametitle{\insertshortsubtitle: \insertsection}
	\framesubtitle{\insertsubsection: Regression (Huber)}
	
	\begin{columns}
		\begin{column}{.68\linewidth}
			\begin{block}{Huber Loss}
				\[J = %\frac{1}{M} \sum\limits_{i=1}^{M} 
				\begin{cases}
					\frac{1}{2} (y - \hat{y})^2 & \text{if } |y - \hat{y}| \le \delta \\
					\delta |y - \hat{y}| - \frac{1}{2} \delta^2 & \text{otherwise }\\
				\end{cases}
				\]
				
				\[
				\frac{\partial J}{\partial \hat{y}} = 
				\begin{cases}
					\hat{y} - y & \text{ if } |\hat{y} - y| \le \delta \\
					-\delta & \text{ if } \hat{y} - y < -\delta \\
					+\delta & \text{ if } \hat{y} - y > \delta \\
				\end{cases}
				\]
				
			\end{block}
			
		\end{column}%
		\begin{column}{.3\linewidth}
			\begin{figure}
				\hgraphpage[\textwidth]{regression-loss_.pdf}
				\caption{Regression cost functions \cite{2017-rosenberg}}
			\end{figure}
		\end{column}
	\end{columns}

\end{frame}

\begin{frame}
	\frametitle{\insertshortsubtitle: \insertsection}
	\framesubtitle{\insertsubsection: Regression (Huber)}
	
	\begin{itemize}
		\item \optword{Advantages}
		\begin{itemize}
			\item Same advantages as MSE and MAE
		\end{itemize}
		\item \optword{Disadvantages}
		\begin{itemize}
			\item Hyper-parameter $\delta$ must be tuned
		\end{itemize}
	\end{itemize}

\end{frame}

\begin{frame}
	\frametitle{\insertshortsubtitle: \insertsection}
	\framesubtitle{\insertsubsection: Binary classification (BCE)}
	
	\begin{block}{Binary Cross Entropy Loss}
		\[J = - (y \log(\hat{y}) + (1 - y) \log(1 - \hat{y}))\]
		
		\[
		\frac{\partial J}{\partial \hat{y}} = \frac{\hat{y} - y}{\hat{y} - \hat{y}^2}
		\]
	\end{block}

	\begin{itemize}
		\item \optword{Advantages}
		\begin{itemize}
			\item works well with activation functions that model probabilities (sigmoid)
			\item Therefore, best estimate of the probabilities
		\end{itemize}
		\item \optword{Disadvantages}
		\begin{itemize}
			\item linear separation margin is not guaranteed to be optimal
		\end{itemize}
	\end{itemize}

\end{frame}

\begin{frame}
	\frametitle{\insertshortsubtitle: \insertsection}
	\framesubtitle{\insertsubsection: Binary classification (Hinge)}
	
	\begin{block}{Hinge Loss (SVM Loss)}
		\[J = \max(0, 1 - y \hat{y}), y \in \{-1, 1\}\]
		
		\[
		\frac{\partial J}{\partial \hat{y}} = 
		\begin{cases}
			0 & \text{if } y \hat{y} \ge 1 \\
			-y & \text{otherwise}
		\end{cases}
		\]
	\end{block}

	\begin{itemize}
		\item \optword{Advantages}
		\begin{itemize}
			\item results in better precision
		\end{itemize}
		\item \optword{Disadvantages}
		\begin{itemize}
			\item non-smooth and non-differentiable, therefore fewer optimization algorithms
			\item does not model the probability $p(y|x)$ directly
		\end{itemize}
	\end{itemize}

\end{frame}

\begin{frame}
	\frametitle{\insertshortsubtitle: \insertsection}
	\framesubtitle{\insertsubsection: Multi-class classification (Cross entropy)}
	
	\begin{block}{Cross Entropy Loss}
		\[J = - \sum\limits_{k=1}^{K} y_{k} \log(\hat{y}_{k}) \text{, où } K \text{ est le nnombre des classes} \]
		
		\[
		\frac{\partial J}{\partial \hat{y}_k} = - \frac{y_{k}}{\hat{y}_{k}}
		\]
	\end{block}

%	\begin{itemize}
%		\item \optword{Advantages}
%		\begin{itemize}
%			\item 
%		\end{itemize}
%		\item \optword{Disadvantages}
%		\begin{itemize}
%			\item 
%		\end{itemize}
%	\end{itemize}

\end{frame}

\begin{frame}
	\frametitle{\insertshortsubtitle: \insertsection}
	\framesubtitle{\insertsubsection: Some humor}
	
	\begin{center}
		\vgraphpage{humor/humor-loss.jpg}
	\end{center}
	
\end{frame}

\subsection{Optimization function}

\begin{frame}
	\frametitle{\insertshortsubtitle: \insertsection}
	\framesubtitle{\insertsubsection: Gradient Descent}
	
	\begin{figure}
		\vgraphpage[.7\textheight]{steepest.png}
		\caption{Illustration of gradient descent \cite{2020-calin}}
	\end{figure}

\end{frame}

\begin{frame}
	\frametitle{\insertshortsubtitle: \insertsection}
	\framesubtitle{\insertsubsection: Gradient Descent}
	
	\begin{algorithm}[H]
		\KwData{$ X, Y, \alpha, T $}
		\KwResult{$ \theta $}
		initialize $ \theta $ ; $ t = 0 $\;
		\While{$ t < T$ and no convergence}{
			$ \theta = \theta - \alpha \Delta_\theta J(X, Y; \theta) $\;
			%		Mettre à jours les $\theta$\;
			t = t + 1 \;
		}
		\caption{Gradient Descent}
	\end{algorithm}

\end{frame}

\begin{frame}
	\frametitle{\insertshortsubtitle: \insertsection}
	\framesubtitle{\insertsubsection: Gradient Descent}
	
	\begin{itemize}
		\item \optword{Properties}:
		\begin{itemize}
			\item Solve the optimal value in the direction of gradient descent.
			\item The method converges with a linear manner.
		\end{itemize}
		\item \optword{Advantages}:
		\begin{itemize}
			\item The solution is a global optimum when the objective function is convex.
		\end{itemize}
		\item \optword{Disadvantages}:
		\begin{itemize}
			\item For each parameter update, the gradients of all samples must be calculated. So the computational cost is very high.
			\item The entire workout may be too large to process in memory
			\item can converge to a local minimum when the objective function is not convex
		\end{itemize}
	\end{itemize}

\end{frame}

\begin{frame}
	\frametitle{\insertshortsubtitle: \insertsection}
	\framesubtitle{\insertsubsection: Stochastic Gradient Descent}
	
	\begin{algorithm}[H]
		\KwData{$ X, Y, \alpha, T$}
		\KwResult{$ \theta $}
		initialize $ \theta $ ; $ t = 0 $\;
		\While{$ t < T$ and no convergence}{
			Randomly choose $ (X^{(i)}, Y^{(i)}) \in (X, Y)$\;
			$ \theta = \theta - \alpha \Delta_\theta J(X^{(i)}, Y^{(i)}; \theta) $\;
			t = t + 1 \;
		}
		\caption{Stochastic Gradient Descent}
	\end{algorithm}

\end{frame}

\begin{frame}
	\frametitle{\insertshortsubtitle: \insertsection}
	\framesubtitle{\insertsubsection: Stochastic Gradient Descent}
	
	\begin{itemize}
		\item \optword{Properties}:
		\begin{itemize}
			\item Parameters are updated based on each sample
		\end{itemize}
		\item \optword{Advantages}:
		\begin{itemize}
			\item Can avoid local minima
		\end{itemize}
		\item \optword{Disadvantages}:
		\begin{itemize}
			% \item Frequent updating is more computationally expensive
			\item It is difficult for the model to converge to a minimum
		\end{itemize}
	\end{itemize}

\end{frame}


\begin{frame}
	\frametitle{\insertshortsubtitle: \insertsection}
	\framesubtitle{\insertsubsection: Mini-Batch Gradient Descent}
	
	\begin{algorithm}[H]
		\KwData{$ X, Y, \alpha, T, b $}
		\KwResult{$ \theta $}
		initialize $ \theta $ ; $ t = 0 $\;
		\While{$ t < T$ et pas de convergence}{
			Randomly shuffle the data\;
			Split the data into $b$ batches\;
			\lForEach{$ (X^b, Y^b) \in batches$}{
				$ \theta = \theta - \alpha \Delta_\theta J(X^b, Y^b; \theta) $
				%			Mettre à jours les $\theta$\;
			}
			\tcp{{\tiny We can update the parameters with the average of the steps}}
			t = t + 1 \;
		}
		\caption{Mini-Batch Gradient Descent}
	\end{algorithm}

\end{frame}

\begin{frame}
	\frametitle{\insertshortsubtitle: \insertsection}
	\framesubtitle{\insertsubsection: Mini-Batch Gradient Descent}
	
	\begin{itemize}
		\item \optword{Properties}:
		\begin{itemize}
			\item Training is done on subsets of the dataset
			\item A hyper-parameter $b$ for the size of a batch
		\end{itemize}
		\item \optword{Advantages}:
		\begin{itemize}
			\item Can avoid local minima due to frequent updates
			\item Effective in case of large training data
		\end{itemize}
		\item \optword{Disadvantages}:
		\begin{itemize}
			\item An additional hyper-parameter must be configured
		\end{itemize}
	\end{itemize}

\end{frame}


\begin{frame}
	\frametitle{\insertshortsubtitle: \insertsection}
	\framesubtitle{\insertsubsection: Adaptive Gradient Descent (AdaGrad)}
	
	\begin{algorithm}[H]
		\KwData{$ X, Y, \alpha, T$}
		\KwResult{$ \theta $}
		initialize $ \theta $ ; $ t = 0 $\;
		initialize $v$ ($|v| = |\theta|$) to zero\;
		\While{$ t < T$ and no convergence}{
			$ v = v + (\Delta_\theta J(X, Y; \theta))^2 $\;
			$ \theta = \theta - \frac{\alpha}{\sqrt{v + \epsilon}} \Delta_\theta J(X, Y; \theta) $;\tcp{{\tiny epsilon is used to avoid division by 0, typically 1e-8}}
			%		Mettre à jours les $\theta$\;
			
			t = t + 1 \;
		}
		\caption{Adaptive Gradient Descent \cite{2011-duchi-al}}
	\end{algorithm}

\end{frame}

\begin{frame}
	\frametitle{\insertshortsubtitle: \insertsection}
	\framesubtitle{\insertsubsection: Adaptive Gradient Descent (AdaGrad)}
	
	\begin{itemize}
		\item \optword{Properties}:
		\begin{itemize}
			\item The learning rate is adaptively adjusted based on the sum of squares of all historical gradients.
		\end{itemize}
		\item \optword{Advantages}:
		\begin{itemize}
			\item The learning rate of each parameter adjusts adaptively
		\end{itemize}
		\item \optword{Disadvantages}:
		\begin{itemize}
			\item Over time, the learning rate falls toward zero; the algorithm is not desirable for non-convex functions.
		\end{itemize}
	\end{itemize}

\end{frame}

\begin{frame}
	\frametitle{\insertshortsubtitle: \insertsection}
	\framesubtitle{\insertsubsection: Root Mean Square Propagation (RMSProp)}
	
	\begin{algorithm}[H]
		\KwData{$ X, Y, \alpha, T, \beta$}
		\KwResult{$ \theta $}
		initialize $ \theta $ ; $ t = 0 $\;
		initialize $v$ ($|v| = |\theta|$) to zero\;
		\While{$ t < T$ and no convergence}{
			$ v = \beta v + (1-\beta)(\Delta_\theta J(X, Y; \theta))^2 $\;
			$ \theta = \theta - \frac{\alpha}{\sqrt{v} + \epsilon} \Delta_\theta J(X, Y; \theta) $\;
			%		Mettre à jours les $\theta$\;
			t = t + 1 \;
		}
		\caption{RMSProp \cite{2014-hinton-al}}
	\end{algorithm}

\end{frame}

\begin{frame}
	\frametitle{\insertshortsubtitle: \insertsection}
	\framesubtitle{\insertsubsection: Root Mean Square Propagation (RMSProp)}

	\begin{itemize}
		\item \optword{Properties}:
		\begin{itemize}
			\item Change the total gradient accumulation mode to exponential moving average.
			\item By default, $\beta = 0.9$
		\end{itemize}
		\item \optword{Advantages}:
		\begin{itemize}
			\item Addresses the inefficient learning problem in late AdaGrad.
			\item Suitable for the optimization of non-stationary and non-convex problems.
		\end{itemize}
		\item \optword{Disadvantages}:
		\begin{itemize}
			\item At the end of training, the update process can be stuck around the local minimum.
		\end{itemize}
	\end{itemize}

\end{frame}

\begin{frame}
	\frametitle{\insertshortsubtitle: \insertsection}
	\framesubtitle{\insertsubsection: Adaptive Moment Estimation (Adam)}
	
	\begin{algorithm}[H]
		\KwData{$ X, Y, \alpha, T, \beta_1, \beta_2$}
		\KwResult{$ \theta $}
		initialize $ \theta $ ; $ t = 0 $\;
		initialize $v$ et $m$ ($|v| = |m| = |\theta|$) to zero\;
		\While{$ t < T$ and no convergence}{
			$ g = \Delta_\theta J(X, Y; \theta) $\;
			$ m = \beta_1 m + (1-\beta_1) g $; $\hat{m} = \frac{m}{1-\beta_1^t}$\;
			$ v = \beta_2 v + (1-\beta_2) g^2 $;
			$\hat{v} = \frac{v}{1-\beta_2^t}$\;
			$ \theta = \theta - \frac{\alpha \hat{m}}{\sqrt{\hat{v}} + \epsilon} $\;
			%		Mettre à jours les $\theta$\;
			t = t + 1 \;
		}
		\caption{Adam \cite{2015-kingma-ba}}
	\end{algorithm}

\end{frame}

\begin{frame}
	\frametitle{\insertshortsubtitle: \insertsection}
	\framesubtitle{\insertsubsection: Adaptive Moment Estimation (Adam)}
	
	\begin{itemize}
		\item \optword{Properties}:
		\begin{itemize}
			\item Use of first-order and second-order moments to dynamically adjust the learning rate for each parameter.
			\item Addition of bias correction.
			\item Default values: $\beta_1 = 0.9, \beta_2 = 0.999$ 
		\end{itemize}
		\item \optword{Advantages}:
		\begin{itemize}
			\item The gradient descent process is relatively stable. 
			\item It is suitable for most non-convex optimization problems with large datasets and high-dimensional spaces.
		\end{itemize}
		\item \optword{Disadvantages}:
		\begin{itemize}
			\item The method may not converge in certain cases.
		\end{itemize}
	\end{itemize}

\end{frame}

\begin{frame}
	\frametitle{\insertshortsubtitle: \insertsection}
	\framesubtitle{\insertsubsection: Others}
	
	\begin{itemize}
		\item \textbf{COCOB} \cite{orabona2017training} {\scriptsize \url{https://arxiv.org/pdf/1705.07795.pdf}}
		
		\item \textbf{Yogi} \cite{zaheer2018adaptive} {\scriptsize \url{https://proceedings.neurips.cc/paper/2018/file/90365351ccc7437a1309dc64e4db32a3-Paper.pdf}}
		
		\item \textbf{NovoGrad} \cite{ginsburg2019training} {\scriptsize \url{https://arxiv.org/pdf/1905.11286.pdf}} 
		
		\item \textbf{Lookahead} \cite{10.5555/3454287.3455148} {\scriptsize \url{https://arxiv.org/pdf/1907.08610v1.pdf}} 
		
		\item \textbf{LAMB} \cite{you2020large} {\scriptsize \url{https://arxiv.org/abs/1904.00962}} 
		
		\item \textbf{AdaBelief} \cite{zhuang2020adabelief} {\scriptsize \url{https://arxiv.org/pdf/2010.07468.pdf}}
		
%		\item \textbf{} \cite{} {\scriptsize \url{}} 
		
	\end{itemize}
	
\end{frame}

\begin{frame}
	\frametitle{\insertshortsubtitle: \insertsection}
	\framesubtitle{\insertsubsection: Some humor}
	
	\begin{center}
		\vgraphpage{humor/humor-gd.jpg}
	\end{center}
	
\end{frame}

\section{Feedforward Neural Networks (FFNN)}

\begin{frame}
	\frametitle{\insertshortsubtitle}
	\framesubtitle{\insertsection}
	
	
\end{frame}

\subsection{Architecture multicouches}

\begin{frame}
	\frametitle{\insertshortsubtitle: \insertsection}
	\framesubtitle{\insertsubsection: Notation}

	\begin{center}
		\hgraphpage[0.9\textwidth]{RNPA.pdf}
	\end{center}

\end{frame}

\begin{frame}
	\frametitle{\insertshortsubtitle: \insertsection}
	\framesubtitle{\insertsubsection: Example}

	\hgraphpage[\textwidth]{RNPA-exp.pdf}

\end{frame}

\begin{frame}
	\frametitle{\insertshortsubtitle: \insertsection}
	\framesubtitle{\insertsubsection: Backpropagation [example] (1)}
	
	Update $w_{11}^{(4)}$
	
	\small
	
	\[
	\frac{\partial J}{\partial w_{11}^{(4)}} = \overbrace{\frac{\partial J}{\partial f_{1}^{(4)}} \frac{\partial f_{1}^{(4)}}{\partial z_{1}^{(4)}}}^{\delta_{1}^{(4)}} \overbrace{\frac{\partial z_{1}^{(4)}}{\partial w_{11}^{(4)}}}^{a_{1}^{(3)}}
	\]
	
	$ 
	\frac{\partial J}{\partial f_{1}^{(4)}} = \frac{(0.840, 0.843) - (0, 1)}{(0.840, 0.843) - (0.840, 0.843)^2} 
	= (6.25, -1.186)
	$
	
	$ 
	\frac{\partial f_{1}^{(4)}}{\partial z_{1}^{(4)}} = (0.840, 0.843) (0.160, 0.157) = (0.134, 0.132)
	$
	
	$
	\delta_{1}^{(4)} = (6.25, -1.186) (0.134, 0.132) \approx (0.838, -0.157)
	$
	
	$
	\frac{\partial J}{\partial w_{11}^{(4)}} = avg((0.838, -0.157) (0.555, 0.612)) 
	\approx avg(0.465, -0.096) = 0.184
	$
	
	$
	w_{11}^{(4)} = 0.7 - 1 * (0.184) = 0.516 \text{ , supposing } \alpha = 1
	$

\end{frame}

\begin{frame}
	\frametitle{\insertshortsubtitle: \insertsection}
	\framesubtitle{\insertsubsection: Backpropagation [example] (2)}
	
	Update $w_{21}^{(4)}$
	
	\small
	
	\[
	\frac{\partial J}{\partial w_{21}^{(4)}} = \overbrace{\frac{\partial J}{\partial f_{1}^{(4)}} \frac{\partial f_{1}^{(4)}}{\partial z_{1}^{(4)}}}^{\delta_{1}^{(4)}} \overbrace{\frac{\partial z_{1}^{(4)}}{\partial w_{21}^{(4)}}}^{a_{2}^{(3)}}
	\]
	
	
	$
	\delta_{1}^{(4)} = (0.838, -0.157)
	$
	
	$
	\frac{\partial J}{\partial w_{21}^{(4)}} = avg((0.838, -0.157) (0.386, 0.360)) 
	\approx avg(0.323, -0.056) = 0.134
	$
	
	$
	w_{21}^{(4)} = 0.7 - 1 * (0.134) = 0.566 \text{ , supposing } \alpha = 1
	$

\end{frame}

\begin{frame}
	\frametitle{\insertshortsubtitle: \insertsection}
	\framesubtitle{\insertsubsection: Backpropagation [example] (3)}
	
	Update $w_{11}^{(3)}$
	
	\small
	
	\[
	\frac{\partial J}{\partial w_{11}^{(3)}} = 
	\overbrace{
		\overbrace{
			\frac{\partial J}{\partial f_{1}^{(4)}} 
			\frac{\partial f_{1}^{(4)}}{\partial z_{1}^{(4)}}
		}^{\delta_{1}^{(4)}} 
		\overbrace{
			\frac{\partial z_{1}^{(4)}}{\partial f_{1}^{(3)}}
		}^{w_{11}^{(4)}} 
		\frac{\partial f_{1}^{(3)}}{\partial z_{1}^{(3)}} 
	}^{\delta_{1}^{(3)}} 
	\overbrace{
		\frac{\partial z_{1}^{(3)}}{\partial w_{11}^{(3)}}
	}^{a_{1}^{(2)}}
	\text{ Here, we use the past } w_{11}^{(4)}
	\]
	
	$
	\frac{\partial f_{1}^{(3)}}{\partial z_{1}^{(3)}} \approx 
	(0.555, 0.612) (0.445, 0.388) = (0.247, 0.237)
	$
	
	$
	\delta_{1}^{(3)} = (0.838, -0.157) * 0.7 * (0.247, 0.237) \approx (0.145, -0.026)
	$
	
	$
	\frac{\partial J}{\partial w_{11}^{(2)}} = avg((0.145, -0.026) (0.622, 0.900)) 
	= avg(0.090, -0.023) = 0.033
	$
	
	$
	w_{11}^{(2)} = 0.3 - 1 * (0.033) = 0.267 \text{ , supposing } \alpha = 1
	$

\end{frame}


\begin{frame}
	\frametitle{\insertshortsubtitle: \insertsection}
	\framesubtitle{\insertsubsection: Backpropagation [example] (4)}
	
	Update $w_{12}^{(3)}$
	
	\small
	
	\[
	\frac{\partial J}{\partial w_{12}^{(3)}} = 
	\overbrace{
		\overbrace{
			\frac{\partial J}{\partial f_{1}^{(4)}} 
			\frac{\partial f_{1}^{(4)}}{\partial z_{1}^{(4)}}
		}^{\delta_{1}^{(4)}} 
		\overbrace{
			\frac{\partial z_{1}^{(4)}}{\partial f_{2}^{(3)}}
		}^{w_{11}^{(4)}} 
		\frac{\partial f_{2}^{(3)}}{\partial z_{2}^{(3)}} 
	}^{\delta_{2}^{(3)}} 
	\overbrace{
		\frac{\partial z_{2}^{(3)}}{\partial w_{12}^{(3)}}
	}^{a_{1}^{(2)}}
	\text{ Here, we use use the past } w_{12}^{(4)}
	\]
	
	$
	\frac{\partial f_{2}^{(3)}}{\partial z_{2}^{(3)}} \approx 
	(0.386, 0.360) (0.614, 0.64) = (0.237, 0.230)
	$
	
	$
	\delta_{2}^{(3)} = (0.838, -0.157) * 0.7 * (0.237, 0.230) \approx (0.139, -0.025)
	$
	
	$
	\frac{\partial J}{\partial w_{11}^{(2)}} = avg((0.139, -0.025) (0.622, 0.900)) 
	= avg(0.086, -0.023) = 0.032
	$
	
	$
	w_{11}^{(2)} = - 0.1 - 1 * (0.032) = - 0.132 \text{ , supposing } \alpha = 1
	$

\end{frame}


\begin{frame}
	\frametitle{\insertshortsubtitle: \insertsection}
	\framesubtitle{\insertsubsection: Backpropagation [example] (5)}
	
	Update $w_{11}^{(2)}$
	
	\small
	
	\[
	\frac{\partial J}{\partial w_{11}^{(2)}} = 
	\overbrace{ 
		\left(
		\delta_{1}^{(3)} 
		\overbrace{
			\frac{\partial z_{1}^{(3)}}{\partial f_{1}^{(2)}}
		}^{w_{11}^{(3)}}
		+
		\delta_{2}^{(3)} 
		\overbrace{
			\frac{\partial z_{2}^{(3)}}{\partial f_{1}^{(2)}}
		}^{w_{12}^{(3)}}
		\right)
		\frac{\partial f_{1}^{(2)}}{\partial z_{1}^{(2)}}
	}^{\delta_{1}^{(2)}} 
	\overbrace{
		\frac{\partial z_{1}^{(2)}}{\partial w_{11}^{(2)}}
	}^{a_{1}^{(1)}}
	\]
	
	$
	\frac{\partial f_{1}^{(2)}}{\partial z_{1}^{(2)}} = (0.622, 0.900) (0.378, 0.100) = (0.235, 0.09)
	$
	
	$
	\delta_{1}^{(2)} = \left((0.145, -0.026) * (0.3) + (0.139, -0.025) * (-0.1)\right) * (0.235, 0.09) \approx (0.006956, -0.00047683)
	$
	
	
	%$
	%\frac{\partial J}{\partial w_{11}^{(2)}} = avg((0.039, -0.01) (2, 3)) 
	%= avg(0.078, -0.03) = 0.024
	%$
	%
	%$
	%w_{11}^{(2)} = 0.5 - 1 * (0.024) = 0.476 \text{ , supposing } \alpha = 1
	%$

\end{frame}

\begin{frame}
	\frametitle{\insertshortsubtitle: \insertsection}
	\framesubtitle{\insertsubsection: Backpropagation [General case]}
	
	\begin{itemize}
		\item $\delta^{(l)}$  are calculated, where $l$ is the layer's order
	\end{itemize}
	
	\[ 
	\delta^{(out)} = 
	\frac{\partial J}{\partial f^{(out)}} \frac{\partial f^{(out)}}{\partial z^{(out)}}
	\,,\,
	\delta^{(l)} = \frac{\partial f^{(l)}}{\partial z^{(l)}} w^{(l+1)} \delta^{(l+1)}
	\]
	
	\begin{itemize}
		\item Gradients are calculated
	\end{itemize}
	
	\[ 
	\frac{\partial J}{\partial w^{(l)}} = a^{(l-1)} \delta^{(l)}
	\,,\,
	\frac{\partial J}{\partial b^{(l)}} = \delta^{(l)}
	\]
	
	\begin{itemize}
		\item Parameters are updated
	\end{itemize}
	
	\[ 
	w = w - \alpha \frac{\partial J}{\partial w^{(l)}}
	\,,\,
	b = b - \alpha \frac{\partial J}{\partial b^{(l)}}
	\]

\end{frame}

\begin{frame}
	\frametitle{\insertshortsubtitle: \insertsection}
	\framesubtitle{\insertsubsection: Backpropagation (Some humor)}
	
	\hgraphpage[.62\textwidth]{humor/humor-deep.jpg}
	\hgraphpage[.36\textwidth]{humor/humor-retro3.jpg}

\end{frame}

\subsection{Auto-encoders}

\begin{frame}
	\frametitle{\insertshortsubtitle: \insertsection}
	\framesubtitle{\insertsubsection}
	
	\begin{itemize}
		\item It is a multi-layer neural network.
		\item It learns a compression algorithm.
		\item It is characterized by \cite{2016-keras}:
		\begin{itemize}
			\item Data-driven: unlike compression algorithms like JPEG.
			\item Lossy data: the constructed output is not entirely identical to the input.
			\item Unsupervised learning.
		\end{itemize}
		\item It is not particularly effective for the compression task.
	\end{itemize}

\end{frame}

\begin{frame}
	\frametitle{\insertshortsubtitle: \insertsection}
	\framesubtitle{\insertsubsection: Architecture}

	\hgraphpage[\textwidth]{AE.pdf}

\end{frame}

\begin{frame}
	\frametitle{\insertshortsubtitle: \insertsection}
	\framesubtitle{\insertsubsection: Denoising Auto-encoder}

	\hgraphpage[\textwidth]{AE-noise.pdf}

\end{frame}

\begin{frame}
	\frametitle{\insertshortsubtitle: \insertsection}
	\framesubtitle{\insertsubsection: Denoising Auto-encoder (Some applications)}
	
	\begin{itemize}
		\item Improving the quality of speech (audio) \cite{2013-lu}.
		\item Cleaning dirty documents (refer to this competition: \url{https://www.kaggle.com/c/denoising-dirty-documents}).
		\item Retrieving historical documents \cite{2019-neji}.
		\item Completing hidden parts of the face \cite{2017-li-al}.
	\end{itemize}

\end{frame}

\begin{frame}
	\frametitle{\insertshortsubtitle: \insertsection}
	\framesubtitle{\insertsubsection: Sparse Auto-encoder}
	
	\begin{itemize}
		\item Size of hidden layers must be equal to or larger than the input size.
		\item It is used to automatically learn features (feature engineering) to be used in another task, such as classification.
		\item Parameter (weight) regularization is applied.
		\item \textbf{L1} and \textbf{Kullback-Leibler divergence} are used.
		\item Some parameters converge to zero, allowing the learning of representations such as the contours of an object in an image.
	\end{itemize}

\end{frame}

\begin{frame}
	\frametitle{\insertshortsubtitle: \insertsection}
	\framesubtitle{\insertsubsection: Variational Auto-encoder (Motivation)}
	
	\begin{itemize}
		\item {\scriptsize\url{https://github.com/projeduc/ESI_ML/blob/main/demos/NN/TF_Autoencoder.ipynb}}
	\end{itemize}
	
	\begin{minipage}{0.47\textwidth} 
		\begin{center}
			Clustering Auto-encoder
		\end{center}\vskip-6pt
		\hgraphpage[\textwidth]{auto-enc.png}
	\end{minipage}
	%
	\begin{minipage}{0.47\textwidth}
		\begin{center}
			Variational Auto-encoder
		\end{center}\vskip-6pt
		\hgraphpage[\textwidth]{auto-enc-var.png}
	\end{minipage}
	
\end{frame}

\begin{frame}
	\frametitle{\insertshortsubtitle: \insertsection}
	\framesubtitle{\insertsubsection: Variational Auto-encoder}
	
	\begin{itemize}
		\item To generate new content, some random values can be input into the decoder.
		\item The problem is that the encoder learns to perfectly separate the different clusters.
		\item Solution: force the auto-encoder to learn a distribution.
	\end{itemize}

	\begin{minipage}{0.60\textwidth} 
		\begin{align*}
			z = \mu + \sigma * N(0, 1) \\
			J'(x, \hat{x}) = J(x, \hat{x}) + KL(N(\mu, \sigma), N(0, 1)) \\
			KL(p||q) = \sum_i p(x_i) log(\frac{p(x_i)}{q(x_i)})
		\end{align*}
	\end{minipage}
	%
	\begin{minipage}{0.39\textwidth}
		\hgraphpage[\textwidth]{AE-var.pdf}
	\end{minipage}

\end{frame}

\begin{frame}
	\frametitle{\insertshortsubtitle: \insertsection}
	\framesubtitle{\insertsubsection: Variational Auto-encoder (Summary)}
	
	\begin{itemize}
		\item Standard auto-encoder learns a perfect separation of clusters (non-smooth representation of latent states).
		\item \textbf{Solution}: force the model to learn a distribution (typically similar to the normal distribution).
		\item The model may learn broad distributions.
		\item \textbf{Solution}: use regularization (KL Divergence) to narrow them.
		\item The model may learn negative values for $\sigma$.
		\item \textbf{Solution}: learn $\log \sigma$.
	\end{itemize}

\end{frame}

\begin{frame}
	\frametitle{\insertshortsubtitle: \insertsection}
	\framesubtitle{\insertsubsection: Some humor}

	\begin{center}
		\hgraphpage[0.9\textwidth]{humor/humor-unsupervised.jpg}
	\end{center}

\end{frame}

\subsection{Convolutional Neural Network (CNN)}

\begin{frame}
	\frametitle{\insertshortsubtitle: \insertsection}
	\framesubtitle{\insertsubsection: Image processing (traditional architecture)}

	\begin{center}
		\hgraphpage[\textwidth]{img-learn.pdf}
	\end{center}

\end{frame}

\begin{frame}
	\frametitle{\insertshortsubtitle: \insertsection}
	\framesubtitle{\insertsubsection: Image processing (preprocessing)}
	
	\begin{itemize}
		\item There are several preprocessing techniques (see \cite{2010-alginahi}).
		\item We focus on convolution-based techniques (modifying the value of a pixel relative to its neighbors).
		\item Two parameters: \optword{padding} (surrounding the image with zeros to preserve the original size) and \optword{stride} (the step size of the kernel/mask as it slides).
	\end{itemize}

	\begin{center}
		\hgraphpage[.8\textwidth]{conv.pdf}
	\end{center}

\end{frame}

\begin{frame}
	\frametitle{\insertshortsubtitle: \insertsection}
	\framesubtitle{\insertsubsection: Image processing (example of preprocessing)}

	An example from \href{https://en.wikipedia.org/wiki/Kernel_(image_processing)}{Wikipedia}
	
	\begin{tabular}{p{.3\textwidth}p{.1\textwidth}p{.2\textwidth}p{.2\textwidth}}
		\hline\hline
		Operation & Original & Kernel & Transformed \\
		\hline
		Edge detection & 
		\graphpage[valign=c]{Vd-Orig.png} & 
		$\begin{bmatrix}
			-1 & -1 & -1\\ 
			-1 & 8 & -1\\ 
			-1 & -1 & -1
		\end{bmatrix}$ & 
		\graphpage[valign=c]{Vd-Edge3.png} \\
		
		\hline
		Sharpen & 
		\graphpage[valign=c]{Vd-Orig.png} & 
		$\begin{bmatrix}
			0 & -1 & 0\\ 
			-1 & 5 & -1\\ 
			0 & -1 & 0
		\end{bmatrix}$ & 
		\graphpage[valign=c]{Vd-Sharp.png} \\
		
		\hline
		Box blur & 
		\graphpage[valign=c]{Vd-Orig.png} & 
		$\frac{1}{9}\begin{bmatrix}
			1 & 1 & 1\\ 
			1 & 1 & 1\\ 
			1 & 1 & 1
		\end{bmatrix}$ & 
		\graphpage[valign=c]{Vd-Blur2.png} \\
		\hline\hline
		
	\end{tabular}

\end{frame}

\begin{frame}
	\frametitle{\insertshortsubtitle: \insertsection}
	\framesubtitle{\insertsubsection: Limitations of the previous solution}
	
	According to LeCun and his colleagues \cite{1998-lecun}:
	\begin{itemize}
		\item When the size of the images is large, there will be a large number of parameters to train.
		\item To achieve this, a large dataset must be provided.
		\item The memory required to store the parameters will be very large.
		\item To train the model on images, preprocessing such as translations and distortions must be applied.
		\item Variations in images (such as the position of an object) can only be captured when multiple layers are used.
	\end{itemize}

\end{frame}

\begin{frame}
	\frametitle{\insertshortsubtitle: \insertsection}
	\framesubtitle{\insertsubsection: Conv2D layer}
	
	\begin{minipage}{0.60\textwidth} 
		\begin{itemize}
			\item Preserve images' spatial structure.
			\item $ w' = \frac{w - w_f + 2P}{S} + 1$, $ h' = \frac{h - h_f + 2P}{S} + 1$
			\item A layer can have $k$ filters/kernels.
			\item Number of parameters: $w_f * h_f * c * k$ plus $k$ biases.
			\item Example, \expword{image: 32x32x3;} \expword{kernel: 5x5; s: 1; p: 0; k: 6.} \\ Number of trainable parameters: \expword{456}
		\end{itemize}
	\end{minipage}
	%
	\begin{minipage}{0.39\textwidth}
		\hgraphpage[\textwidth]{conv2d.pdf}
		
		\hgraphpage[.7\textwidth]{conv2d-exp_.pdf}
	\end{minipage}

\end{frame}

\begin{frame}
	\frametitle{\insertshortsubtitle: \insertsection}
	\framesubtitle{\insertsubsection: Conv2D layer (Example)}
	
	
	\begin{center}
		\vskip-6pt\hgraphpage[0.8\textwidth]{conv2D_imlp.pdf}
	\end{center}\vskip-16pt

	{\scriptsize 
		\[O_{11} = ReLU(\Sigma = b + F_{11} I_{11} + \textcolor{blue}{F_{12} I_{12}} + F_{21} I_{21} + \textcolor{red}{F_{22} I_{22}})  \]
		\[O_{12} = ReLU(\Sigma = b + \textcolor{blue}{F_{11} I_{12}} + F_{12} I_{13} + \textcolor{red}{F_{21} I_{22}} + F_{22} I_{23})  \]
		\[O_{21} = ReLU(\Sigma = b + F_{11} I_{21} + \textcolor{red}{F_{12} I_{22}} + F_{21} I_{31} + F_{22} I_{32})  \]
		\[O_{22} = ReLU(\Sigma = b + \textcolor{red}{F_{11} I_{22}} + F_{12} I_{23} + F_{21} I_{32} + F_{22} I_{33})  \]
	    	
	    \[\textcolor{red}{\frac{\partial J}{\partial I_{22}}} 
		= \underbrace{\frac{\partial J}{\partial O_{11}}}_{G_{11} = 1} 
		\underbrace{\frac{\partial ReLU}{\partial \sum}}_{1} 
		\underbrace{\frac{\partial \sum}{\partial I_{22}}}_{F_{22} = 3}
		+ \underbrace{\frac{\partial J}{\partial O_{12}}}_{G_{12} = 2} 
		\underbrace{\frac{\partial ReLU}{\partial \sum}}_{1} 
		\underbrace{\frac{\partial \sum}{\partial I_{22}}}_{F_{21} = -2}
		+ \underbrace{\frac{\partial J}{\partial O_{21}}}_{G_{21} = 3} 
		\underbrace{\frac{\partial ReLU}{\partial \sum}}_{1} 
		\underbrace{\frac{\partial \sum}{\partial I_{22}}}_{F_{12} = -1}
		+ \underbrace{\frac{\partial J}{\partial O_{22}}}_{G_{22} = 4} 
		\underbrace{\frac{\partial ReLU}{\partial \sum}}_{0} 
		\underbrace{\frac{\partial \sum}{\partial I_{22}}}_{F_{11} = 1}\]
		
		\[\textcolor{blue}{\frac{\partial J}{\partial I_{12}}} 
		= \underbrace{\frac{\partial J}{\partial O_{11}}}_{G_{11} = 1} 
		\underbrace{\frac{\partial ReLU}{\partial \sum}}_{1} 
		\underbrace{\frac{\partial \sum}{\partial I_{12}}}_{F_{12} = -1}
		+ \underbrace{\frac{\partial J}{\partial O_{12}}}_{G_{12} = 2} 
		\underbrace{\frac{\partial ReLU}{\partial \sum}}_{1} 
		\underbrace{\frac{\partial \sum}{\partial I_{12}}}_{F_{11} = 1}\]
	}

\end{frame}

\begin{frame}
	\frametitle{\insertshortsubtitle: \insertsection}
	\framesubtitle{\insertsubsection: Pooling}
	
	\begin{minipage}{0.60\textwidth} 
		\begin{itemize}
			\item Make the representation smaller and more manageable.
			\item $ w' = \frac{w - w_f + 2P}{S} + 1$, $ h' = \frac{h - h_f + 2P}{S} + 1$
			\item No parameters.
			\item \optword{Max Pool}: The gradient is only passed to the winning cell. If there are multiple max values, choose one (usually the first).
			\item \optword{Average Pool}: The average gradient is passed to participating cells. Each cell will have a gradient $\frac{\text{gradient}}{w_f * h_f}$.
		\end{itemize}
	\end{minipage}
	%
	\begin{minipage}{0.39\textwidth}
		\hgraphpage[\textwidth]{maxpool.pdf}
		
		\hgraphpage[.8\textwidth]{pool-exp_.pdf}
	\end{minipage}

\end{frame}

\begin{frame}
	\frametitle{\insertshortsubtitle: \insertsection}
	\framesubtitle{\insertsubsection: MaxPool2D layer (Example)}
	
	
	\begin{center}
		\vskip-6pt\hgraphpage[0.7\textwidth]{pool2D_imlp.pdf}
	\end{center}\vskip-16pt
	
	{\scriptsize 
		\[O_{11} = Max(I_{11} + \textcolor{blue}{I_{12}} + I_{21} + \textcolor{red}{I_{22}}) = \textcolor{red}{I_{22}}  \]
		\[O_{12} = Max(\textcolor{blue}{I_{12}} + I_{13} + \textcolor{red}{I_{22}} + I_{23}) = I_{23}  \]
		\[O_{21} = Max(I_{21} + \textcolor{red}{I_{22}} + I_{31} + I_{32}) = \textcolor{red}{I_{22}} \]
		\[O_{22} = Max(\textcolor{red}{I_{22}} + I_{23} + I_{32} + I_{33}) = \textcolor{red}{I_{22}}  \]
		
		\[\textcolor{red}{\frac{\partial J}{\partial I_{22}}} 
		= \underbrace{\frac{\partial J}{\partial O_{11}}}_{G_{11} = 1} 
		\underbrace{\frac{\partial Max}{\partial I_{22}}}_{1}
		+ \underbrace{\frac{\partial J}{\partial O_{12}}}_{G_{12} = 2} 
		\underbrace{\frac{\partial Max}{\partial I_{22}}}_{0}
		+ \underbrace{\frac{\partial J}{\partial O_{21}}}_{G_{21} = 3} 
		\underbrace{\frac{\partial Max}{\partial I_{22}}}_{1}
		+ \underbrace{\frac{\partial J}{\partial O_{22}}}_{G_{22} = 4} 
		\underbrace{\frac{\partial Max}{\partial I_{22}}}_{1}\]
		
		\[\textcolor{blue}{\frac{\partial J}{\partial I_{12}}} 
		= \underbrace{\frac{\partial J}{\partial O_{11}}}_{G_{11} = 1} 
		\underbrace{\frac{\partial Max}{\partial I_{12}}}_{0}
		+ \underbrace{\frac{\partial J}{\partial O_{12}}}_{G_{12} = 2} 
		\underbrace{\frac{\partial Max}{\partial I_{12}}}_{0}\]
	}
	
\end{frame}

\begin{frame}
	\frametitle{\insertshortsubtitle: \insertsection}
	\framesubtitle{\insertsubsection: Some humor}

	\begin{center}
		\vgraphpage{humor/humor-conv2d.jpg}
	\end{center}

\end{frame}

\subsection{Regularization}

\begin{frame}
	\frametitle{\insertshortsubtitle: \insertsection}
	\framesubtitle{\insertsubsection: Droupout}
	
	\begin{itemize}
		\item When training large neural networks on a small dataset, overfitting can occur.
		\item One solution is to reduce the size of the network.
		\item Another is to temporarily deactivate connections in a random manner.
		\item This technique is called \optword{dropout}.
		\item The dropout layer takes a dropout rate as a parameter.
		\item This layer deactivates some outputs (considers them as 0) during training (not during inference).
	\end{itemize}

\end{frame}


\begin{frame}
	\frametitle{\insertshortsubtitle: \insertsection}
	\framesubtitle{\insertsubsection: Some humor}

	\hgraphpage[.35\textwidth]{humor/humor-overfiting2.jpeg}
	\hgraphpage[.63\textwidth]{humor/humor-overfiting1.png}

\end{frame}


\section{Recurrent Neural Networks (RNN)}

\begin{frame}
	\frametitle{\insertshortsubtitle}
	\framesubtitle{\insertsection}
	
	
\end{frame}


\subsection{Architecture (RNN)}

\begin{frame}
	\frametitle{\insertshortsubtitle: \insertsection}
	\framesubtitle{\insertsubsection}
	
	\begin{minipage}{0.49\textwidth} 
		\begin{itemize}
			\item \optword{Elman network}
			\begin{align*}
				h_t = f(w_x x_t + w_h \textcolor{red}{h_{t-1}} + b_h) \\
				\hat{y}_t = g(w_y h_t + b_y)
			\end{align*}
			\item \optword{Jordan network}
			\begin{align*}
				h_t = f(w_x x_t + w_h \textcolor{red}{\hat{y}_{t-1}} + b_h) \\
				\hat{y}_t = g(w_y h_t + b_y)
			\end{align*}
		\end{itemize}
	\end{minipage}
	%
	\begin{minipage}{0.5\textwidth}
		\hgraphpage[\textwidth]{RNN.pdf}
	\end{minipage}

\end{frame}

\begin{frame}
	\frametitle{\insertshortsubtitle: \insertsection}
	\framesubtitle{\insertsubsection: Backpropagation}
	
	\begin{itemize}
		\item Define the cost function for each output.
		\item Use backpropagation through time \cite{1990-werbos}
		\begin{itemize}
			\item Unroll the recurrent network over time.
			\item The network will be similar to a feed-forward one.
			\item Accumulate gradients starting from the last state.
			\item Update parameters when reaching the first state.
		\end{itemize}
	\end{itemize}

\end{frame}

\begin{frame}
	\frametitle{\insertshortsubtitle: \insertsection}
	\framesubtitle{\insertsubsection: Applications}
	
	\begin{tabular}{p{.32\textwidth}p{.15\textwidth}p{.4\textwidth}}
		%			\hline\hline
		\textbf{Type} & \textbf{Illustration} & \textbf{Example} \\
		%			\hline
		Many to many & 
		\vgraphpage[1.4cm, valign=c]{RNNpp1.pdf} & 
		Named entity detection \\
		
		%			\hline
		Many to many (Seq2seq) & 
		\vgraphpage[1.4cm, valign=c]{RNNpp2.pdf} & 
		Machine translation \\
		
		%			\hline
		Many to one & 
		\vgraphpage[1.4cm, valign=c]{RNNp1.pdf} & 
		Sentiment classification \\
		
		%			\hline
		One to many & 
		\vgraphpage[1.4cm, valign=c]{RNN1p.pdf} & 
		Image caption generation \\
		
		%			\hline\hline
		
	\end{tabular}

\end{frame}


\begin{frame}
	\frametitle{\insertshortsubtitle: \insertsection}
	\framesubtitle{\insertsubsection: Applications (direct many to many}
%	\vspace*{-0.5cm}
	\[ h_t = f(x_t, h_{t-1}, W_x, W_h),\,\,\, \hat{y}_t = g(h_t, W_y),\,\,\,  J(\hat{y}, y) = \frac{1}{T} \sum_{t=1}^{T} j(\hat{y}_t, y_t)\]

	\begin{minipage}{0.6\textwidth}\scriptsize
		\begin{align*}
			\frac{\partial J}{\partial W_h} & = \frac{1}{T} \sum_{t=1}^{T} \frac{\partial j(\hat{y}_t, y_t)}{\partial W_h}
			= \frac{1}{T} \sum_{t=1}^{T} \frac{\partial j(\hat{y}_t, y_t)}{\partial \hat{y}_t} 
			\frac{\partial g(h_t, W_y)}{\partial h_t} 
			\frac{\partial h_t}{\partial W_h} \\
			\frac{\partial h_t}{\partial W_h} & = 
			\frac{\partial f(x_t, h_{t-1}, W_x, W_h)}{\partial W_h} + 
			\frac{\partial f(x_t, h_{t-1}, W_x, W_h)}{\partial h_{t-1}} \frac{\partial h_{t-1}}{\partial W_h} \\
			\frac{\partial J}{\partial W_y} & =
			\frac{1}{T} \sum_{t=1}^{T} \frac{\partial j(\hat{y}_t, y_t)}{\partial W_y}
			= \frac{1}{T} \sum_{t=1}^{T} \frac{\partial j(\hat{y}_t, y_t)}{\partial \hat{y}_t} 
			\frac{\partial g(h_t, W_y)}{\partial W_y} \\
			\frac{\partial J}{\partial W_x} & = 
			\frac{1}{T} \sum_{t=1}^{T} \frac{\partial j(\hat{y}_t, y_t)}{\partial W_x}
			= \frac{1}{T} \sum_{t=1}^{T} \frac{\partial j(\hat{y}_t, y_t)}{\partial \hat{y}_t} 
			\frac{\partial g(h_t, W_y)}{\partial h_t} 
			\frac{\partial h_t}{\partial W_x} \\
			\frac{\partial h_t}{\partial W_x} & = 
			\frac{\partial f(x_t, h_{t-1}, W_x, W_h)}{\partial W_x} + 
			\frac{\partial f(x_t, h_{t-1}, W_x, W_h)}{\partial h_{t-1}} \frac{\partial h_{t-1}}{\partial W_x} \\
		\end{align*}
	\end{minipage}
	\begin{minipage}{0.38\textwidth}
		\hgraphpage[\textwidth]{RNNpp1_exp.pdf}
	\end{minipage}
	
\end{frame}

\begin{frame}
	\frametitle{\insertshortsubtitle: \insertsection}
	\framesubtitle{\insertsubsection: Applications (indirect many to many)}
	
	\vskip-12pt

	\[ \text{Encoder: } h^e_t = f^e(x_t, h^e_{t-1}, W^e_x, W^e_h), \,\,\, \hat{y}_1 = g^e(h^e_{T}, W_y^e)\]
	
	\[ \text{Decoder: } h^d_{t} = f^d(\hat{y}_{t}, h^d_{t-1}, W^d_x, W^d_h), \,\,\, \hat{y}_{t} = g^d(h^d_{t-1}, W^d_y)\]

	\[ J(\hat{y}, y) = \frac{1}{T'} \sum_{t=1}^{T'} j(\hat{y}_{t}, y_t)\]
	
	\begin{center}
		\hgraphpage[0.8\textwidth]{RNNpp2_exp.pdf}
	\end{center}
\end{frame}

\begin{frame}
	\frametitle{\insertshortsubtitle: \insertsection}
	\framesubtitle{\insertsubsection: Applications (many to one)}
	
	\vspace*{-0.5cm}
	\[ h_t = f(x_t, h_{t-1}, W_x, W_h),\,\,\, \hat{y} = g(h_T, W_y),\,\,\,  J(\hat{y}, y) = j(\hat{y}, y)\]
	
%	\vspace*{-0.5cm}
	\begin{minipage}{0.6\textwidth}\scriptsize
		\begin{align*}
			\frac{\partial J}{\partial W_h} &= \frac{\partial j(\hat{y}, y)}{\partial W_h}
			= \frac{\partial j(\hat{y}, y)}{\partial \hat{y}} 
			\frac{\partial g(h_T, W_y)}{\partial h_T} 
			\frac{\partial h_T}{\partial W_h} \\
			\frac{\partial h_t}{\partial W_h} &= 
			\frac{\partial f(x_t, h_{t-1}, W_x, W_h)}{\partial W_h} + 
			\frac{\partial f(x_t, h_{t-1}, W_x, W_h)}{\partial h_{t-1}} \frac{\partial h_{t-1}}{\partial W_h}\\
		\end{align*}
	\end{minipage}
	\begin{minipage}{0.38\textwidth}
		\hgraphpage{RNNp1_exp.pdf}
	\end{minipage}

\end{frame}

\begin{frame}
	\frametitle{\insertshortsubtitle: \insertsection}
	\framesubtitle{\insertsubsection: Applications (one to many)}
	
	\vspace*{-0.5cm}
	\[ J(\hat{y}, y) = \frac{1}{T} \sum_{t=1}^{T} j(\hat{y}_{t}, y_t)\]
	
%	\vspace*{-0.5cm}
	\begin{minipage}{0.6\textwidth}
		\begin{align*}
			h_1 & = f(x, h_{0}, W_x, W_h) \\
			h_{t} & = f(\hat{y}_{t-1}, h_{t-1}, W_y, W_h) \\
			\hat{y}_{t} & = g(h_{t-1}, W_y) \\
		\end{align*}
	\end{minipage}
	\begin{minipage}{0.38\textwidth}
		\hgraphpage{RNN1p_exp.pdf}
	\end{minipage}
%\begin{center}
%	\hgraphpage[0.4\textwidth]{RNN1p_exp.pdf}
%\end{center}

\end{frame}

\begin{frame}
	\frametitle{\insertshortsubtitle: \insertsection}
	\framesubtitle{\insertsubsection: Bidirectional RNN}
	
	\begin{center}
		\hgraphpage[0.8\textwidth]{bi-rnn_exp.pdf}
	\end{center}
	
\end{frame}

\begin{frame}
	\frametitle{\insertshortsubtitle: \insertsection}
	\framesubtitle{\insertsubsection: Some humor}
	
	\begin{center}
		\vgraphpage{humor/humor-rnn.jpg}
	\end{center}
	
\end{frame}


\subsection{Long Short-Term Memory (LSTM)}

\begin{frame}
	\frametitle{\insertshortsubtitle: \insertsection}
	\framesubtitle{\insertsubsection: Architecture}
	
	\begin{minipage}{0.50\textwidth} 
		\begin{itemize}
			\item $i$: \optword{input gate};
			How much information will be added to the cell's intern state?
			\item $f$: \optword{Forget gate};
			Must the past state be forgotten?
			\item $o$: \optword{Output gate} ;
			Combining the cell's intern and current states	
		\end{itemize}
	\end{minipage}
	%
	\begin{minipage}{0.49\textwidth}
		\hgraphpage[\textwidth]{LSTM.pdf}
	\end{minipage}
	
	\begin{itemize}
		\item $g$ : \optword{Input node};
		Will the state be  increased or decreased?
		(mostly considered to be a part of input gate)
		
	\end{itemize}

\end{frame}

\begin{frame}
	\frametitle{\insertshortsubtitle: \insertsection}
	\framesubtitle{\insertsubsection: Equations}
	\huge\vskip-24pt
	\begin{align*}
	f_t &= \sigma(W_f X_t + U_f h_{t-1} + b_f) \\
	i_t &= \sigma(W_i X_t + U_i h_{t-1} + b_i) \\
	g_t &= \tanh(W_g X_t + U_g h_{t-1} + b_g) \\
	o_t &= \tanh(W_o X_t + U_o h_{t-1} + b_o) \\
	c_t &= f_t \circ c_{t-1} \oplus i_t \circ g_t \\
	h_t &= o_t + \tanh(c_t)
	\end{align*}
\end{frame}

\begin{frame}
	\frametitle{\insertshortsubtitle: \insertsection}
	\framesubtitle{\insertsubsection: Some humor}

	\begin{center}
		\hgraphpage[.5\textwidth]{humor/humor-lstm.jpg}
	\end{center}

\end{frame}

\subsection{Gated Recurrent Unit (GRU)}

\begin{frame}
	\frametitle{\insertshortsubtitle: \insertsection}
	\framesubtitle{\insertsubsection: Architecture}
	
	\begin{minipage}{0.50\textwidth} 
		\begin{itemize}
			\item $z$: \optword{Update gate};
			Merges LSTM's input and forget gates. 
			\item $r$ : \optword{Reset gate}; 
			How much update from the past state?
			\item $h'$ : \optword{Candidate hidden state};
			Current state.	
		\end{itemize}
	\end{minipage}
	%
	\begin{minipage}{0.49\textwidth}
		\hgraphpage[\textwidth]{GRU.pdf}
	\end{minipage}

\end{frame}

\begin{frame}
	\frametitle{\insertshortsubtitle: \insertsection}
	\framesubtitle{\insertsubsection: Equations}
	
	{\huge\vskip-24pt
	\begin{align*}
	z_t &= \sigma(W_z X_t + U_z h_{t-1} + b_z) \\
	r_t &= \sigma(W_r X_t + U_r h_{t-1} + b_r) \\
	h'_t &= \tanh(W X_t + U (h_{t-1} \circ r_t) + b) \\
	h_t &= z_t \circ h_{t-1} \oplus (1-z_t) \circ h'_{t-1}
	\end{align*}
	}
	
	\vskip12pt
	GRU Backpropagation Tutorial:
	
	\url{https://www.math.ucla.edu/~minchen/doc/BPTTTutorial.pdf}
	
\end{frame}

\begin{frame}
	\frametitle{\insertshortsubtitle: \insertsection}
	\framesubtitle{\insertsubsection: Some humor}
	
	\begin{center}
		\vgraphpage{humor/humor-gru.png}
	\end{center}
	
\end{frame}


\section{Attention}

\begin{frame}
	\frametitle{\insertshortsubtitle}
	\framesubtitle{\insertsection: Motivation}
	
	\begin{center}
		\hgraphpage[0.8\textwidth]{seq2seq_exp.pdf}
	\end{center}

	\begin{itemize}
		\item Model: Seq2Seq (Encoder-Decoder)
		\item Long-term dependencies are not considered.
		\item The generation of target words depends only on previously generated words and the last words of the original text.
	\end{itemize}
	
\end{frame}

\subsection{Attention mechanism}

\begin{frame}
	\frametitle{\insertshortsubtitle: \insertsection}
	\framesubtitle{\insertsubsection}
	
	\begin{center}
		\hgraphpage[0.8\textwidth]{seq2seq_att_exp.pdf}
	\end{center}
	
	\vskip-6pt
	\begin{itemize}
		\item Keep $T$ hidden states from the encoder: $h_i$
		\item For each $s_{t-1}$, calculate its alignment with the hidden states.
		$ e_{t, i} = a(s_{t-1}, h_i) $
		\item Calculate the weights of the hidden states: $\alpha_{t, i} = \text{softmax}(e_{t, i})$
		\item Calculate the state vector $c_t = \sum_{i=1}^{T} \alpha_{t, i} h_i$
	\end{itemize}
	
\end{frame}

\begin{frame}
	\frametitle{\insertshortsubtitle: \insertsection}
	\framesubtitle{\insertsubsection: Formalism (1)}
	
	\begin{center}
		\hgraphpage[0.8\textwidth]{attention.pdf}
	\end{center}
	
	\vskip-6pt
	\begin{itemize}
		\item Keys ($k_i$): $m$ states which are used to estimate values' weights
		\item Values ($v_i$): $m$ states used to construct the context 
		\item Queries ($q_j$): current representation 
	\end{itemize}
	
\end{frame}

\begin{frame}
	\frametitle{\insertshortsubtitle: \insertsection}
	\framesubtitle{\insertsubsection: Formalism (2)}
	
	\begin{itemize}
		\item Let $ w_1 \cdots w_m $ a source sequence
		\begin{itemize}
			\item $ k_i =  Embedding_k(w_i)$ a vector of $ d $ elements
			\item $ v_i =  Embedding_v(w_i)$ a vector of $ v $ elements
		\end{itemize}
		\item Let $ w'_j $ a current target word
		\begin{itemize}
			\item $ q_j =  Embedding_q(w'_j)$ a vector of $ d $ elements
		\end{itemize}
		\item Calculate a similarity between each key $ k_i $ and our current target word 
		\[e_i = a(q_j, k_i)\]
		\item Calculate the percentage similarity (softmax can be masked)
		\[\alpha_i = softmax(e_i) = \frac{exp(e_i)}{\sum_k exp(e_k)}\] 
		\item The attention is a vector formed by cumulating percentages of each value vector
		\vspace{-6pt}\[attention(q, K, V) = \sum_{i=1}^{m} \alpha_i v_i\]
	\end{itemize}
	
\end{frame}

%\begin{frame}
%	\frametitle{\insertshortsubtitle: \insertsection}
%	\framesubtitle{\insertsubsection: Architecture}
%	
%	\begin{center}
%		\hgraphpage[\textwidth]{attention.pdf}
%	\end{center}
%	
%\end{frame}


\begin{frame}
	\frametitle{\insertshortsubtitle: \insertsection}
	\framesubtitle{\insertsubsection: Attention functions (non parametric)}
	
	\[Q \in \mathbb{R}^{n \times d}, \, K \in \mathbb{R}^{m \times d}, \, a(Q, K) \in \mathbb{R}^{n \times m}, V \in \mathbb{R}^{m \times v} \]
	
	\vspace{12pt}
	
	Dot product (\textbf{tf.keras.layers.Attention})
	\[a(Q, K) = Q \cdot K^\top\]
	
	Scaled dot product 
	\[a(Q, K) = \frac{Q \cdot K^\top}{\sqrt{d}}\]
	
	Cosine similarity
	\[a(Q, K) = \frac{Q \cdot K^\top}{||Q|| \, ||K||}\]
	
\end{frame}

\begin{frame}
	\frametitle{\insertshortsubtitle: \insertsection}
	\framesubtitle{\insertsubsection: Attention functions (parametric)}
	
	\[Q \in \mathbb{R}^{n \times d}, \, K \in \mathbb{R}^{m \times d}, \, a(Q, K) \in \mathbb{R}^{n \times m}, V \in \mathbb{R}^{m \times v} \]
	
	\vspace{12pt}
	
	Additive attention (\textbf{tf.keras.layers.AdditiveAttention})
	
	Bahdanau Attention
	
	\[a(Q, K) = W_v^\top \cdot \tanh(W_q Q + W_k K)\]
	
\end{frame}


\begin{frame}
	\frametitle{\insertshortsubtitle: \insertsection}
	\framesubtitle{\insertsubsection: Some humor}
	
	\begin{center}
		\vgraphpage{humor/humor-attention.jpeg}
	\end{center}
	
\end{frame}

\subsection{Multi-Head Attention}

\begin{frame}
	\frametitle{\insertshortsubtitle: \insertsection}
	\framesubtitle{\insertsubsection}

%	\vspace{-0.5cm}
	\begin{figure}
		\centering
		\hgraphpage[0.9\textwidth]{multi_head_att_.pdf}
		
		\vskip-12pt
		\caption{Multi-Head Attention \cite{2017-vaswani-al}}
	\end{figure}
	
\end{frame}

\begin{frame}
	\frametitle{\insertshortsubtitle: \insertsection}
	\framesubtitle{\insertsubsection: Formulas}
	
	\[Q \in \mathbb{R}^{n \times d}, \, K \in \mathbb{R}^{m \times d}, \, MultiHead \in \mathbb{R}^{n \times d_{model}}, V \in \mathbb{R}^{m \times v} \]
	
	\[W^Q_i \in \mathbb{R}^{d_{model} \times d}, \,  W^K_i \in \mathbb{R}^{d_{model} \times d}, \, W^V_i \in \mathbb{R}^{d_{model} \times v}, \, W^O \in \mathbb{R}^{hv \times d_{model}}\]
	
	\[head_i = Attention(Q W^Q_i, K W^Q_i, V W^V_i)\]
	
	\[MultiHead(Q, K, V) = Concat(head_1, \ldots, head_h) \cdot W^O\]
	
	
\end{frame}

\begin{frame}
	\frametitle{\insertshortsubtitle: \insertsection}
	\framesubtitle{\insertsubsection: Some humor}
	
	\begin{center}
		\vgraphpage{humor/humor-multihead.jpg}
	\end{center}
	
\end{frame}



\subsection{Self-attention}

\begin{frame}
	\frametitle{\insertshortsubtitle: \insertsection}
	\framesubtitle{\insertsubsection}
	
	\begin{itemize}
		\item \textbf{GOAL}: Encode an element of a sequence with respect to the entire sequence.
		\item \textbf{Example}: \expword{Encode a word with respect to the sentence to capture its semantics}.
		\item \textbf{Method}:
		\begin{itemize}
			\item Use MLPs to learn the query, key, and value for each element.
			\item Set the maximum number of elements in a sequence.
			\item Truncate sequences that exceed this number.
			\item Pad those that do not reach this number with a predefined code.
			\item Optionally use multi-head attention to learn multiple representations.
			\item Add positional embeddings to retain positional information in the sequence.
		\end{itemize}
	\end{itemize}
	
\end{frame}

\begin{frame}
	\frametitle{\insertshortsubtitle: \insertsection}
	\framesubtitle{\insertsubsection: Example}
	
	\begin{center}
		\vgraphpage{self-attention_exp.pdf}
	\end{center}
	
\end{frame}

\begin{frame}
	\frametitle{\insertshortsubtitle: \insertsection}
	\framesubtitle{\insertsubsection: Some humor}
	
%	\begin{center}
		\hgraphpage{humor/humor-self-attention.jpg}
%	\end{center}
	
\end{frame}


\subsection{Transformer}

\begin{frame}
	\frametitle{\insertshortsubtitle: \insertsection}
	\framesubtitle{\insertsubsection}

%	\vspace{-4pt}
	\begin{minipage}{0.49\textwidth}
		\begin{figure}
			\centering
			\hgraphpage[0.68\textwidth]{transformers_.pdf}
			\vskip-8pt
			\caption{Transformers' architecture \cite{2017-vaswani-al}}
		\end{figure}
	\end{minipage}
	\begin{minipage}{0.1\textwidth}
		=
	\end{minipage}
	\begin{minipage}{0.39\textwidth}
			\hgraphpage[\textwidth]{transformers.pdf}
	\end{minipage}

	
\end{frame}

\begin{frame}
	\frametitle{\insertshortsubtitle: \insertsection}
	\framesubtitle{\insertsubsection: Encoder}
	
	\begin{itemize}
		\item Encodes each input element based on its context (the entire input).
		\item Uses multi-head self-attention
		\item Completely bi-directional
		\begin{itemize}
			\item In BiRNN architecture, the encoder represents long-term relation indirectly (using a context).
			\item In Encoders it is directly represented, but the notion of element's position is lost.
		\end{itemize} 
		\item We can introduce position's embedding into the word's initial representation (before being transformed into keys, values and queries).
	\end{itemize}
	
\end{frame}

\begin{frame}
	\frametitle{\insertshortsubtitle: \insertsection}
	\framesubtitle{\insertsubsection: Decoder}
	
	\begin{itemize}
		\item \keyword{Autoregressive}: The output is a \optword{direct} function of the past elements
		\[\hat{y}_t = \sum_{i = 1}^{\rho} \varphi_{i} y_{t-i} + \epsilon_t\]
		\item Not as RNNs where the dependence is indirect
		\[\hat{y}_t = \arg\max_y p(y | X;\ y_1, \cdots, y_{i-1};\ W;\ b)\]
		\item Each time an element is generated, it will be added into the input to generate the next element.
		\item To stop generating, a maximum size is used. Also, a special token is used to mark the end of generation.
	\end{itemize}
	
\end{frame}

\begin{frame}
	\frametitle{\insertshortsubtitle: \insertsection}
	\framesubtitle{\insertsubsection: Pretrained models (BERT: Enc) \cite{devlin-etal-2019-bert}}

	\begin{center}
		\hgraphpage[0.95\textwidth]{bert-arch.pdf}
	\end{center}
	
\end{frame}

\begin{frame}
	\frametitle{\insertshortsubtitle: \insertsection}
	\framesubtitle{\insertsubsection: Pretrained models (ViT: Enc) \cite{dosovitskiy2021an}}
	
%	\begin{itemize}
%		\item Encoder chaque partie de l'image (\keyword{patch}) par rapport à d'autres.
%		\item Voir \url{https://arxiv.org/pdf/2010.11929.pdf}
%	\end{itemize}
	
	\vspace{-6pt}
	\begin{figure}[htp!]
		\centering
		\hgraphpage[0.62\textwidth]{vit.pdf}
		\vskip-8pt
		\caption{Example of images encoder [modified from \cite{zhang2021dive}]}
	\end{figure}
	
\end{frame}


\begin{frame}
	\frametitle{\insertshortsubtitle: \insertsection}
	\framesubtitle{\insertsubsection: Pretrained models (GPT: Dec) \cite{radford2018improving}}
	
	\begin{figure}[htbp]
		\hgraphpage[\textwidth]{gpt-arch_.pdf}
		\caption{GPT's architecture and different tasks \cite{radford2018improving}}
	\end{figure}
	
\end{frame}


\begin{frame}
	\frametitle{\insertshortsubtitle: \insertsection}
	\framesubtitle{\insertsubsection: Pretrained models (T5: Enc-Dec) \cite{T5}}
	
	\begin{figure}[htbp]
		\hgraphpage[\textwidth]{t5-arch_.pdf}
		\caption{T5 (Text-to-Text Transfer Transformer) training \cite{T5}}
	\end{figure}
	
\end{frame}

\begin{frame}
	\frametitle{\insertshortsubtitle: \insertsection}
	\framesubtitle{\insertsubsection: Pretrained models (BART: Enc-Dec) \cite{bart}}
	
	\begin{figure}[htbp]
		\centering
		\hgraphpage[0.58\textwidth]{bart-arch1_.pdf}
		\hgraphpage[0.4\textwidth]{bart-arch2_.pdf}
		\caption{BART training \cite{bart}}
	\end{figure}
	
\end{frame}


\begin{frame}
	\frametitle{\insertshortsubtitle: \insertsection}
	\framesubtitle{\insertsubsection: Some humor}
	
	\begin{center}
		\vgraphpage{humor/humor-transformer.png}
	\end{center}
	
\end{frame}


\insertbibliography{ML_Lect-NN}{*}

\begin{frame}[plain]
	\begin{center}
		\hgraphpage[.9\textwidth]{humor/humor-nn.jpg}
	\end{center}
\end{frame}

\end{document}


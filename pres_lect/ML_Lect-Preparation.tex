% !TEX TS-program = pdflatex
% !TeX program = pdflatex
% !TEX encoding = UTF-8
% !TeX spellcheck = en_US

\documentclass[xcolor=table]{beamer}

\usepackage{../extra/beamer/karimml}

\def\accyear{2024-2025}
\institute{
	Laboratoire de la Communication dans les Systèmes Informatiques (LCSI)\\
	École  nationale Supérieure d'Informatique (ESI, ex. INI), Algiers, Algeria
}

\author[Abdelkrime Aries (\accyear)] %
{Abdelkrime Aries}

\titlegraphic{%
	\includegraphics[height=1cm]{../extra/logo/esi-logo.png}%
	\hspace*{2cm}%
	\includegraphics[height=1cm]{../extra/logo/lcsi-logo.png}%
	\hspace*{2cm}%
	\includegraphics[height=1cm]{../extra/logo/esi.ml.pdf}%
}%\hspace*{4.75cm}~

\date{Academic year: \accyear} %\today

\title[ML]{Machine Learning} 

\usefolder{../extra/beamer}

\usetheme{Karimml} % Antibes Boadilla Warsaw

\beamertemplatenavigationsymbolsempty

\usepackage{longfbox}

\newfboxstyle{kared}{padding=1pt,margin=1pt,border-radius=1ex,border-color=red}
\newfboxstyle{kgboxx}{padding=1pt,margin=1pt,border-radius=1ex,border-color=green}


\subtitle[Data \& eval]{Data preparation and models' evaluation} 

\changegraphpath{../img/preparation/}

\begin{document}
	
\begin{frame}
	\frametitle{\inserttitle}
	\framesubtitle{\insertshortsubtitle: Introduction}
	
	\begin{figure}
		\centering
		\hgraphpage[.4\textwidth]{crisp-dm.png}
		\caption{CRISP-DM (Cross-industry standard process for data mining)}
	\end{figure}
	
\end{frame}


\begin{frame}
	\frametitle{\inserttitle}
	\framesubtitle{\insertshortsubtitle: Plan}
	
	\begin{multicols}{2}
		%	\small
		\tableofcontents
	\end{multicols}
\end{frame}

\section{Data collection}

\begin{frame}
	\frametitle{\insertshortsubtitle}
	\framesubtitle{\insertsection}
	
	
\end{frame}

\subsection{Data quality}

\begin{frame}
	\frametitle{\insertshortsubtitle: \insertsection}
	\framesubtitle{\insertsubsection}
	
	\begin{itemize}
		\item Size: Number of samples.
		\item Number and type of characteristics (nominal, binary, ordinal or continuous).
		\item Annotation errors.
		\item Amount of noise in the data: errors and exceptions.
	\end{itemize}

\end{frame}


\subsection{Data integration}

\begin{frame}
	\frametitle{\insertshortsubtitle: \insertsection}
	\framesubtitle{\insertsubsection}
	
	\begin{itemize}
		\item \optword{Data sources}
		\begin{itemize}
			\item Structured data: \expword{DB, tabulators (CSV, etc.)}
			\item Semi-structured data: \expword{XML, JSON, etc.}
			\item Unstructured data: \expword{text documents, images, metadata, etc.}
		\end{itemize}
	
		\item \optword{Data integrity}
		\begin{itemize}
			\item Conformance of XML files to their DTD definitions
			\item Correct separators for CSV files
		\end{itemize}
	
		\item \optword{Data merge}
		\begin{itemize}
			\item Naming problem: \expword{bd1.nbrclient, bd2.clientid}
			\item Value conflicts: \expword{bd1.height(cm), bd2.height(inches)}
			\item Redundancy: calculated attributes and identical records.
			\item Different types of attributes: \expword{bd1.temperature (cold, hot, ...), bd2.temperature (15, 23, ...)}
		\end{itemize}
	\end{itemize}

\end{frame}


\subsection{Data annotation}

\begin{frame}
	\frametitle{\insertshortsubtitle: \insertsection}
	\framesubtitle{\insertsubsection: Internal}
	
	\begin{itemize}
		\item A team designated for annotation
		\item Better if there are sufficient human, financial and time resources
		\item \textbf{Advantages}
		\begin{itemize}
			\item Ability to track progress
			\item Good quality
		\end{itemize}
		\item \textbf{Disadvantages}
		\begin{itemize}
			\item Too slow: if we gain quality, we will lose time.
		\end{itemize}
	\end{itemize}

\end{frame}

\begin{frame}
	\frametitle{\insertshortsubtitle: \insertsection}
	\framesubtitle{\insertsubsection: Outsourcing}
	
	\begin{itemize}
		\item Hire independent annotators (freelancers)
		\item If there are not enough human resources or time
		\item \textbf{Steps}
		\begin{enumerate}
			\item Prepare the data and set the required time to annotate it
			\item Divide data into subsets
			\item Post job offers on social media
		\end{enumerate}
		\item \textbf{Advantages}
		\begin{itemize}
			\item We know who we hired: We can check their skills
		\end{itemize}
		\item \textbf{Disadvantages}
		\begin{itemize}
			\item Preparing detailed annotation instructions
			\item Task submission and verification time
			\item Create a workflow: an interface that helps annotators.
		\end{itemize}
	\end{itemize}

\end{frame}

\begin{frame}
	\frametitle{\insertshortsubtitle: \insertsection}
	\framesubtitle{\insertsubsection: Crowdsourcing}
	
	\begin{columns}
		\begin{column}{0.78\textwidth}
			\begin{itemize}
				\item Use crowdsourcing platforms.
				\item Ex. \expword{Amazon Mechanical Turk (MTurk) and Clickworker}.
				\item \textbf{Types of crowdsourcing}
				\begin{enumerate}
					\item \optword{Explicit}: By directly requesting contributions
					\item \optword{Implicit}: By integrating tasks into other forms
					\begin{itemize}
						\item Unavoidable tasks (Ex. \expword{reCAPTCHA})
						\item Games with objectives (Ex. \expword{ESP game})
					\end{itemize}
				\end{enumerate}
			\end{itemize}
		\end{column}
		\begin{column}{0.22\textwidth}
			\hgraphpage{captcha.png}
		\end{column}
	\end{columns}
	
	\begin{itemize}
		\item \textbf{Advantages}
		\begin{itemize}
			\item Fast results
			\item Affordable costs
		\end{itemize}
		\item \textbf{Disadvantages}
		\begin{itemize}
			\item Quality of annotations
			\item Preparing detailed instructions on the annotation process
		\end{itemize}
	\end{itemize}

\end{frame}

\begin{frame}
	\frametitle{\insertshortsubtitle: \insertsection}
	\framesubtitle{\insertsubsection: Evaluation (Kappa; classification)}
	
	\begin{itemize}
		\item Inter-rater agreement (inter-rater reliability)
	\end{itemize}

\begin{columns}
\begin{column}{0.6\textwidth}
	\begin{itemize}
%		\item Concordance inter-juges (fiabilité inter-évaluateurs) 
		\item \optword{Cohen's Kappa}
		\begin{itemize}
			\item Measures agreement between two raters and two categories
			\item $K = \frac{P_o - P_e}{1 - P_e}$
			\item $P_o = \frac{a + d}{a+b+c+d}$
			\item $P_{Yes} = \frac{a+b}{a+b+c+d} * \frac{a+c}{a+b+c+d}$
			\item $P_{No} = \frac{c+d}{a+b+c+d} * \frac{b+d}{a+b+c+d}$
			\item $P_e = P_{Yes} + P_{No}$
			\item \expword{sklearn.metrics.cohen\_kappa\_score}
		\end{itemize}
	\end{itemize}
\end{column}
\begin{column}{0.4\textwidth}
	\begin{tabular}{|c|c|c|c|}
		\cline{3-4}
		\multicolumn{2}{c|}{}& \multicolumn{2}{c|}{B} \\
		\cline{3-4}
		\multicolumn{2}{c|}{}& Yes & No \\
		\hline
		\multirow{2}{*}{A} & Yes & a & b \\
		\cline{2-4}
		& No & c & d \\
		\hline
	\end{tabular}
\end{column}
\end{columns}

\begin{itemize}
	\item \optword{Scott's Pi}
	\begin{itemize}
		\item Like Cohen's but $P_e$ is calculated differently (multi-classes)
	\end{itemize}
	\item \optword{Fleiss' Kappa}
	\begin{itemize}
		\item Generalization of Scott's Pi for more than two raters
	\end{itemize}
\end{itemize}

\end{frame}

\begin{frame}
	\frametitle{\insertshortsubtitle: \insertsection}
	\framesubtitle{\insertsubsection: Evaluation (Correlation; regression)}
	
	\begin{itemize}
		\item \optword{Pearson}
		\begin{itemize}
			\item Continuous data (regression)
			\item $r_{Y^{(1)},Y^{(2)}} = \frac{covariance(Y^{(1)},Y^{(2)})}{\sigma_{Y^{(1)}} \sigma_{Y^{(2)}}}$
			\item \expword{scipy.stats.pearsonr}
		\end{itemize}
		\item \optword{Spearman}
		\begin{itemize}
			\item Ordinal data
			\item $\rho_{Y^{(1)},Y^{(2)}} = \frac{covariance(R(Y^{(1)}),R(Y^{(2)}))}{\sigma_{R(Y^{(1)})} \sigma_{R(Y^{(2)})}}$
			\item $R$ is the rank in ascending order
			\item \expword{scipy.stats.spearmanr}
		\end{itemize}
		\item \optword{Kendall}
		\begin{itemize}
			\item Ordinal data
			\item $\tau_{Y^{(1)},Y^{(2)}} = \frac{2}{n (n-1)} \sum_{i<j} sign(y^{(1)}_i - y^{(1)}_j) sign(y^{(2)}_i - y^{(2)}_j)$
			\item \expword{scipy.stats.kendalltau}
		\end{itemize}
	\end{itemize}
	
\end{frame}

\subsection{Data cleaning}

\begin{frame}
	\frametitle{\insertshortsubtitle: \insertsection}
	\framesubtitle{\insertsubsection: Problems}
	
	\begin{itemize}
		\item \optword{Omitted values (Unavailable data)}
		\begin{itemize}
			\item Equipment malfunction
			\item Inconsistent with other data and therefore removed
			\item Not entered because not (or poorly) understood
			\item Considered unimportant at the time of entry
		\end{itemize}
		\item \optword{Duplicate samples}
		\begin{itemize}
			\item Multiple data sources
		\end{itemize}
		\item \optword{Bad annotations}
		\begin{itemize}
			\item Inconsistency in naming conventions
		\end{itemize}
		\item \optword{Noise (error or random variance of a measured variable)}
		\begin{itemize}
			\item Defective measuring instrument
			\item Input problem
			\item Transmission problem
		\end{itemize}
	\end{itemize}
	
\end{frame}

\begin{frame}
	\frametitle{\insertshortsubtitle: \insertsection}
	\framesubtitle{\insertsubsection: Possible solutions}
	
	\begin{itemize}
		\item \optword{Omitted values (Unavailable data)}
		\begin{itemize}
			\item Deletion or manual entry
			\item Replacement by a global constant. Ex., \expword{``unknown" or ``0"}.
			\item Replacement by the average of numeric values, preference: same class.
			\item Replacement with the most frequent nominal value.
		\end{itemize}
		\item \optword{Duplicate samples}
		\begin{itemize}
			\item Deletion
		\end{itemize}
		\item \optword{Bad annotations}
		\begin{itemize}
			\item Measuring inter-rater reliability
		\end{itemize}
		\item \optword{Noise}
		\begin{itemize}
			\item Clustering to detect exceptions
			\item Automatic detection of suspicious values and human verification.
			\item Smoothing of data using regression methods.
		\end{itemize}
	\end{itemize}
	
\end{frame}

\begin{frame}
	\frametitle{\insertshortsubtitle: \insertsection}
	\framesubtitle{\insertsubsection: Some humor}
	
	\begin{center}
		\vgraphpage{humour-collection.jpg}
	\end{center}
	
\end{frame}

\section{Data transformation}

\begin{frame}
	\frametitle{\insertshortsubtitle}
	\framesubtitle{\insertsection}
	
	
\end{frame}

\subsection{Numerical features}

\begin{frame}
	\frametitle{\insertshortsubtitle: \insertsection}
	\framesubtitle{\insertsubsection: Binning (bucketing)}
	
	\begin{columns}
		\begin{column}{0.5\textwidth}
			\hgraphpage{bucketing-1.png}
			
			\vspace{-6pt}
			\begin{center}
				quartiles
			\end{center}
		\end{column}
		\begin{column}{0.5\textwidth}
			\hgraphpage{bucketing-2.png}
			
			\vspace{-6pt}
			\begin{center}
				identical ranges
			\end{center}
		\end{column}
	\end{columns}

	\begin{center}
		Example of binning \cite{2021-google-prep}
	\end{center}

	\begin{itemize}
		\item when there is no linear relationship with the output
		\item \expword{sklearn.preprocessing.KBinsDiscretizer}
	\end{itemize}
	
\end{frame}

\begin{frame}
	\frametitle{\insertshortsubtitle: \insertsection}
	\framesubtitle{\insertsubsection: Normalization}
	
	\hgraphpage{norm.png}
	
	\begin{center}
		Comparison between normalization functions \cite{2021-google-prep}
	\end{center}
	
\end{frame}

\begin{frame}
	\frametitle{\insertshortsubtitle: \insertsection}
	\framesubtitle{\insertsubsection: Normalization (Clipping)}
	
	\begin{columns}
	\begin{column}{.5\textwidth}
	\[x' = \begin{cases}
	\alpha & \text{if } x \ge \alpha \\
	\beta & \text{if } x \le \beta \\
	x & \text{otherwise}
	\end{cases}
	\]
	
	\begin{itemize}
		\item This normalization is used when
		\begin{itemize}
			\item there is some extremely outliers.
		\end{itemize}
		\item \textbf{Example}
		\begin{itemize}
			\item \expword{Number of rooms per person}
		\end{itemize}
		\item \expword{numpy.clip}
	\end{itemize}
	\end{column}
	\begin{column}{.5\textwidth}
		\hgraphpage{norm-coupure.png}
		
		\begin{center}
			Example of clipping \cite{2021-google-prep}
		\end{center}
	\end{column}
 \end{columns}
	
\end{frame}

\begin{frame}
	\frametitle{\insertshortsubtitle: \insertsection}
	\framesubtitle{\insertsubsection: Normalization (Min-max scaling)}
	
	\[x' = \frac{x - x_{min}}{x_{max} - x_{min}}\]
	
	\begin{itemize}
		\item This normalization is used when
		\begin{itemize}
			\item we know the lower and upper limits;
			\item the values are almost evenly distributed over this range
		\end{itemize}
		\item \textbf{Example}
		\begin{itemize}
			\item \textbf{Good}: \expword{ages}
			\item \textbf{Bad}: \expword{income (few people with high income)}
		\end{itemize}
		\item \expword{sklearn.preprocessing.MinMaxScaler}
	\end{itemize}
	
\end{frame}

\begin{frame}
	\frametitle{\insertshortsubtitle: \insertsection}
	\framesubtitle{\insertsubsection: Normalization (Log scaling)}
	
	\begin{columns}
		\begin{column}{.5\textwidth}
			
	\[x' = \log(x)	\]
	
	\begin{itemize}
		\item This normalization is used when
		\begin{itemize}
			\item the feature follows power law
		\end{itemize}
		\item \textbf{Example}
		\begin{itemize}
			\item \expword{Number of rooms per person}
		\end{itemize}
		\item \expword{numpy.log}
	\end{itemize}
	\end{column}
	\begin{column}{.5\textwidth}
		\hgraphpage{norm-log.png}
		
		\begin{center}
			Example of Log scaling \cite{2021-google-prep}
		\end{center}
	\end{column}
\end{columns}

\end{frame}

\begin{frame}
	\frametitle{\insertshortsubtitle: \insertsection}
	\framesubtitle{\insertsubsection: Normalization (Z-score; standardization)}
	
	\begin{columns}
	\begin{column}{.5\textwidth}
	\[x' = \frac{x - \mu}{\sigma}	\]
	
	\begin{itemize}
		\item This normalization is used when
		\begin{itemize}
			\item there are few outliers.
			\item we want to have a mean of 0 and standard deviation of 1
		\end{itemize}
		\item Example
		\begin{itemize}
			\item \expword{Number of rooms per person}
		\end{itemize}
		\item \expword{sklearn.preprocessing.StandardScaler}
	\end{itemize}
	\end{column}
	\begin{column}{.5\textwidth}
		\hgraphpage{norm-z.png}
		
		\begin{center}
			Example of standardization \cite{2021-google-prep}
		\end{center}
	\end{column}
\end{columns}
	
\end{frame}

\begin{frame}
	\frametitle{\insertshortsubtitle: \insertsection}
	\framesubtitle{\insertsubsection: Features generation}
	
%	\begin{columns}
%		\begin{column}{.5\textwidth}
			
			\begin{itemize}
				\item Features generation is used when
				\begin{itemize}
					\item we want more complex features
				\end{itemize}
				\item \optword{Polynomial generation} (interactions)
				\begin{itemize}
					\item $(X_1, X_2, \ldots) \longrightarrow (1, X_1, X_2, X_1 X_2, X_1^2, X_2^2, \ldots)$
					\item \expword{sklearn.preprocessing.PolynomialFeatures}
				\end{itemize}
				\item \optword{Sparse Auto-encoders}
				\begin{itemize}
					\item A neural network trained using unsupervised learning
					\item It aims to represent the samples with more characteristics (a vector representation with more dimensionality)
				\end{itemize}
			\end{itemize}
%		\end{column}
%		\begin{column}{.5\textwidth}
%			\hgraphpage{norm-z.png}
%		\end{column}
%	\end{columns}
	
\end{frame}

\subsection{Categorical features}

\begin{frame}
	\frametitle{\insertshortsubtitle: \insertsection}
	\framesubtitle{\insertsubsection: Encoding}
	
	\begin{itemize}
		\item \optword{Ordinal encoding}
		\begin{itemize}
			\item the API used for training does not accept categories
			\item the learning algorithm does not consider the feature's categories as ordered: Multinomial Naive Bayes
			\item or if the categories order is important
			\item \expword{sklearn.preprocessing.OrdinalEncoder}
		\end{itemize}
	
		\item \optword{One-Hot encoding}
		\begin{itemize}
			\item when we want to give the same chance to different categories
			\item Each feature's category will be encoded as a separate column (binary)
			\item \expword{sklearn.preprocessing.OneHotEncoder}
		\end{itemize}

	\end{itemize}
	
\end{frame}

\begin{frame}
	\frametitle{\insertshortsubtitle: \insertsection}
	\framesubtitle{\insertsubsection: Some humor}
	
	\begin{center}
		\hgraphpage[0.5\textwidth]{humour-transformation.jpeg}
	\end{center}
	
\end{frame}

\section{Data sampling and splitting}

\begin{frame}
	\frametitle{\insertshortsubtitle}
	\framesubtitle{\insertsection}
	
	
\end{frame}

\subsection{Unbalanced data}

\begin{frame}
	\frametitle{\insertshortsubtitle: \insertsection}
	\framesubtitle{\insertsubsection}
	
	\begin{table}
	\SetTblrInner{rowsep=0pt,colsep=1pt}
	\begin{tblr}{
			colspec = {p{0.3\textwidth}p{0.5\textwidth}},
			row{odd} = {lightblue, font=\small},
			row{even} = {lightyellow, font=\small},
			row{1} = {darkblue, font=\bfseries},
		}
			\textcolor{white}{Imbalance degree} & \textcolor{white}{Minority class proportion} \\
			mild & 20-40\% of data \\
			moderate & 1-20\% of data \\
			extreme &	\textless 1\% of data \\
	\end{tblr}
	\caption{Imbalance degree \cite{2021-google-prep}}
	\end{table}

	\begin{itemize}
		\item \url{https://github.com/scikit-learn-contrib/imbalanced-learn}
		\item \textbf{pip install -U imbalanced-learn}
		\item \textbf{conda install -c conda-forge imbalanced-learn}
	\end{itemize}
	
\end{frame}

\begin{frame}
	\frametitle{\insertshortsubtitle: \insertsection}
	\framesubtitle{\insertsubsection: Under sampling}
	
	\begin{itemize}
		\item Remove samples from majority class (downsampling)
		\item Calibrate the model (upweithing)
		\item \optword{Random deletion}
		\begin{itemize}
			\item \expword{imblearn.under\_sampling.RandomUnderSampler}
		\end{itemize}
		\item \optword{Cluster centroids}
		\begin{itemize}
			\item Apply \keyword{K-Means} on the majority class (with K being the number of desired samples)
			\item Take the center of each cluster as representative
			\item \expword{imblearn.under\_sampling.ClusterCentroids}
		\end{itemize}
		\item \optword{Tomek links}
		\begin{itemize}
			\item Detect the closest majority class points to the minority class points
			\item Delete these points
			\item \expword{imblearn.under\_sampling.TomekLinks}
		\end{itemize}
		\item \keyword{I'm too lazy to present more methods! :)}
	\end{itemize}
	
\end{frame}

\begin{frame}
	\frametitle{\insertshortsubtitle: \insertsection}
	\framesubtitle{\insertsubsection: Over sampling}
	
	\begin{itemize}
		\item Add samples to the minority class (oversampling)
		\item \optword{Random duplication}
		\begin{itemize}
			\item \expword{imblearn.over\_sampling.RandomOverSampler}
		\end{itemize}
		\item \optword{SMOTE (Synthetic Minority Over-sampling Technique)}
		\begin{itemize}
			\item Find the K nearest neighbors of each point of the minority class
			\item Define a new point between this point and one of its neighbors
			\item \expword{imblearn.over\_sampling.SMOTE}
		\end{itemize}
		\item \optword{ADASYN (Adaptive Synthetic Sampling)}
		\begin{itemize}
			\item Like \keyword{SMOTE}
			\item The method favors points in non-homogeneous spaces
			\item \expword{imblearn.over\_sampling.ADASYN}
		\end{itemize}
		\item \keyword{Still lazy}
	\end{itemize}
	
\end{frame}

\subsection{Data splitting}

\begin{frame}
	\frametitle{\insertshortsubtitle: \insertsection}
	\framesubtitle{\insertsubsection}
	
	\begin{itemize}
		\item Data splitting
		\begin{itemize}
			\item \optword{Training}: a majority of samples (70-80\%)
			\item \optword{Test}: a minority of samples (30-20\%)
			\item \optword{Validation}: a minority of samples to adjust hyper-parameters (20\% from training)
		\end{itemize}
		\item Split condition
		\begin{itemize}
			\item The test data is sufficient to have meaningful results.
			\item Test data is representative. You should not take a set with different features from those of the training data.
			\item If the problem is a prediction of the future based on the past, the dataset must be divided such that the test data follows the training one
			\item To avoid losing training data, use cross-validation
		\end{itemize}
	\end{itemize}
	
\end{frame}

\begin{frame}
	\frametitle{\insertshortsubtitle: \insertsection}
	\framesubtitle{\insertsubsection: Data and training}
	
	\begin{itemize}
		\item \optword{Underfitting}
		\begin{itemize}
			\item The model took less time to generalize on the training data
			\item Few training data
			\item Training data is not representative
		\end{itemize}

		\item \optword{Overfitting}
		\begin{itemize}
			\item The model was exactly adapted to the training data
			\item There is noise in the training data
			\item Test data is not representative.
		\end{itemize}

	\end{itemize}
	
\end{frame}

\begin{frame}
	\frametitle{\insertshortsubtitle: \insertsection}
	\framesubtitle{\insertsubsection: Stratified sampling}
	
	\begin{itemize}
		\item Random data splitting
		\begin{itemize}
			\item Test data may not contain classes
			\item Test data may not be proportional to the training one
		\end{itemize}
		\item Steps
		\begin{itemize}
			\item Separate data according to their classes
			\item Take random samples from each class to form the test dataset
			\item We can calculate the original proportions of each class in the data and extract samples according to its proportions
		\end{itemize}
	\end{itemize}
	
\end{frame}

\begin{frame}
	\frametitle{\insertshortsubtitle: \insertsection}
	\framesubtitle{\insertsubsection: Cross validation}
	
	\begin{figure}
		\centering
		\hgraphpage[.5\textwidth]{grid_search_workflow.png}
		\caption{Machine learning workflow with cross-validation \cite{2020-sklearn-man}}
	\end{figure}
	
\end{frame}


\begin{frame}
	\frametitle{\insertshortsubtitle: \insertsection}
	\framesubtitle{\insertsubsection: Cross validation (K-Folds)}
	
	\begin{figure}
		\centering
		\hgraphpage[.6\textwidth]{grid_search_cross_validation.png}
		\caption{Illustration of K-Folds cross-validation \cite{2020-sklearn-man}}
	\end{figure}
	
\end{frame}

\begin{frame}
	\frametitle{\insertshortsubtitle: \insertsection}
	\framesubtitle{\insertsubsection: Some humor}
	
		\begin{center}
			\vgraphpage{humour-fractionnement.jpeg}
			\vgraphpage{humour-traintest.jpg}
		\end{center}
	
\end{frame}

\section{Models' evaluation}

\begin{frame}
	\frametitle{\insertshortsubtitle}
	\framesubtitle{\insertsection}
	
	
\end{frame}

\subsection{Classification}

\begin{frame}
	\frametitle{\insertshortsubtitle: \insertsection}
	\framesubtitle{\insertsubsection: Confusion matrix}
	
	\begin{tabular}{|c|c|c|c|}
		\cline{3-4}
		\multicolumn{2}{c|}{}& \multicolumn{2}{c|}{Classes réelles} \\
		\cline{3-4}
		\multicolumn{2}{c|}{}& Positive & Negative \\
		\hline
		\multirow{2}{*}{Prediction} & Positive & True Positive (TP) & False Positive (FP) \\
		\cline{2-4}
		& Negative & False Negative (FN) & True Negative (TN) \\
		\hline
	\end{tabular}

	\begin{itemize}
		\item \optword{True positive}: The model correctly predicts the positive class.
		\item \optword{True negative}: The model correctly predicts the negative class.
		\item \optword{False positive}: The model incorrectly predicts the positive class.
		\item \optword{False negative}: The model incorrectly predicts the negative class.
	\end{itemize}
	
\end{frame}

\begin{frame}
	\frametitle{\insertshortsubtitle: \insertsection}
	\framesubtitle{\insertsubsection: Accuracy}
	
	\[Accuracy = \frac{TP + TN}{TP + FP + FN + TN}\]
	
	\begin{tabular}{|c|c|c|c|}
		\cline{3-4}
		\multicolumn{2}{c|}{}& \multicolumn{2}{c|}{Real} \\
		\cline{3-4}
		\multicolumn{2}{c|}{}& Spam & Not-Spam \\
		\hline
		\multirow{2}{*}{Prediction} & Spam & 0 & 1 \\
		\cline{2-4}
		& Not-Spam & 3 & 16 \\
		\hline
	\end{tabular}
	
	\begin{itemize}
		\item $Accuracy = \frac{0 + 16}{1 + 1 + 3 + 16} = \frac{16}{20} = 80\%$
		\item Accuracy is not a good metric if
		\begin{itemize}
			\item classes are unbalanced
			\item we want to test the model's performance only on the positive class
		\end{itemize}
		\item \expword{sklearn.metrics.accuracy\_score}
	\end{itemize}
	
\end{frame}

\begin{frame}
	\frametitle{\insertshortsubtitle: \insertsection}
	\framesubtitle{\insertsubsection: Recall, Precision and F1 score}
	
	\vspace{-6pt}
	\[R = \frac{TP}{TP + FN} = \frac{\text{correctly predicted positives}}{\text{real positives}}\]
	
	\begin{itemize}
		\item The classifier's ability to find all positive samples
		\item \expword{sklearn.metrics.recall\_score}
	\end{itemize}
	
	\vspace{-6pt}
	\[P = \frac{TP}{TP + FP} = \frac{\text{correctly predicted positives}}{\text{predicted positives}}\]
	
	\begin{itemize}
		\item The classifier's ability to not label negative samples as positive ones 
		\item \expword{sklearn.metrics.precision\_score}
	\end{itemize}
	
	\vspace{-6pt}
	\[F1 = \frac{2 P R}{P + R}\]
	
	\begin{itemize}
		\item Harmonic mean between P and R
		\item \expword{sklearn.metrics.f1\_score}
	\end{itemize}
	
\end{frame}

\begin{frame}
	\frametitle{\insertshortsubtitle: \insertsection}
	\framesubtitle{\insertsubsection: R, P et F1 (multi-classes)}

	\begin{itemize}
		\item $\hat{y}$: the set of predictions, $y$: the set of correct labels
		\item $L$: all labels (classes)
		\item $S$: all samples
		\item $M$: the metric which can be $R$, $P$ or $F1$
	\end{itemize}

	\begin{center}
		\SetTblrInner{rowsep=0pt,colsep=1pt}
		\begin{tblr}{
				colspec = {p{0.3\textwidth}p{0.5\textwidth}},
				row{odd} = {lightblue, font=\small},
				row{even} = {lightyellow, font=\small},
				row{1} = {darkblue, font=\bfseries},
			}
			\textcolor{white}{Mean} & \textcolor{white}{Calculation formula} \\
			micro (Precision) & \vspace{-6pt}\[\frac{\sum_{l \in L} TP_l}{\sum_{l \in L} (TP_l + FP_l)}\]\vspace{-6pt} \\
			macro & \vspace{-6pt}\[\frac{1}{|L|} \sum_{l \in L} M(y_l, \hat{y}_l)\]\vspace{-6pt}\\
			weighted & \vspace{-6pt}\[\frac{1}{\sum_{l \in L} |\hat{y}_l|} \sum_{l \in L} |\hat{y}_l| M(y_l, \hat{y}_l)\]\vspace{-6pt} \\
		\end{tblr}
	\end{center}
	
\end{frame}

\begin{frame}
	\frametitle{\insertshortsubtitle: \insertsection}
	\framesubtitle{\insertsubsection: Matthews correlation coefficient}
	
	\[CCM = \frac{TP \times TN - FP \times FN}{\sqrt{(TP + FP)(TP + FN)(TN + FP)(TN + FN)}}\]
	\begin{itemize}
		\item The coefficient can be between \textbf{-1} and \textbf{+1}
		\begin{itemize}
			\item -1: the predictions are completely wrong.
			\item 0: the performance of the model is comparable with a random system.
			\item +1: the predictions are perfect.
		\end{itemize}
		\item \expword{sklearn.metrics.matthews\_corrcoef}
	\end{itemize}
		
\end{frame}

\subsection{Regression}

\begin{frame}
	\frametitle{\insertshortsubtitle: \insertsection}
	\framesubtitle{\insertsubsection: Mean Squared Error}
	
	\[MSE(\hat{y}, y) = \mathbb{E}(y - \hat{y})^2 =  \frac{1}{|S|} \sum_{i=1}^{|S|} (y_i - \hat{y}_i)^2\]

	\begin{itemize}
		\item Punishes the model when there are big mistakes (magnifies big mistakes)
		\item Actually, it calculates the squared error
		\item \expword{sklearn.metrics.mean\_squared\_error}
		\item To get the error, we can apply a square root
		\item $RMSE(\hat{y}, y) = \sqrt{MSE(\hat{y}, y)}$
		\item \optword{Root Mean Squared Error}
	\end{itemize}
	
\end{frame}

\begin{frame}
	\frametitle{\insertshortsubtitle: \insertsection}
	\framesubtitle{\insertsubsection: Mean Absolute Error}
	
	\[MAE(\hat{y}, y) = \mathbb{E}|y - \hat{y}| =  \frac{1}{|S|} \sum_{i=1}^{|S|} |y_i - \hat{y}_i|\]
	
	\begin{itemize}
		\item Attributes the same magnitude to small errors and large errors 
		\item \expword{sklearn.metrics.mean\_absolute\_error}
	\end{itemize}
	
\end{frame}

\subsection{Clustering}

\begin{frame}
	\frametitle{\insertshortsubtitle: \insertsection}
	\framesubtitle{\insertsubsection: Rand index}
	
	\[RI(\hat{y}, y) = \frac{|\hat{y} \bigcap y|}{|y|}\]
	
	\begin{itemize}
		\item Test dataset must be annotated
		\item \expword{sklearn.metrics.rand\_score}
	\end{itemize}
	
\end{frame}

\begin{frame}
	\frametitle{\insertshortsubtitle: \insertsection}
	\framesubtitle{\insertsubsection: Silhouette}
	
		\begin{itemize}
		\item $C$: Clusters' set
		\item $a_i$: the average of distances of the point $i$ from those of the same cluster
		\item $b_i$: the average of distances of the point $i$ from those of the closest cluster
		\item \expword{sklearn.metrics.silhouette\_score}
	\end{itemize}
	
	\[a_i = \frac{1}{|C_k| - 1} \sum_{j \in C_k,\; i \ne j} d(i, j) / \; i \in C_k\]
	\[b_i = \min_{i \notin C_k} \frac{1}{|C_k|} \sum_{j \in C_k} d(i, j)\]
	\[s_i = \frac{b_i - a_i}{max(a_i, b_i)}\]
	
	
\end{frame}

\begin{frame}
	\frametitle{\insertshortsubtitle: \insertsection}
	\framesubtitle{\insertsubsection: Some humor}
	
	\begin{center}
		\vgraphpage{humour-eval1.png}
		\vgraphpage{humour-eval2.jpg}
	\end{center}
	
\end{frame}

\insertbibliography{ML_Lect-Preparation}{*}

\end{document}




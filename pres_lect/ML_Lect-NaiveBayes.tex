% !TEX TS-program = pdflatex
% !TeX program = pdflatex
% !TEX encoding = UTF-8
% !TeX spellcheck = en_US

\documentclass[xcolor=table]{beamer}

\usepackage{../extra/beamer/karimml}

\def\accyear{2024-2025}
\institute{
	Laboratoire de la Communication dans les Systèmes Informatiques (LCSI)\\
	École  nationale Supérieure d'Informatique (ESI, ex. INI), Algiers, Algeria
}

\author[Abdelkrime Aries (\accyear)] %
{Abdelkrime Aries}

\titlegraphic{%
	\includegraphics[height=1cm]{../extra/logo/esi-logo.png}%
	\hspace*{2cm}%
	\includegraphics[height=1cm]{../extra/logo/lcsi-logo.png}%
	\hspace*{2cm}%
	\includegraphics[height=1cm]{../extra/logo/esi.ml.pdf}%
}%\hspace*{4.75cm}~

\date{Academic year: \accyear} %\today

\title[ML]{Machine Learning} 

\usefolder{../extra/beamer}

\usetheme{Karimml} % Antibes Boadilla Warsaw

\beamertemplatenavigationsymbolsempty

\usepackage{longfbox}

\newfboxstyle{kared}{padding=1pt,margin=1pt,border-radius=1ex,border-color=red}
\newfboxstyle{kgboxx}{padding=1pt,margin=1pt,border-radius=1ex,border-color=green}


\subtitle[Naive Bayes]{Naive Bayes}  

\changegraphpath{../img/naive-bayes/}

\begin{document}

\begin{frame}
	\frametitle{\inserttitle}
	\framesubtitle{\insertshortsubtitle: Bayes theory}
	\huge
	\[
	\overbrace{P(A|B)}^\text{Posterior} = \frac{\overbrace{P(A)}^\text{Prior} \overbrace{P(B|A)}^{\text{Likelihood}}}{\underbrace{P(B)}_\text{Evidence}}
	\]
\end{frame}

\begin{frame}
	\frametitle{\inserttitle}
	\framesubtitle{\insertshortsubtitle: Plan}

	\begin{multicols}{2}
		%	\small
		\tableofcontents
	\end{multicols}
\end{frame}

\section{Classification models}

\begin{frame}
	\frametitle{\insertshortsubtitle}
	\framesubtitle{\insertsection}
	
	
	
\end{frame}

\subsection{Discriminative models}

\begin{frame}
	\frametitle{\insertshortsubtitle: \insertsection}
	\framesubtitle{\insertsubsection}
	
	\begin{itemize}
		\item The model learns the conditional probability $P(Y|X=x)$
		\item it distinguishes decision boundaries 
		\item Examples of classifiers
		\begin{itemize}
			\item Logistic regression
			\item SVM
			\item Decision trees
			\item Neural network
			\item...
		\end{itemize}
	\end{itemize}
	
\end{frame}

\subsection{Generative models}

\begin{frame}
	\frametitle{\insertshortsubtitle: \insertsection}
	\framesubtitle{\insertsubsection}

	\begin{itemize}
		\item The model learns the conditional probability $P(X|Y=y)$
		\item it learns the distribution of individual classes
		\item Examples of classifiers
		\begin{itemize}
			\item \optword{Naive Bayes}
			\item Hidden Markov Model
			\item Variational Autoencoder
			\item...
		\end{itemize}
	\end{itemize}
	
\end{frame}

\section{Foundation}

\begin{frame}
	\frametitle{\insertshortsubtitle}
	\framesubtitle{\insertsection}
	
	\begin{itemize}
		\item Given a sample $ x $, the probability of generating a class $ k $ can be expressed as:
		\[p(y=k|x) = \frac{p(y=k) p(x|y=k)}{p(x)}\]
		\item Given $ L $ classes, the output class is the one that maximizes this probability
		\[\hat{y} = \arg\max_{k} p(y=k|x),\ k \in \{1, \cdots, L\} \]
		\item in this case, no need for Evidence probability (not dependent to $ y $)
		\[p(y=k|x) \propto p(y=k) p(x|y=k)\]
	\end{itemize}
	
\end{frame}

\subsection{Estimation}

\begin{frame}
	\frametitle{\insertshortsubtitle: \insertsection}
	\framesubtitle{\insertsubsection}
	
	\begin{itemize}
		\item The output class $ \hat{y} $ is estimated as
		\[\hat{y} = \arg\max_{k} p(y=k|x) = \arg\max_{k} p(y=k) p(x|y=k),\ k \in \{1, \cdots, L\} \]
		
		\item \textbf{The naive part}: the assumption of features independence
		\[p(x|y=k) \approx \prod_{j=1}^{N} P(x_j|y=k)\]
		
		\item In this case, the estimation function would be:
		\[\hat{y} = \arg\max_{y=k} p(y=k) \prod_{j=1}^{N} P(x_j|y=k),\ k \in \{1, \cdots, L\}\]
		
		\item In practice, the calculation is simplified 
		\[\hat{y} = \arg\max_{y=k} \log p(y=k) + \sum_{j=1}^{N} \log p(x_j|y=k),\ k \in \{1, \cdots, L\}\]
	\end{itemize}
	
\end{frame}

\subsection{Training}

\begin{frame}
	\frametitle{\insertshortsubtitle: \insertsection}
	\framesubtitle{\insertsubsection: Prior probability}
	
	\[p(y=k) = \frac{|\{y^{(i)} = k,\ i \in \{1, \cdots, M\}\}|}{M}\]
	
	\begin{itemize}
		\item $|\{y^{(i)} = k,\ i \in \{1, \cdots, M\}\}|$ is the number of training samples having $k$ as class
		\item $M$ is the size of the training dataset
		\item If classes' distribution is uniform, this probability can be ignored 
		\item If we want to give the same prior probability to classes, this probability can be ignored
	\end{itemize}
	
\end{frame}


\begin{frame}
	\frametitle{\insertshortsubtitle: \insertsection}
	\framesubtitle{\insertsubsection: Likelihood (Multinomial distribution)}
	
	\[p(x_j = v|y=k) = \frac{|\{y^{(i)} = k \wedge x^{(i)}_j = v,\ i \in \{1, \cdots, M\}\}|}{|\{y^{(i)} = k,\ i \in \{1, \cdots, M\}\}|} = \frac{\#(y = k \wedge x_j = v)}{\#(y = k)}\]
	
	\begin{itemize}
		\item $x_j$ is a categorical feature having a value $ v $
		\item $v$ is a value among unique values $ V_j $ (called \keyword{vocabulary}) of the feature $j$
		\item $\#(y = k \wedge x_j = v)$ is the number of training samples with feature $j$ equals to $v$ and having $k$ as class
		\item $\#(y = k)$ is the number of training samples having $k$ as class
		\item Smoothing can be used in case there are unseen values $v$ in the test dataset
		\[P(x_j = v|y_k) = \frac{\#(y = k \wedge x_j = v) + \alpha}{\#(y = k) + \alpha |V_j|}\]
		\item \expword{sklearn.naive\_bayes.CategoricalNB}
		
	\end{itemize}
	
	
\end{frame}

\begin{frame}
	\frametitle{\insertshortsubtitle: \insertsection}
	\framesubtitle{\insertsubsection: Likelihood (Multinomial distribution - text)}
	
	\[p(word = w_j|y=k) = \frac{C_{jk} + \alpha}{C_k + \alpha |V|}\]
	\[\hat{y} = \arg\max_{k} p(y=k) * \prod_{w \in text} p(word = w|y=k)\]
	
	\begin{itemize}
		\item A text can be seen as one feature with words as values
		\item $C_k$ is the number of training samples having $k$ as class
		\item $C_{jk}$ is the number of occurrences of word $w_j$ in texts having $k$ as class
		\item $V$ is the vocabulary (unique words in the training dataset)
		\item \expword{sklearn.naive\_bayes.MultinomialNB}
	\end{itemize}
	
\end{frame}

\begin{frame}
	\frametitle{\insertshortsubtitle: \insertsection}
	\framesubtitle{\insertsubsection: Likelihood (Bernoulli distribution)}
	
	\[p(x_j=v|y=k) = p(x_j=1|y=k) v + (1-p(x_j=1|y=k)) (1-v)\]
	\[p(x_j=1|y_k) = \frac{|\{x_j^{(i)} = 1 \wedge y^{(i)} = k,\ i \in \{1, \cdots, M\}\}|}{|\{y^{(i)} = k,\ i \in \{1, \cdots, M\}\}|}\]
	
	\begin{itemize}
		\item $x_j$ is a boolean feature having a value $v \in \{0, 1\}$
		\item \expword{sklearn.naive\_bayes.BernoulliNB}
	\end{itemize}
	
	
\end{frame}

\begin{frame}
	\frametitle{\insertshortsubtitle: \insertsection}
	\framesubtitle{\insertsubsection: Likelihood (Normal distribution)}
	
	\[p(x_j = v|Y_k) = \frac{1}{\sqrt{2\pi \sigma_{kj}^2}} e^\frac{-(v-\mu_{kj})^2}{2 \sigma_{kj}^2}\]
	
	\begin{itemize}
		\item $x_j$ is a numerical feature having values $ v in ]-\infty, +\infty[ $
		\item $\mu_{kj}$ is the mean of $x_j$'s values having $k$ as class
		\item $\sigma_{kj}^2$ is the variance $x_j$'s values having $k$ as class
		\item \expword{sklearn.naive\_bayes.GaussianNB}
	\end{itemize}
	
	
\end{frame}

\section{Application}

\begin{frame}
	\frametitle{\insertshortsubtitle}
	\framesubtitle{\insertsection}
	
	
\end{frame}

\subsection{Multinomial distribution}

\begin{frame}
	\frametitle{\insertshortsubtitle: \insertsection}
	\framesubtitle{\insertsubsection: Example (1)}
	
	\begin{minipage}{0.35\textwidth}
		\scriptsize
		\SetTblrInner{rowsep=0pt,colsep=1pt}
		\begin{tblr}{
				colspec = {lllll},
				row{odd} = {lightblue},
				row{even} = {lightyellow},
				row{1} = {bg=darkblue, fg=white},
			} 
			%			\begin{tabular}{llllll}
				outlook & temp & humidity & windy & play \\
				sunny & hot & high & false & no \\
				sunny & hot & high & true & no \\
				overcast & hot & high & false & yes \\
				rainy & mild & high & false & yes \\
				rainy & cool & normal & false & yes \\
				rainy & cool & normal & true & no \\
				overcast & cool & normal & true & yes \\
				sunny & mild & high & false & no \\
				sunny & cool & normal & false & yes \\
				rainy & mild & normal & false & yes \\
				sunny & mild & normal & true & yes \\
				overcast & mild & high & true & yes \\
				overcast & hot & normal & false & yes \\
				rainy & mild & high & true & no \\
				%			\end{tabular}
		\end{tblr}
	\end{minipage}
	\begin{minipage}{0.64\textwidth}
		\begin{itemize}
			\item Prior probability
			\begin{itemize}
				\item $p(play=yes) = \frac{\#(play = yes)}{M} = \frac{9}{14}$
				\item $p(play=no) = \frac{\#(play = no)}{M} = \frac{5}{14}$
			\end{itemize}
			\item Likelihood probability of $ outlook=rainy $
			\begin{itemize}
				\item $p(outlook=rainy|play=yes) = \frac{\#(outlook=rainy \wedge play = yes)}{\#(play = yes)} = \frac{3}{9}$
				\item $p(outlook=rainy|play=no) = \frac{\#(outlook=rainy \wedge play = no)}{\#(play = yes)} = \frac{2}{5}$
			\end{itemize}
			\item Likelihood probability of $ temp=hot $
			\begin{itemize}
				\item $p(temp=hot|play=yes) = \frac{\#(temp=hot \wedge play = yes)}{\#(play = yes)} = \frac{2}{9}$
				\item $p(temp=hot|play=no) = \frac{\#(temp=hot \wedge play = no)}{\#(play = yes)} = \frac{2}{5}$
			\end{itemize}
		\end{itemize}
	\end{minipage}
	
\end{frame}

\begin{frame}
	\frametitle{\insertshortsubtitle: \insertsection}
	\framesubtitle{\insertsubsection: Example (2)}
	
	\begin{itemize}
		\item Likelihood probability of $ humidity=high $
		\begin{itemize}
			\item $p(humidity=high|play=yes) = \frac{\#(humidity=high \wedge play = yes)}{\#(play = yes)} = \frac{3}{9}$
			\item $p(humidity=high|play=no) = \frac{\#(humidity=high \wedge play = no)}{\#(play = yes)} = \frac{4}{5}$
		\end{itemize}
		\item Likelihood probability of $ windy=false $
		\begin{itemize}
			\item $p(windy=false|play=yes) = \frac{\#(windy=false \wedge play = yes)}{\#(play = yes)} = \frac{6}{9}$
			\item $p(windy=false|play=no) = \frac{\#(windy=false \wedge play = no)}{\#(play = yes)} = \frac{2}{5}$
		\end{itemize}
	\end{itemize}
	
	\vfill
	Given $ \vec{v} = [rainy, hot, high, false]$
	\begin{itemize}
		\item $p(play=yes|x=\vec{v}) \propto \frac{9}{14} (\frac{3}{9} \frac{2}{9} \frac{3}{9} \frac{6}{9}) = \frac{6}{567} \approx 0.0106$
		\item $p(play=no|x=\vec{v}) \propto \frac{5}{14} (\frac{2}{5} \frac{2}{5} \frac{4}{5} \frac{2}{5}) = \frac{16}{875} \approx 0.0183$
		\item $ \hat{y} = no $
	\end{itemize}
	
\end{frame}


\subsection{Bernoulli distribution}

\begin{frame}
	\frametitle{\insertshortsubtitle: \insertsection}
	\framesubtitle{\insertsubsection: Example (1)}
	
	\begin{minipage}{0.5\textwidth}
		\scriptsize
		\SetTblrInner{rowsep=0pt,colsep=1pt}
		\begin{tblr}{
				colspec = {p{.2\textwidth}p{.2\textwidth}p{.2\textwidth}p{.2\textwidth}},
				row{odd} = {lightblue, font=\small},
				row{even} = {lightyellow, font=\small},
				row{1} = {darkblue, font=\bfseries},
			}
			\textcolor{white}{confident} & \textcolor{white}{studied} & \textcolor{white}{sick} & \textcolor{white}{result} \\
			1 & 0 & 0 & fail \\
			1 & 0 & 1 & pass \\
			0 & 1 & 1 & fail \\
			0 & 1 & 0 & pass \\
			1 & 1 & 1 & pass \\
		\end{tblr}
	\end{minipage}
	\begin{minipage}{0.49\textwidth}
		\begin{itemize}
			\item Prior probability
			\begin{itemize}
				\item $p(pass) = \frac{\#(pass)}{M} = \frac{3}{5}$
				\item $p(fail) = \frac{\#(fail)}{M} = \frac{2}{5}$
			\end{itemize}
		\end{itemize}
	\end{minipage}

	\begin{itemize}
		\item Prior probability
	\end{itemize}

	\begin{minipage}{0.12\textwidth}\small
		\hfill
	\end{minipage}
	\begin{minipage}{0.43\textwidth}\small
		\begin{itemize}
			\item $p(confident|pass) = \frac{2}{3}$
			\item $p(studied|pass) = \frac{2}{3}$
			\item $p(sick|pass) = \frac{2}{3}$
		\end{itemize}
	\end{minipage}
	\begin{minipage}{0.43\textwidth}\small
		\begin{itemize}
			\item $p(confident|fail) = \frac{1}{2}$
			\item $p(studied|fail) = \frac{1}{2}$
			\item $p(sick|fail) = \frac{1}{2}$
		\end{itemize}
	\end{minipage}

	\vfill
	Given $ \vec{v} = [1, 0, 0]$
	\begin{itemize}
		\item $p(pass|\vec{v}) \propto \frac{3}{5} [\frac{2}{3} (1-\frac{2}{3}) (1-\frac{2}{3})]  = \frac{2}{45} \approx 0.0444$
		\item $p(fail|\vec{v}) \propto \frac{2}{5} [\frac{1}{2} (1-\frac{1}{2}) (1-\frac{1}{2})]  = \frac{1}{20} \approx 0.05$
		\item $ \hat{y} = fail $
	\end{itemize}
	
	
\end{frame}


\subsection{Normal distribution}

\begin{frame}
	\frametitle{\insertshortsubtitle: \insertsection}
	\framesubtitle{\insertsubsection: Example}
	
	\begin{minipage}{0.35\textwidth}
		\scriptsize
		\SetTblrInner{rowsep=1pt,colsep=1pt}
		\begin{tblr}{
				colspec = {llll},
				row{odd} = {lightblue, font=\small},
				row{even} = {lightyellow, font=\small},
				row{1} = {darkblue, font=\bfseries},
			}
			\textcolor{white}{height} & \textcolor{white}{weight} & \textcolor{white}{footsize} & \textcolor{white}{person} \\
			182 & 81.6 & 30 & male\\
			180 & 86.2 & 28 & male\\
			170 & 77.1 & 30 & male\\
			180 & 74.8 & 25 & male\\
			152 & 45.4 & 15 & female\\
			168 & 68.0 & 20 & female\\
			165 & 59.0 & 18 & female\\
			175 & 68.0 & 23 & female\\
		\end{tblr}
	\end{minipage}
	\begin{minipage}{0.64\textwidth}
		\begin{itemize}
			\item Prior probability: no need since the classes distribution is uniform
		\end{itemize}
		\scriptsize\SetTblrInner{rowsep=2pt,colsep=3pt}
		\begin{tblr}{
				colspec = {ccccccc},
				row{odd} = {lightblue, font=\small},
				row{even} = {lightyellow, font=\small},
				row{1,2} = {darkblue, font=\bfseries},
				cell{1}{1} = {r=2}{c},
				cell{1}{2, 4, 6} = {c=2}{c},
			}
			\textcolor{white}{person} & \textcolor{white}{height} & & \textcolor{white}{weight} & & \textcolor{white}{footsize} & \\
			& \textcolor{white}{$ \mu $} & \textcolor{white}{$ \sigma^2 $} & \textcolor{white}{$ \mu $} & \textcolor{white}{$ \sigma^2 $} &
			\textcolor{white}{$ \mu $} & \textcolor{white}{$ \sigma^2 $} \\
			male & 178 & 29.33 & 79.92 & 25.48 & 28.25 & 5.58\\
			female & 165 & 92.67 & 60.1 & 114.04 & 19 & 11.33\\
		\end{tblr}
	\end{minipage}
	
	\vfill
	Given $ \vec{v} = [183, 59, 20]$
	\begin{itemize}
		\item $ p(height=183|male) = \frac{1}{\sqrt{2\pi * 29.33}} e^\frac{-(183-165)^2}{2 * 29.33} $
		\item $ p(height=183|female) = \frac{1}{\sqrt{2\pi * 92.67}} e^\frac{-(183-165)^2}{2 * 92.67} $
	\end{itemize}
	
	
\end{frame}

%\insertbibliography{ML-RN}{*}

\end{document}


% !TEX TS-program = pdflatex
% !TeX program = pdflatex
% !TEX encoding = UTF-8
% !TEX spellcheck = fr

\documentclass[11pt, a4paper]{article}
%\usepackage{fullpage}
\usepackage[left=1cm,right=1cm,top=1cm,bottom=2cm]{geometry}
\usepackage[fleqn]{amsmath}
\usepackage{amssymb}
%\usepackage{indentfirst}
\usepackage[T1]{fontenc}
\usepackage[utf8]{inputenc}
\usepackage[french,english]{babel}
\usepackage{txfonts} 
\usepackage[]{graphicx}
\usepackage{multirow}
\usepackage{hyperref}
\usepackage{parskip}
\usepackage{multicol}
\usepackage{wrapfig}

\usepackage{turnstile}%Induction symbole

\usepackage{tikz}
\usetikzlibrary{arrows, automata}
\usetikzlibrary{decorations.pathmorphing}

\renewcommand{\baselinestretch}{1}

\setlength{\parindent}{24pt}


\begin{document}

\selectlanguage {french}
%\pagestyle{empty} 

\noindent
\begin{tabular}{ll}
\multirow{3}{*}{\includegraphics[width=2cm]{../img/esi-logo.png}} & \'Ecole national Supérieure d'Informatique\\
& 2\textsuperscript{ème} année cycle supérieure (2CSSIL2, 2CSSIQ2, 2CSSIQ3)\\
& Machine Learning (2021-2022)
\end{tabular}\\[.25cm]
\noindent\rule{\textwidth}{1pt}\\%[-0.25cm]
\begin{center}
{\LARGE \textbf{T.P. Tutoriel : Outils}}
\begin{flushright}
	ARIES Abdelkrime
\end{flushright}
\end{center}
\noindent\rule{\textwidth}{1pt}

\section*{Exercice : Numpy}

%\vspace{-12pt}
%\begin{tabular}{|p{\textwidth}|}
%	\hline\\
%	Objectif : l'étudiant doit pouvoir : 
%	\begin{itemize}
%		\item utiliser numpy 
%	\end{itemize}\\
%	\hline
%\end{tabular}

On veut étudier la popularité des utilisateurs sur Twitter. 
Pour ce faire, on va étudier les interactions (Reply, Retweet, Like) entre 5 utilisateurs.
Le nombre des tweets de chaque utilisateur est représenté par le vecteur : [2, 5, 10, 4, 3].
On ne considère pas les réponses comme des tweets.
Le graphe suivant représente les interactions de la forme (Reply, Retweet, Like) entre les utilisateurs :
\begin{center}
	
\begin{tikzpicture}[
> = stealth, % arrow head style
shorten > = 1pt, % don't touch arrow head to node
auto,
node distance = 3cm, % distance between nodes
semithick % line style
]

\tikzstyle{every state}=[
draw = black,
thick,
fill = white,
minimum size = 4mm
]

\node[state] (u0) {0};
\node[state] (u1) [right of=u0, xshift=3cm] {1};
\node[state] (u4) [below of=u1] {4};
\node[state] (u2) [right of=u1, xshift=3cm] {2};
\node[state] (u3) [below of=u4] {3};

\path[->] 	
(u0) 	edge [bend right] 	node {(3, 0, 3)} (u4)
        edge [loop left] 	node {(1, 0, 2)} ()
        edge [] 	node {(5, 2, 5)} (u1)
		edge [bend left] 	node {(5, 4, 1)} (u2)
(u1) 	edge [bend left] 	node {(2, 0, 0)} (u0)
		edge [bend left] 	node {(3, 0, 0)} (u4)
		edge [bend left] 	node {(5, 0, 0)} (u2)
(u2) 	edge [] 	node {(0, 0, 5)} (u1)
		edge [bend left] 	node {(0, 0, 3)} (u4)
		edge [bend left] 	node {(0, 0, 3)} (u3)
(u3) 	edge [] 	node {(0, 3, 1)} (u4)
		edge [bend left] 	node {(2, 0, 1)} (u0)
		edge [loop right] 	node {(4, 4, 4)} ()
(u4) 	edge [bend left] 	node {(4, 2, 0)} (u1)
		edge [] 	node {(2, 5, 9)} (u2)
;

\end{tikzpicture}

\end{center}

\begin{itemize}
	\item Représenter les interactions sous forme de trois matrices 5X5 pour chaque type d'interaction.  
\end{itemize}

\subsection*{Statistiques d'engagement}

Ici, on veut extraire quelques statistiques sur l'engagement des utilisateurs envers les autres.
Dans la théorie des graphes, on s'intéresse par les arcs sortants.

\begin{itemize}
	\item Trouver l'utilisateur qui répond (Reply) plus que les autres
	\item Trouver l'utilisateur qui partage (Retweet) plus que les autres
	\item Trouver les utilisateurs qui ne répondent pas (Reply) et qui ne partagent pas (Retweet)
	\item Afficher le nombre des Likes que chaque utilisateur fait pour ses propres tweets
	\item Afficher le nombre des Likes que chaque utilisateur fait pour les tweets autre que les tien
	\item Trouver les utilisateurs qui aiment leurs propres tweets plus qu'ils aiment les tweets des autres
	\item Calculer leur nombre
	\item Afficher le nombre des interactions que chaque utilisateur a fait
\end{itemize}

\subsection*{Statistiques sue les interactions}

\begin{enumerate}
	\item Afficher le max des Likes que chaque utilisateur a reçu
	\item Trouver les utilisateurs qui ont des likes pour tous leurs tweets au moins par une personne
	\item Calculer le nombre des arcs entrants (ayant au moins une interaction)
	\item Calculer le nombre des arcs sortants (qui ont fait au moins une interaction)
	\item Trouver les nœuds où le nombre des arcs entrants est égale au nombre de ceux sortants
	\item Afficher les Likes qu'ils ont fait entre eux
\end{enumerate}

\subsection*{Popularité}

\begin{itemize}
	\item On ne s'intéresse pas par les interactions réflexives (Ex. quelqu'un qui aime ses propres tweets) qui sont représentées par la diagonale. 
	Pour annuler la diagonale, on peut créer une matrice diagonale 10X10 et inverser les valeurs.
	Ensuite, appliquer une multiplication élément par élément.
	\item On veut pondérer les matrices à tel sorte que :
	\begin{itemize}
		\item Reply : on suppose que les Reply sont des objections en général ; donc, ils ont un effet négatif. 
		Pour chaque interaction, si le nombre des Reply reçus d'une personne est plus que celui des Likes, on le multiplie par (-0.5). S'il est égale, on le multiplie par 0. \textbf{HINT : utiliser numpy.where}
		\item Retweet : on suppose que les Retweet sont neutres en général ; donc, on ne fait aucune transformation.
		\item Likes : on suppose que les Likes ont un effet positif sur la popularité.
		Pour chaque interaction, si le nombre des Likes reçus d'une personne est plus que la moitié des tweets publiées, on le multiplie par (1.5).
	\end{itemize}
	\item 

\end{itemize}

\end{document}

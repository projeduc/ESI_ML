% !TEX TS-program = pdflatex
% !TeX program = pdflatex
% !TEX encoding = UTF-8
% !TEX spellcheck = fr

\documentclass[11pt, a4paper]{article}
%\usepackage{fullpage}
\usepackage[left=1cm,right=1cm,top=1cm,bottom=2cm]{geometry}
\usepackage[fleqn]{amsmath}
\usepackage{amssymb}
%\usepackage{indentfirst}
\usepackage[T1]{fontenc}
\usepackage[utf8]{inputenc}
\usepackage[french,english]{babel}
\usepackage{txfonts} 
\usepackage[]{graphicx}
\usepackage{multirow}
\usepackage{hyperref}
\usepackage{parskip}
\usepackage{multicol}
\usepackage{wrapfig}

\usepackage{turnstile}%Induction symbole

\usepackage{tikz}
\usetikzlibrary{arrows, automata}
\usetikzlibrary{decorations.pathmorphing}

\renewcommand{\baselinestretch}{1}

\setlength{\parindent}{24pt}


\begin{document}

\selectlanguage {french}
%\pagestyle{empty} 

\noindent
\begin{tabular}{ll}
\multirow{3}{*}{\includegraphics[width=2cm]{../img/esi-logo.png}} & \'Ecole national Supérieure d'Informatique\\
& 2\textsuperscript{ème} année cycle supérieure (2CSSIL2, 2CSSIQ2, 2CSSIQ3)\\
& Machine Learning (2021-2022)
\end{tabular}\\[.25cm]
\noindent\rule{\textwidth}{1pt}\\%[-0.25cm]
\begin{center}
{\LARGE \textbf{Lab : Réseaux de neurones (auto-encodeurs et CNN)}}
\begin{flushright}
	ARIES Abdelkrime
\end{flushright}
\end{center}
\noindent\rule{\textwidth}{1pt}

\section*{Information}

\begin{itemize}
	\item Dataset : Pokemon Image Dataset : {\scriptsize\url{https://www.kaggle.com/datasets/vishalsubbiah/pokemon-images-and-types}}
	\item Il n'y a pas des spécifications pour construire les architectures : l'étudiant est libre à utiliser les composants vus dans le tutoriel : {\scriptsize \url{https://github.com/projeduc/ESI_2CS_ML/tree/main/tuto/NN}}
	\item Le pré-traitement est fourni avec ce lab (il faut renommer le dossier comme "pokemon" ou changer le chemin du dataset dans le code fourni)
\end{itemize}

\section*{Classement CNN}

Concevoir un modèle qui classe les pokémons selon leurs type selon les spécifications : 
\begin{itemize}
	\item Utiliser les CNNs (pas un simple réseau de neurones à propagation avant)
	\item Entraîner le modèle avec validation en calculant l'accuracy
	\item Dessiner l'évolution des coûts et des accuracy
\end{itemize}

\section*{Auto-encodeur de clustering}

Concevoir un modèle qui apprend la codification des pokémons en suivant les étapes : 
\begin{itemize}
	\item Le code est un vecteur de 10 à 50 éléments (au choix)
	\item L'auto-encodeur peut être complètement FFNN ou en utilisant des CNN
	\item Il faut séparer l'encodeur et le décodeur afin de les ré-utiliser
	\item Générer les vecteurs représentant chaque échantillon de l'entraînement afin de construire un autre dataset d'entraînement
	\item Appliquer K-Means sur ce nouveau dataset avec K est le nombre des classes réelles
	\item Pour chaque cluster généré, nous supposons que la classe majoritaire est celle représentée par ce cluster.
	\item Calculer l'indice de Rand : \textbf{sklearn.metrics.rand\_score}
\end{itemize}

\section*{Auto-encodeur de clustering pour classement}

Concevoir un modèle qui prend le code généré par l'auto-encodeur précédent et classe le pokémon selon son type : 
\begin{itemize}
	\item Il faut utiliser l'encodeur entraîné précédemment ; c-à-d. L'entrée doit être une image et pas le vecteur de son code
	\item Le modèle est un simple FFNN (obviously!) qui prend la sortie de l'encodeur comme entrée et qui génère le type du pokémon
	\item Lors de l'entraînement du classeur, il ne faut pas mettre à jours l'encodeur ; il faut désactiver l'entraînement du modèle : \textbf{model.trainable = False} 
	\item Comparer entre ce modèle et celui du classement CNN 
\end{itemize}

\section*{Génération des images}

Concevoir un modèle qui génère des nouveaux pokémons en suivant les spécifications suivantes : 
\begin{itemize}
	\item Le code est un vecteur de 10 à 50 éléments (au choix)
	\item Deux architectures sont à considérer au choix : Auto-encodeur variationnelle ou GAN
	\item De préférence, utiliser des CNN
	\item De préférence, utiliser la classe du pokémon afin d'apprendre la génération. C-à-d, l'entrée du décodeur doit être un code aléatoire fusionné avec le code one-hot de la classe.
\end{itemize}


\end{document}

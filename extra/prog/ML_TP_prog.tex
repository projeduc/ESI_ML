% !TEX TS-program = pdflatex
% !TeX program = pdflatex
% !TEX encoding = UTF-8
% !TEX spellcheck = fr

\documentclass[11pt, a4paper]{article}
%\usepackage{fullpage}
\usepackage[left=1cm,right=1cm,top=1cm,bottom=2cm]{geometry}
\usepackage[fleqn]{amsmath}
\usepackage{amssymb}
%\usepackage{indentfirst}
\usepackage[T1]{fontenc}
\usepackage[utf8]{inputenc}
\usepackage[french,english]{babel}
\usepackage{txfonts} 
\usepackage[]{graphicx}
\usepackage{multirow}
\usepackage{hyperref}
\usepackage{parskip}
\usepackage{multicol}
\usepackage{wrapfig}
\usepackage{fancyhdr}
\usepackage{hyperref}
\usepackage{tcolorbox}
\usepackage{natbib}

\usepackage{turnstile}%Induction symbole

\renewcommand{\baselinestretch}{1}

\setlength{\parindent}{24pt}

%\usepackage{etoolbox}
%\patchcmd{\thebibliography}{\section*{\refname}}{}{}{}

\fancyhf{}

\lfoot{ARIES Abdelkrime}
\cfoot{ESI/ML/SIQ2-SIL2  (2022-2023)}
\rfoot{\textbf{\thepage}}

\renewcommand{\headrulewidth}{0pt} % remove lines as well
\renewcommand{\footrulewidth}{1pt}

\newcommand\repeatstr[1]{\leavevmode\xleaders\hbox{#1}\hfill\kern0pt}

\renewcommand{\bibsection}{}
%\bibliographystyle{humannat}

\begin{document}

\selectlanguage {french}
%\pagestyle{empty} 
\pagestyle{fancy}

\noindent
\begin{tabular}{ll}
\multirow{3}{*}{\includegraphics[width=1.5cm]{../logo/esi.ml.pdf}} & \'Ecole national Supérieure d'Informatique\\
& 2ème année cycle supérieur (SIQ2/SIL2)\\
& Machine Learning (2023-2024)
\end{tabular}\\[2pt]
\noindent\rule{\textwidth}{1pt}\\[-0.5cm]
\begin{center}
{\LARGE \textbf{TP : Programme}}
\begin{flushright}
	ARIES Abdelkrime
\end{flushright}
\end{center}\vspace{-0.5cm}
\noindent\rule{\textwidth}{1pt}

\begin{abstract}
	Machine learning (ML, en français : Apprentissage automatique) fait partie des méthodes de l'intelligence artificielle.
	Il se base sur les approches mathématiques et statistiques afin de permettre à une machine d'apprendre à partir de certaines données. 
	Ces dernières jouent un grand rôle dans la réussite d'un modèle ML.
	
	Certes, la préparation des données est une grande étape dans la conception d'un système ML.
	Mais, la compréhension des algorithmes ML est aussi importante.
	Dans ces TPs, nous allons apprendre comment ces algorithmes fonctionnent en essayant d'implémenter des versions simples.
	Aussi, nous testerons quelques paramètres afin de comprendre leurs effets.
	
	Dans la séance du TPs, nous allons avoir deux types des TPs : TPs notés et Workshops.
	Les TPs notés ont comme but de comprendre les concepts vus en cours.
	Ils sont divisés sur deux parties : Réalisation et Analyse.
	La première partie est une implémentation des algorithmes de zéro (from scratch).
	Le code complet est implémenté ensuite une partie est supprimée afin de la remplir (gap filling).
	La deuxième partie exploite des APIs existantes afin de faire des tests sur les différents paramètres d'un algorithme.
	Dans cette partie, l'étudiant doit analyser les résultats du test (pourquoi avons-nous eu ces résultats ?).
	
	Les Workshops sont des TPs très guidés : étape par étape. 
	Ils ne sont pas notés, mais restent importants.
	Leur but est d'apprendre un outil ou une architecture comme par exemple les GANs.
	Des démos sont données afin de guider ces workshops.
	Voir ce lien pour les démos : \url{https://github.com/projeduc/ESI_ML/tree/main/demos}.
\end{abstract}

\section{Informations générales}

\begin{minipage}{0.49\textwidth}
	
\subsection{Réception}

S'il y a des questions concernant le TP 
\begin{itemize}
	\item Bureau : Labo LCSI (bloc à côté de la DG, premier étage), à gauche, bureau 68.
	\item Date : Chaque mardi de 13h30 jusqu'à 16h30.
\end{itemize}

\subsection{Pré-requis}

\begin{itemize}
	\item Probabilités et statistiques
	\item Algèbre et analyse 
	\item Programmation (Python avec Jupyter notebook)
\end{itemize}
\end{minipage}
\begin{minipage}{0.49\textwidth}
\subsection{Objectifs}

\begin{itemize}
	\item Appliquer des notions mathématiques apprises au long du cursus sur des problèmes réels.
	\item Implémenter quelques algorithmes ML afin de comprendre leur fonctionnement.
	\item Découvrir quelques outils et ressources du ML.
	\item Apprendre à mener des tests et à analyser les résultats (Aspect synthèse).
\end{itemize}

\end{minipage}

\subsection{Outils}

\begin{itemize}
	\item \href{https://jupyter.org}{Jupyter Notebook} : Les TPs seront présentés sous forme d'un fichier Jupyter Notebook afin d'intégrer la documentation avec le code Python.
	\item \href{https://pandas.pydata.org/}{pandas} : Un API d'analyse et manipulation de données
	\item \href{https://scikit-learn.org/}{scikit-learn} : Un API utilisé pour le prétraitement et l'apprentissage automatique (différents algorithmes)
	\item \href{https://www.tensorflow.org/}{Tensorflow} : Un API utilisé pour les réseaux de neurones. Ce n'ai pas la peine de l'installer pour l'instant.
\end{itemize}

Les trois premiers outils peuvent être installés en plus de Python en installant \href{https://www.anaconda.com/products/individual#Downloads}{Anaconda}.
Sur Linux, ces outils peuvent être installés en utilisant l'outil \textbf{pip} (l'outil d'installation des packages python).
Il est, aussi, possible d'utiliser Google Colab. 
Mais, c'est préféré d'utiliser les machines personnelles puisque les TPs ne nécessitent pas une grande quantité de ressources (mémoire, puissance de calcul). 


\section{Organisation de la séance}

\begin{itemize}
	\item \textbf{Mercredi matin} : l'énoncé du TP sera partagé sur \textit{Google Classroom}.
	
	\item \textbf{20mn de la séance} : le TP passé sera discuté.
	\begin{itemize}
		\item La deuxième partie seulement.
		\item Il faut participer à la discussion même si la réponse est fausse.
		\item Il faut suivre la discussion afin de comprendre les notions du TP passé.
		\item Aucun corrigé-type ne sera fourni. Ce TP se base seulement sur la discussion.
	\end{itemize}

	\item \textbf{10mn de la séance} : le nouveau TP sera expliqué.
	\begin{itemize}
		\item Il faut suivre l'explication afin de comprendre ce qu'il faut faire.
	\end{itemize}
	
	\item \textbf{1h20mn de la séance} : le nouveau TP sera fait par les étudiants.
	\begin{itemize}
		\item Le TP se fait en binôme ou en monôme.
		\item Pour chaque nouveau TP, il est admis de changer le binôme.
		\item Durant la séance, il faut appeler l'enseignant en cas d'ambiguïté ou au cas où on stagne sur un code.
	\end{itemize}

	\item \textbf{10mn de la séance} : l'état d'avancement sera vérifié.
	\begin{itemize}
		\item La première partie du TP doit être complétée durant la séance.
		\item L'avancement est noté.
	\end{itemize}

	\item \textbf{mardi soir} : la version finale du TP doit être rendue sur \textit{Google Classroom}.
	\begin{itemize}
		\item Il faut rendre le TP avant la date limite.
		\item En cas d'un binôme, un seul étudiant doit rendre le TP (pas les deux).
		\item On peut rendre plusieurs versions du TP (seule la dernière sera prise en compte).
		\item Il faut seulement rendre le fichier \textbf{.ipynb} sans les données.
		\item Le noms du fichier doit être sous format \textbf{TPXX\_NOM1\_NOM2\_vY.ipynb}
	\end{itemize}

\end{itemize}


\section{Plan des TPs}

Un planning est partagé sur Google Drive. 
Il faut le vérifier à chaque fois au cas où il y aurait des modifications.


\section{Notation et plagiat}

\begin{itemize}
	\item Chaque TP est noté sur 64/80 (16/20). La distribution des notes sur la partie 1 et 2 est selon la nature du TP.
	\item L'avancement est noté sur 8/80 (2/20). 
	Un étudiant qui n'a pas assisté à la séance aura 0 dans l'avancement même si son binôme a assisté.
	\item Rendre le TP au temps aura 8/80 (2/20). 
	La date limite est le mardi 23:59 (sauf si changé dans classroom).
	Après 30 minutes de cette limite, cette note sera 0.
	Dire que c'est un problème de connexion n'est pas accepté. 
	Vous avez eu une semaine pour rendre le TP.
	Attendre jusqu'à la dernière minute est une mauvaise gestion de votre part ; vous devez être sanctionné pour ça. 
	\item Une fois le TP est discuté (la première séance entre les deux groupes), les TPs envoyés auront seulement la note d'avancement plus le pourcentage fait en classe.
	Si un étudiant a été absent durant la séance, il aura un zéro (La partie faite dans la classe est celle de son binôme seulement).
	\item Il est possible d'utiliser des ressources externes (livres, sites, etc.) pour répondre aux questions. 
	Mais, il ne faut pas copier la réponse telle qu'elle est (verbatim).
	Il faut tout d'abord comprendre la question, ensuite répondre avec votre expression. 
	\item Toutes réponses similaires (partie 2) seront sanctionnées
	\begin{itemize}
		\item Si le fichier contient des noms d'autre binôme, les deux équipes auront un zéros même si le contenu est différent.
		\item Si deux (ou plus) équipes répondent sur une question de la même façon, ils auront un 0 pour cette question.
		Rendre une phrase à la voie passive ou remplacer un mot par son synonyme reste toujours la même réponse.
		\item Aucune de ces raisons ne sera acceptée : "\textit{J'ai envoyé le mauvais fichier}" (le fait d'avoir la solution d'une autre équipe est déjà une erreur), "\textit{Mon binôme a envoyé la solution à une autre équipe sans que je le sache}" (Vous avez choisi votre binôme, donc il faut assumer), "\textit{Je les ai envoyé la solution pour qu'ils puissent comprendre ce qu'il faut faire}" (Ils comprennent en assistant au TP ; Rester à la maison et attendre les solutions de ces collègues n'a jamais été une méthode d'apprentissage), "\textit{Nous avons utilisé la même ressource.}" (Il faut rédiger! Recopier la réponse telle qu'elle est veut dire que vous n'avez rien compris), "\textit{Je vous en supplie !}" (Vous allez avoir un zéro parmi plusieurs TPs, donc ce n'est pas grave. Aussi, en lisant ceci, vous connaissez le risque. ), etc.
	\end{itemize}
	\item S'il y a une réclamation à faire, il faut le faire dans une semaine de l'attribution de la note. 
	En général, les TP seront corrigés le samedi.
	
\end{itemize}

\section{Bibliographie}

Ceci est le répo Github officiel du TP-ML : \url{https://github.com/projeduc/ESI_ML}

\nocite{*}

\bibliographystyle{apalike}

\bibliography{ML_TP_biblio}

\end{document}

% !TEX TS-program = pdflatex
% !TeX program = pdflatex
% !TEX encoding = UTF-8
% !TeX spellcheck = en_US

\documentclass[xcolor=table]{beamer}

\usepackage{../extra/beamer/karimml}

\def\accyear{2024-2025}
\institute{
	Laboratoire de la Communication dans les Systèmes Informatiques (LCSI)\\
	École  nationale Supérieure d'Informatique (ESI, ex. INI), Algiers, Algeria
}

\author[Abdelkrime Aries (\accyear)] %
{Abdelkrime Aries}

\titlegraphic{%
	\includegraphics[height=1cm]{../extra/logo/esi-logo.png}%
	\hspace*{2cm}%
	\includegraphics[height=1cm]{../extra/logo/lcsi-logo.png}%
	\hspace*{2cm}%
	\includegraphics[height=1cm]{../extra/logo/esi.ml.pdf}%
}%\hspace*{4.75cm}~

\date{Academic year: \accyear} %\today

\title[ML]{Machine Learning} 

\usefolder{../extra/beamer}

\usetheme{Karimml} % Antibes Boadilla Warsaw

\beamertemplatenavigationsymbolsempty

\usepackage{longfbox}

\newfboxstyle{kared}{padding=1pt,margin=1pt,border-radius=1ex,border-color=red}
\newfboxstyle{kgboxx}{padding=1pt,margin=1pt,border-radius=1ex,border-color=green}


\subtitle[SVM]{Support vector machine (SVM)} 

\changegraphpath{../img/svm/}

\begin{document}
	
\begin{frame}
	\frametitle{\inserttitle}
	\framesubtitle{\insertshortsubtitle: Introduction}
	
	\begin{itemize}
		\item We have already seen ...
		\begin{itemize}
			\item logistic regression looks for a linear separation between two classes
			\item it draws a hyperplane between them
			\item there could be many possible hyperplanes
		\end{itemize}
		\item But ...
		\begin{itemize}
			\item how about drawing a hyperplane which has the same distance from the two classes
		\end{itemize}

	\end{itemize}
\end{frame}


\begin{frame}
	\frametitle{\inserttitle}
	\framesubtitle{\insertshortsubtitle: Plan}
	
	\begin{multicols}{2}
		%	\small
		\tableofcontents
	\end{multicols}
\end{frame}

\section{Problem definition}

\begin{frame}
	\frametitle{\insertshortsubtitle}
	\framesubtitle{\insertsection}
	
	\begin{minipage}{0.75\textwidth} 
		\begin{itemize}
			\item in logistic regression, the hyperplane's equation is $ \sigma(z = b + \sum_{j=1}^{N} w_j x_j) = 0.5 $
			\item we want it to be $ z = b + \sum_{j=1}^{N} w_j x_j = 0 $
			\item in this case, $ y \in \{-1, +1\} $
			\item thus, $ z^{(i)} \ge 1 \Rightarrow \hat{y}^{(i)} = 1 $ and $ z^{(i)} \le -1 \Rightarrow \hat{y}^{(i)} = -1 $
		\end{itemize}
	\end{minipage}
	\begin{minipage}{0.24\textwidth} 
		\hgraphpage{SVM_primal.pdf}
	\end{minipage}
	
	\begin{itemize}
		\item the space between $ -1 $ and $ +1 $ is a margin which equals $ \frac{2}{||w||} $
		\item the idea is to maximize this margin, thus minimizing $ \frac{||w||^2}{2} $
	\end{itemize}
	
\end{frame}

\subsection{Hard-margin}


\begin{frame}
	\frametitle{\insertshortsubtitle: \insertsection}
	\framesubtitle{\insertsubsection}
	
	\begin{itemize}
		\item $ z^{(i)} \ge 1 \Rightarrow \hat{y}^{(i)} = 1 $ and $ z^{(i)} \le -1 \Rightarrow \hat{y}^{(i)} = -1 $
		\item So, $ \hat{y}^{(i)} = sign(z^{(i)}) $ where $ z^{(i)} \in ]-\infty, +\infty[ $
		\item we want, $ y^{(i)} =  \hat{y}^{(i)} $ where $ y^{(i)} \in \{-1, +1\} $ 
		\item in this case, $ y^{(i)} \hat{y}^{(i)} = 1 $, thus $ y^{(i)} z^{(i)} \ge 1 $
		\item the optimization problem will be formulated as:
		\begin{align*}
			& \min_w \frac{||w||^2}{2}  \\
			& \text{subject to } y^{(i)} z^{(i)} \ge 1,\ \forall i \in {1 \cdots M} \\
		\end{align*}
		
		\item in this case, no sample must be inside the margin; even the ones in the correct side
	\end{itemize}
	
\end{frame}

\subsection{Soft-margin}

\begin{frame}
	\frametitle{\insertshortsubtitle: \insertsection}
	\framesubtitle{\insertsubsection}
	
	\begin{itemize}
		\item when $y^{(i)} z^{(i)} < 1 $ there are two possible interpretations:
		\begin{itemize}
			\item $y^{(i)} z^{(i)} < 0 $ the sample is on the wrong side of the decision vector
			\item $y^{(i)} z^{(i)} \ge 0 $ it is on the correct side, but it is inside the margin 
		\end{itemize}
		\item to allow this second case, \keyword{hinge loss} is used:
		\[\zeta^{(i)} = \max (0, 1 - y^{(i)} z^{(i)})\]
		\item then, the goal will be to minimize this loss function:
		\[ \frac{||w||^2}{2}  + C \sum_{i=1}^{M} \zeta^{(i)} \]
		\item $ C $ is trad-off between increasing the margin size and having samples on the correct side
	\end{itemize}
	
\end{frame}


\section{Primal form}

\begin{frame}
	\frametitle{\insertshortsubtitle}
	\framesubtitle{\insertsection}
	
	\begin{minipage}{0.70\textwidth} 
		\begin{itemize}
			\item Draw a decision line between two classes $ y $ 
			\item Maximize the margin between the two classes.
			\item Use linear combination over features $ x $ like LR. 
			\item Based on the coordinates $ x $, a sample belongs to a given class if it is located on its side
		\end{itemize}
	\end{minipage}
	\begin{minipage}{0.29\textwidth} 
		\hgraphpage{SVM_primal.pdf}
	\end{minipage}
	

\end{frame}

\subsection{Cost function}

\begin{frame}
	\frametitle{\insertshortsubtitle: \insertsection}
	\framesubtitle{\insertsubsection}
	
	\begin{itemize}
		\item The loss function can be formulated as 
		\[J_w = \frac{||w||^2}{2}  +  C \sum_{i=1}^{M} \zeta^{(i)} \text{ where } \zeta^{(i)} = \max (0, 1 - y^{(i)} z^{(i)}) \]
		\item when $ C $ is ...
		\begin{itemize}
			\item big, having samples on the correct side is preferred over having a big margin
			\item small, having having a big margin is preferred over samples on the correct side
		\end{itemize}
	\end{itemize}
	
	\[\frac{\partial J_w}{\partial w_{j}} = w_j + C \sum\limits_{i=1}^{M} \frac{\partial \zeta^{(i)}}{\partial w_{j}} \text{ where }  
	\frac{\partial \zeta^{(i)}}{\partial w_{j}} =
	\begin{cases}
		0 & \text{if } y^{(i)} z^{(i)} \ge 1\\
		- x^{(i)}_j y^{(i)} & \text{otherwise}  \\
	\end{cases}\]
	
\end{frame}

\subsection{Class estimation}

\begin{frame}
	\frametitle{\insertshortsubtitle: \insertsection}
	\framesubtitle{\insertsubsection}
	
	\begin{itemize}
		\item Once the model trained, $ z^{(i)} = 0 $ will be the decision hyperplane
		\item we define the sign function as $ sign(x) = \begin{cases}
			1 & \text{if } x \ge 0\\
			-1 & \text{otherwise}\\
		\end{cases} $
		\item $ \hat{y} = sign(z)$
		\item $ z = b + \sum_{j=1}^{N} w_j x_j $
	\end{itemize}
	
	
\end{frame}

\subsection{Optimization algorithms}

\begin{frame}
	\frametitle{\insertshortsubtitle: \insertsection}
	\framesubtitle{\insertsubsection}
	
	\begin{itemize}
		\item Since the loss function is differentiable, \keyword{Gradient descent} can be used
	\end{itemize}
	
\end{frame}


\section{Dual form}

\begin{frame}
	\frametitle{\insertshortsubtitle}
	\framesubtitle{\insertsection}
	
	\begin{minipage}{0.70\textwidth} 
		\begin{itemize}
%			\item Draw a decision line between two classes $ y $ 
%			\item Maximize the margin between the two classes.
			\item Use similarity of the new sample with training samples. 
			\begin{itemize}
				\item If it is more similar to positive samples, then its class is positive
				\item If it is more similar to negative samples, then its class is negative
			\end{itemize}
			\item The similarities does not have the same weight
			\begin{itemize}
				\item If the new sample is similar to a training sample which is far from the other class, then its class is more likely to be similar
				\item If the new sample is similar to a training sample which is near the other class, then its class is less likely to be similar
			\end{itemize}
		\end{itemize}
	\end{minipage}
	\begin{minipage}{0.29\textwidth} 
		\hgraphpage{SVM_dual.pdf}
	\end{minipage}
	
	
\end{frame}


\subsection{Cost function}

\begin{frame}
	\frametitle{\insertshortsubtitle: \insertsection}
	\framesubtitle{\insertsubsection (1)}
	
	\begin{itemize}
		\item By deconstructing the hinge loss, the optimization problem will be :
		\begin{align*}
			& \min_{w, \zeta^{(i)}} \frac{||w||^2}{2} +  C \sum_{i=1}^{M} \zeta^{(i)} \\
			& \text{subject to } y^{(i)} z^{(i)} \ge 1 - \zeta^{(i)},\ \zeta^{(i)} \ge 0,\ \forall i \in {1 \cdots M} \\
		\end{align*}
		\item where $ \zeta^{(i)} = \max (0, 1 - y^{(i)} z^{(i)}) $
	\end{itemize}

\end{frame}

\begin{frame}
	\frametitle{\insertshortsubtitle: \insertsection}
	\framesubtitle{\insertsubsection (2)}
	
	\begin{itemize}
		\item By solving for the Lagrangian dual:
		\begin{align*}
			& \max_{\lambda_i}  \sum\limits_{i=1}^{M} \lambda_i - \frac{1}{2} \sum\limits_{i=1}^{M} \sum\limits_{j=1}^{M} \lambda_i \lambda_j y^{(i)} y^{(j)} x^{(i)} x^{(j)} \\
			& \text{subject to } \sum\limits_{i=1}^{M} \lambda_i y^{(i)} = 0,\ 0 \le \lambda_i \le C,\ \forall i \in {1 \cdots M} \\
		\end{align*}
		\item $ x^{(i)} x^{(j)} $ can be seen as a similarity measure called dot product
		\item We can use other similarity measures $ K(x^{(i)}, x^{(j)}) $ 
%		\[J_\alpha = \sum\limits_{i=1}^{M} \alpha_i - \frac{1}{2} \sum\limits_{i=1}^{M} \sum\limits_{j=1}^{M} \alpha_i \alpha_j y^{(i)} y^{(j)} K(x^{(i)}, x^{(j)})\]
		\item $ K(x^{(i)}, x^{(j)}) $ is called \keyword{kernel}
	\end{itemize}
	
\end{frame}

\subsection{Class estimation}

\begin{frame}
	\frametitle{\insertshortsubtitle: \insertsection}
	\framesubtitle{\insertsubsection}
	
	\[\hat{y_t} = sign(b + \sum^M_{i=1} \lambda_i y^{(i)} K(x^{(i)}, x_t))\]
	\begin{itemize}
		\item $ \hat{y_t} $ is the estimated class of the given test sample $ x_t $
		\item $ x^{(i)} $ are training samples
		\item $ K(a, b) $ is a kernel function
		\begin{itemize}
			\item Linear kernel $ K(A, B) = A \cdot B^T$
			\item RBF kernel $ K(A, B) = \exp(-\frac{||A-B||^2}{2\sigma})$
		\end{itemize}
	\end{itemize}
	
\end{frame}

\subsection{Optimization algorithms}

\begin{frame}
	\frametitle{\insertshortsubtitle: \insertsection}
	\framesubtitle{\insertsubsection}
	
	\begin{itemize}
		\item The problem can be solved using Quadratic programming
		\item One optimization algorithm is \keyword{Sequential minimal optimization} \cite{1998-platt}
	\end{itemize}

	
\end{frame}



\insertbibliography{ML_Lab-SVM}{*}


\begin{frame}[plain]
	
	\begin{center}
		\Huge 
		The margin has been reached  
		
		stop scrolling
	\end{center}
	
\end{frame}

\end{document}



